\subsection{Центральная предельная теорема для независимых одинаково распределённых случайных величин с конечной дисперсией.}

\Remind

$\varphi_\xi(t) = \mathbb{E}e^{it\xi}$ назовем \textbf{характеристической функцией} случайной величины $\xi$.

\Example

Посчитаем для $N(0, 1)$:
$$
f_\xi(x) = \frac{1}{\sqrt{2\pi}} \exp\left(-\frac{x^2}{2} \right)
$$
$$
\varphi_\xi(t) = \int_{-\infty}^{+\infty} e^{itx} f_\xi(x) dx = \int_{-\infty}^{+\infty} e^{itx} \frac{1}{\sqrt{2\pi}} \exp\left(-\frac{x^2}{2}\right) dx
$$
$$
-\frac{x^2}{2} + itx = -\frac{1}{2}(x^2 - 2itx) = -\frac{1}{2}(x - it)^2 - \frac{t^2}{2}
$$
$$
\varphi_\xi(t) = \exp\left( -\frac{t^2}{2} \right) \int_{-\infty}^{+\infty} \frac{1}{\sqrt{2\pi}} \exp\left(-\frac{(x - it)^2}{2}\right) dx = e^{-\frac{t^2}{2}}
$$

Последовательность случайных величин $\xi_1, \xi_2, \ldots$ сходится слабо к случайной величине $\xi$:
\[
  \xi_n \xrightarrow{d} \xi \ (\xi_n \Rightarrow \xi),
\]
если для любого $x$ такого, что функция распределения $F_\xi$ непрерывна в точке $x$, имеет место сходимость
\[
  F_{\xi_n}(x) \rightarrow F_\xi(x) \quad \text{при}\ n \rightarrow \infty.
\]

\Theor{О непрерывном соответствии}
Случайные величины $\xi_n$ слабо сходятся к случайной величине $\xi$ $\iff$ для любого $t$
характеристические функции $\varphi_{\xi_n}(t)$ сходятся к характеристической функции $\varphi_\xi(t)$.

Доказательство теоремы Леви (характеристические функции $\Leftrightarrow$ слабая сходимость) здесь опускается.

\Theor{Центральная предельная теорема}

Пусть $\xi_1, \xi_2, \ldots$ — последовательность независимых в совокупности и одинаково распределённых случайных величин с конечной и ненулевой дисперсией.

Тогда верно, что

\[
  \frac{S_n - n a}{\sigma \sqrt{n}} = \frac{\xi_1 + \ldots + \xi_n - n a}{\sigma \sqrt{n}} \xrightarrow{d} N(0,1).
\]

(где а - матожидание, $\sigma^2$ - дисперсия случайной величины $\xi_i$)

\Proof

Введём стандартизованные случайные величины $\zeta_i = (\xi_i - a)/\sigma$ — независимые случайные величины с нулевыми математическими ожиданиями и единичными дисперсиями.
Пусть $Z_n$ есть их сумма:
\[
  Z_n = \zeta_1 + \ldots + \zeta_n = \frac{S_n - n a}{\sigma}
\]
Требуется доказать, что $Z_n/\sqrt{n} \xrightarrow{d} N(0,1)$.

Характеристическая функция величины $Z_n/\sqrt{n}$ равна:
\[
  \varphi_{Z_n/\sqrt{n}}(t) = \varphi_{Z_n}\left( \frac{t}{\sqrt{n}} \right ) = \left[ \varphi_{\zeta_1}\left( \frac{t}{\sqrt{n}} \right ) \right]^n
\]

Характеристическую функцию случайной величины $\zeta_1$ можно разложить по формуле Тейлора (сначала разложим, потом возьмем матожидание по линейности):
\[
  \varphi_{\zeta_1}(t) = 1 + it \mathbb{E}\zeta_1 - \frac{t^2}{2}\mathbb{E}\zeta_1^2 + o(t^2)
  = 1 - \frac{t^2}{2} + o(t^2)
\]

Последний переход верен, т.к известно, что $\mathbb{E}\zeta_1 = 0$, $\mathbb{E}\zeta_1^2 = D\zeta_1 = 1$:

Подставим это разложение в точке $t/\sqrt{n}$ и устремим $n$ к бесконечности:
\[
  \varphi_{Z_n/\sqrt{n}}(t) =
  \left( 1 - \frac{t^2}{2n} + o\left( \frac{t^2}{n} \right ) \right)^n \rightarrow e^{-t^2/2}, \quad n \to \infty
\]

В пределе получили характеристическую функцию стандартного нормального распределения.
По теореме о непрерывном соответствии между функциями распределения и характеристическими функциями можно сделать вывод о слабой сходимости:
\[
  \frac{S_n - n a}{\sigma \sqrt{n}} \xrightarrow{d} N(0,1).
\]

\Endproof

\Example \textbf{(Применение ЦПТ)}

Монета подбрасывается $n = 100$ раз. Найти приближённую вероятность того, что число выпавших орлов окажется от 45 до 55.

\Solution Пусть $S_{100}$ — число орлов. Тогда $S_{100} = \xi_1 + \ldots + \xi_{100}$, где $\xi_i \sim Bern(1/2)$.

Имеем: $a = \mathbb{E}\xi_i = 1/2$, $\sigma^2 = D\xi_i = 1/4$, $\sigma = 1/2$.

По ЦПТ: $\displaystyle\frac{S_{100} - 100 \cdot \frac{1}{2}}{\frac{1}{2}\sqrt{100}} = \frac{S_{100} - 50}{5} \approx N(0,1)$.

Тогда:
\[
  \mathbb{P}(45 \leq S_{100} \leq 55) = \mathbb{P}\left(\frac{45-50}{5} \leq \frac{S_{100}-50}{5} \leq \frac{55-50}{5}\right) \approx \mathbb{P}(-1 \leq Z \leq 1),
\]
где $Z \sim N(0,1)$.

Используя таблицу стандартного нормального распределения: $\mathbb{P}(-1 \leq Z \leq 1) = 2\Phi(1) - 1 \approx 2 \cdot 0{,}8413 - 1 = 0{,}6826$.

Здесь $\Phi(x) = \mathbb{P}(Z \leq x)$ — функция распределения $N(0,1)$.
