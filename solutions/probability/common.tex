\subsection{Общие утверждения}

\Def \textit{Пространство элементарных исходов} $\Omega$ - произвольное множество, содержащее все возможные исходы случайного эксперимента.

\Def $\sigma$-\textit{алгебра} $\mathcal{F}$ — множество подмножеств $\Omega$, обладающее следующими свойствами:
\begin{enumerate}
  \item $\Omega \in \mathcal{F}$;
  \item $A \in \mathcal{F} \implies \overline{A} \in \mathcal{F}$;
  \item $A_1, A_2, \ldots, A_n, \ldots \in \mathcal{F} \implies \bigcup_{i=1}^\infty A_i \in \mathcal{F}$.
\end{enumerate}

\Example Множество всех подмножеств $2^{\Omega}$ и множество $\{\varnothing, \Omega\}$ — $\sigma$-алгебры над $\Omega$.

\Def \textit{Борелевская $\sigma$-алгебра} $\mathcal{B}$ — $\sigma$-алгебра, порождённая множеством всех открытых интервалов на $\mathbb{R}$ (иными словами, минимальная
$\sigma$-алгебра, содержащая все открытые интервалы). Элемент $B \in \mathcal{B}$ — \textit{борелевское множество}.

\Def \textit{Вероятностная мера} или \textit{вероятность} — функция $\mathbb{P}:\mathcal{F} \mapsto \mathbb{R}$, обладающая следующими свойствами:
\begin{enumerate}
  \item $\mathbb{P}(A) \geq 0 \quad \forall A \in \mathcal{F}$ \; (неотрицательность);
  \item $\mathbb{P}(\Omega) = 1$ \; (нормировка);
  \item $\forall A_1, A_2, \ldots, A_n, \ldots \in \mathcal{F}$, $A_i \cap A_j = \varnothing$ $(i \neq j): \mathbb{P}\left(\bigcup_{i=1}^{\infty} A_i\right) =
    \sum_{i=1}^{\infty} \mathbb{P}(A_i)$ \; (счётная аддитивность).
\end{enumerate}

\Def \textit{Вероятностное пространство} — тройка $(\Omega, \mathcal{F}, \mathbb{P})$, где $\Omega$ — множество элементарных исходов, $\mathcal{F}$ — $\sigma$-алгебра
над $\Omega$, вероятность $\mathbb{P}$ определена на $\mathcal{F}$.

\Def Пусть задано вероятностное пространство $(\Omega, \mathcal{F}, \mathbb{P})$. \textit{Случайной величиной} называется измеримая функция $X: \Omega \to \mathbb{R}$,
т.е. такая функция, что для любого борелевского множества $B \subset \mathbb{R}$ множество $X^{-1}(B) = \{\omega \in \Omega : X(\omega) \in B\} \in \mathcal{F}$.

\Def Случайная величина $X$ называется \textit{дискретной}, если множество её возможных значений $\{x_1, x_2, \ldots, x_n, \ldots\}$ не более чем счётно. \textit{Законом
распределения} дискретной случайной величины называется последовательность $\{p_k\}$, где
\[
  p_k = \mathbb{P}(X = x_k), \quad k = 1, 2, \ldots
\]
При этом $p_k \geq 0$ и $\sum\limits_{k} p_k = 1$.

\Def Случайная величина $X$ называется \textit{абсолютно непрерывной}, если существует неотрицательная функция $f(x)$, называемая \textit{плотностью распределения},
такая что для любого множества $B \subset \mathbb{R}$
\[
  \mathbb{P}(X \in B) = \int_B f(x) \, dx.
\]
При этом $f(x) \geq 0$ и $\int\limits_{-\infty}^{+\infty} f(x) \, dx = 1$.

\textbf{Замечание.} Для абсолютно непрерывной случайной величины $\mathbb{P}(X = x_0) = 0$ для любого фиксированного значения $x_0 \in \mathbb{R}$.

\Def \textit{Функция распределения} случайной величины $X$ — функция $F_X: \mathbb{R} \to [0,1]$, определённая как
\[
  F_X(x) = \mathbb{P}(X \leq x).
\]

\Props
\begin{enumerate}
  \item $F_X(x)$ — неубывающая функция;
  \item $\lim\limits_{x \to -\infty} F_X(x) = 0$, $\lim\limits_{x \to +\infty} F_X(x) = 1$;
  \item $F_X(x)$ непрерывна справа: $\lim\limits_{y \to x+} F_X(y) = F_X(x)$;
  \item $\mathbb{P}(a < X \leq b) = F_X(b) - F_X(a)$.
\end{enumerate}

\textbf{Замечание.} Для абсолютно непрерывной случайной величины $F_X(x) = \int\limits_{-\infty}^{x} f(t)\, dt$, откуда $f(x) = F'_X(x)$ в точках непрерывности плотности.

\vspace{0.5em}
\textbf{Основные распределения:}

\vspace{0.3em}
\textit{Дискретные:}
\begin{itemize}
  \item \textbf{Бернулли} $Bern(p)$: $\mathbb{P}(\xi = 1) = p$, $\mathbb{P}(\xi = 0) = 1-p$. $\mathbb{E}\xi = p$, $D\xi = p(1-p)$.
  \item \textbf{Биномиальное} $Bin(n, p)$: $\mathbb{P}(\xi = k) = C_n^k p^k (1-p)^{n-k}$, $k = 0, \ldots, n$. $\mathbb{E}\xi = np$, $D\xi = np(1-p)$.
  \item \textbf{Геометрическое} $Geom(p)$: $\mathbb{P}(\xi = k) = (1-p)^{k-1}p$, $k = 1, 2, \ldots$ $\mathbb{E}\xi = \frac{1}{p}$, $D\xi = \frac{1-p}{p^2}$.
  \item \textbf{Пуассона} $Pois(\lambda)$: $\mathbb{P}(\xi = k) = \frac{\lambda^k e^{-\lambda}}{k!}$, $k = 0, 1, 2, \ldots$ $\mathbb{E}\xi = \lambda$, $D\xi = \lambda$.
\end{itemize}

\textit{Абсолютно непрерывные:}
\begin{itemize}
  \item \textbf{Равномерное} $U[a,b]$: $f(x) = \frac{1}{b-a}$ при $x \in [a,b]$, иначе $0$. $\mathbb{E}\xi = \frac{a+b}{2}$, $D\xi = \frac{(b-a)^2}{12}$.
  \item \textbf{Экспоненциальное} $Exp(\lambda)$: $f(x) = \lambda e^{-\lambda x}$ при $x > 0$, иначе $0$. $\mathbb{E}\xi = \frac{1}{\lambda}$, $D\xi = \frac{1}{\lambda^2}$.
  \item \textbf{Нормальное} $\mathcal{N}(a, \sigma^2)$: $f(x) = \frac{1}{\sqrt{2\pi\sigma^2}} e^{-\frac{(x-a)^2}{2\sigma^2}}$. $\mathbb{E}\xi = a$, $D\xi = \sigma^2$.
\end{itemize}
