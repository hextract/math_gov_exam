\subsection{Общие утверждения}

\textbf{Определение.} \textit{Пространство элементарных исходов} $\Omega$ - произвольное множество, содержащее все возможные исходы случайного эксперимента.

\textbf{Определение.} $\sigma$-\textit{алгебра} $\mathcal{F}$ — множество подмножеств $\Omega$, обладающее следующими свойствами:
\begin{enumerate}
    \item $\Omega \in \mathcal{F}$;
    \item $A \in \mathcal{F} \implies \overline{A} \in \mathcal{F}$;
    \item $A_1, A_2, \ldots, A_n, \ldots \in \mathcal{F} \implies \bigcup_{i=1}^\infty A_i \in \mathcal{F}$.
\end{enumerate}

\textbf{Пример.} Множество всех подмножеств $2^{\Omega}$ и множество $\{\varnothing, \Omega\}$ — $\sigma$-алгебры над $\Omega$.

\textbf{Определение.} \textit{Борелевская $\sigma$-алгебра} $\mathcal{B}$ — $\sigma$-алгебра, \textcolor{blue}{порождённая} множеством всех открытых интервалов на $\mathbb{R}$ (иными словами, минимальная $\sigma$-алгебра, содержащая все открытые интервалы). Элемент $B \in \mathcal{B}$ — \textit{борелевское множество}.


\textbf{Определение.} \textit{Вероятностная мера} или \textit{вероятность} — функция $\mathbb{P}:\mathcal{F} \mapsto \mathbb{R}$, обладающая следующими свойствами:
\begin{enumerate}
    \item $\mathbb{P}(A) \geq 0 \quad \forall A \in \mathcal{F}$ \; (неотрицательность);
    \item $\mathbb{P}(\Omega) = 1$ \; (нормировка);
    \item $\forall A_1, A_2, \ldots, A_n, \ldots \in \mathcal{F}$, $A_i A_j = \varnothing$ $(i \neq j): \mathbb{P}\left(\bigcup_{i=1}^{\infty} A_i\right) = \sum_{i=1}^{\infty} \mathbb{P}(A_i)$ \; (счётная аддитивность).
\end{enumerate}

\textbf{Определение.} \textit{Вероятностное пространство} — тройка $(\Omega, \mathcal{F}, \mathbb{P})$, где $\Omega$ — множество элементарных исходов, $\mathcal{F}$ — $\sigma$-алгебра над $\Omega$, вероятность $\mathbb{P}$ определена на $\mathcal{F}$.

\textbf{Определение.} Пусть задано вероятностное пространство $(\Omega, \mathcal{F}, \mathbb{P})$. \textit{Случайной величиной} называется измеримая функция $X: \Omega \to \mathbb{R}$, т.е. такая функция, что для любого борелевского множества $B \subset \mathbb{R}$ множество $X^{-1}(B) = \{\omega \in \Omega : X(\omega) \in B\} \in \mathcal{F}$.

\textbf{Определение.} Случайная величина $X$ называется \textit{дискретной}, если множество её возможных значений $\{x_1, x_2, \ldots, x_n, \ldots\}$ не более чем счётно. \textit{Законом распределения} дискретной случайной величины называется последовательность $\{p_k\}$, где
\[
p_k = \mathbb{P}(X = x_k), \quad k = 1, 2, \ldots
\]
При этом $p_k \geq 0$ и $\sum\limits_{k} p_k = 1$.

\textbf{Определение.} Случайная величина $X$ называется \textit{абсолютно непрерывной}, если существует неотрицательная функция $f(x)$, называемая \textit{плотностью распределения}, такая что для любого множества $B \subset \mathbb{R}$
\[
\mathbb{P}(X \in B) = \int_B f(x) \, dx.
\]
При этом $f(x) \geq 0$ и $\int\limits_{-\infty}^{+\infty} f(x) \, dx = 1$.

\textbf{Замечание.} Для абсолютно непрерывной случайной величины $\mathbb{P}(X = x_0) = 0$ для любого фиксированного значения $x_0 \in \mathbb{R}$.