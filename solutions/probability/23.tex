\subsection{Полная система событий. Формула полной вероятности. Формула Байеса. Независимость событий и классов событий.}

\Def Пусть задано вероятностное пространство $(\Omega, \mathcal{F}, \mathbb{P})$, события $A, B \in \mathcal{F}$, $\mathbb{P}(B) > 0$. \textit{Условная вероятность события $A$ при событии $B$:}
\[
\mathbb{P}(A \mid B) = \frac{\mathbb{P}(AB)}{\mathbb{P}(B)}.
\]

\textbf{Замечание.} Некоторые свойства условной вероятности:
\begin{enumerate}
    \item Если $A \cap B = \varnothing$, то $\mathbb{P}(A \mid B) = 0$.
    \item Если $B \subset A$, то $\mathbb{P}(A \mid B) = 1$. Например, $\mathbb{P}(B \mid B) = 1$.
\end{enumerate}

\Def Пусть есть вероятностное пространство $(\Omega, \mathcal{F}, \mathbb{P})$. События $A_1, \ldots, A_n \in \mathcal{F}$ называются \textit{независимыми в совокупности}, если $\forall k = 2, \ ... \ n$ $\forall i_1,\ldots,i_k : 1 \leq i_1 < i_2 < \ldots < i_k \leq n$ выполняется:
\[
\mathbb{P}\left(\bigcap_{j=1}^k A_{i_j}\right) = \prod_{j=1}^k \mathbb{P}(A_{i_j}).
\]

Иными словами, события независимы в совокупности, если вероятность одновременного наступления \textit{любого набора} из этих событий равна \textit{произведению вероятностей} событий, входящих в этот набор. В частности, при $n=2$ события $A$ и $B$ независимы, если $\mathbb{P}(AB) = \mathbb{P}(A)\mathbb{P}(B)$.

\textbf{Замечание.} В общем случае из попарной независимости событий $A_1, ... A_n$ не следует их независимость в совокупности

\Example Рассмотрим правильный тетраэдр, три грани которого окрашены соответственно в красный, синий, зелёный цвета, а четвёртая грань содержит все три цвета. Событие $R$ (соответственно, $G$, $B$) означает, что выпала грань, содержащая красный (соответственно, зелёный, синий) цвет.

Т.к. каждый цвет есть на двух гранях из четырёх, то
\[
\mathbb{P}(R) = \mathbb{P}(G) = \mathbb{P}(B) = \frac{1}{2}.
\]

Вероятность пересечения, соответственно:
\[
\mathbb{P}(RG) = \mathbb{P}(GB) = \mathbb{P}(RB) = \frac{1}{4} = \frac{1}{2} \cdot \frac{1}{2},
\]
следовательно, все события попарно независимы. Однако вероятность пересечения всех трёх:
\[
\mathbb{P}(RGB) = \frac{1}{4} \neq \mathbb{P}(R) \, \mathbb{P}(G)\, \mathbb{P}(B),
\]
т.е. события не являются независимыми в совокупности.

\Def Пусть задано вероятностное пространство $(\Omega, \mathcal{F}, \mathbb{P})$, события $A_1, A_2, \ldots, A_n \in \mathcal{F}$. Совокупность событий $A_1, \ldots, A_n$ называется \textit{полной системой событий}, если выполняются условия:
\begin{enumerate}
    \item $A_i \cap A_j = \varnothing$ \quad при $i \ne j$\; (\textit{попарно независимы});
    \item $A_1 \cup A_2 \cup \ldots \cup A_n = \Omega$\; (\textit{в совокупности образуют всё пространство элементарных исходов}).
\end{enumerate}


\textbf{Формула полной вероятности.} Пусть даны события $A, B_1, \ldots, B_n, \ldots;$ $\mathbb{P}(B_i) > 0$, причём $B_i B_j = \varnothing~(i \ne j)$ и $\bigcup\limits_{i=1}^\infty B_i \supset A$ (например, $\bigcup\limits_{i=1}^\infty B_i = \Omega$). 

\textit{Тогда справедлива формула:}
\[
\mathbb{P}(A) = \sum\limits_{i=1}^\infty \mathbb{P}(B_i)\, \mathbb{P}(A|B_i).
\]

\Proof

Так как события $\{B_i\}_{j=1}^\infty$ образуют полную систему событий, то по определению событие $A$ можно представить следующим образом:
\[
A = A \cap \Omega = A \cap \left(\bigcup_{i=1}^{\infty} B_i\right) = \bigcup_{i=1}^{\infty}(A \cap B_i)
\]

События $\{B_i\}_{i=1}^{\infty}$ попарно несовместимы, значит, события $(A \cap B_i)$ тоже несовместимы. Тогда, воспользовавшись определением условной вероятности, получаем:
\[
\mathbb{P}(A) = \mathbb{P}\left(\bigcup_{i=1}^{\infty}(A \cap B_i)\right) = \sum_{i=1}^{\infty} \mathbb{P}(A \cap B_i) = \sum_{i=1}^{\infty} \mathbb{P}(A \mid B_i)\mathbb{P}(B_i)
\]

\Endproof


\textbf{Формулы Байеса.} Пусть даны события $A, H_1, \ldots, H_n, \ldots;$ $\mathbb{P}(A) > 0$, $\mathbb{P}(H_i) > 0$, причём $H_i H_j = \varnothing~(i \ne j)$ и $\bigcup\limits_{i=1}^\infty H_i \supset A$ (например, $\bigcup\limits_{i=1}^\infty H_i = \Omega$).

\textit{Тогда справедливы формулы Байеса:}
\[
\mathbb{P}(H_i|A) = \frac{\mathbb{P}(H_i)\, \mathbb{P}(A|H_i)}
{\sum\limits_{j=1}^\infty \mathbb{P}(H_j)\, \mathbb{P}(A|H_j)}, \qquad i = 1, ... \ n.
\]

\Proof

Согласно формуле полной вероятности, в знаменателе дроби стоит вероятность события $A$. Тогда
\[
\frac{\mathbb{P}(H_i)\, \mathbb{P}(A|H_i)}{\mathbb{P}(A)}
= \frac{\mathbb{P}(H_i)\, \mathbb{P}(A H_i)}{\mathbb{P}(A)\, \mathbb{P}(H_i)}
= \frac{\mathbb{P}(A H_i)}{\mathbb{P}(A)}
= \mathbb{P}(H_i|A).
\]

\Endproof