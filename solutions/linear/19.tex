\subsection{Приведение квадратичных форм в линейном пространстве к каноническому виду. Положительно определенные квадратичные формы. Критерий Сильвестра.}

\textbf{Приведение квадратичных форм в линейном пространстве к каноническому виду}

\Def Пусть $\Lc$ - ЛП. Пусть $b: \Lc^2 \to \R$. $b$ будем называть биленейной формой (БФ), если:
\begin{enumerate}
  \item $\forall x, y, z \in \Lc \hookrightarrow b(x + y, z) = b(x, z) + b(y, z)$
  \item $\forall x, y, z \in \Lc \hookrightarrow b(x, y + z) = b(x, y) + b(x, z)$
  \item $\forall \alpha \in \R, \forall x, y \in \Lc \hookrightarrow b(\alpha x, y) = \alpha b(x, y) = b(x, \alpha y)$
\end{enumerate}

\Def Пусть $\Lc$ - ЛП. Пусть $b: \Lc^2 \to \R$. $b$ - БФ. Отображение $b(x, x)$ будем называть квадратичной формой (КФ).

\Def Пусть $\Lc$ - ЛП. Пусть $b: \Lc \to \R$. $b$ - КФ. $b$ имеет диагональный вид в базисе $(e_1, \dots, e_n)$, если
\begin{equation*}
  b(x) = \sum\limits_{k = 1}^n \alpha_k x_k^2, \alpha_k \in \R
\end{equation*}

\Def Пусть $\Lc$ - ЛП. Пусть $b: \Lc \to \R$. $b$ - КФ. $b$ имеет канонический вид в базисе $(e_1, \dots, e_n)$, если
\begin{equation*}
  b(x) = \sum\limits_{k = 1}^n \alpha_k x_k^2, \alpha_k \in \{\pm 1, 0\}
\end{equation*}

Далее сформулируем теорему, которая утверждает, что всякая КФ приводима к каноническому виду.
В билете О.К. Подлипский рекомендует формулировать именно матричный метод, а не метод Лангранжа, т.к. критерий Сильвестра строится по такой же методологии.

\Th Для всякой квадратичной формы существует базис, где она имеет канонический вид.

\Proof

Пусть дана матрица квадратичной формы. Рассмотрим случаи:
\begin{enumerate}
  \item $b_{11} \neq 0$. Занулим весь столбец снизу вычитанием из каждой строки первой, домноженной на определенный коэффициент (делая одновеременно симметричную
    операцию со столбцами). Затем делаем $b_{11}$ равным $\pm 1$ делением на $\sqrt{|b_{11}|}$ первой строки и первого столбца.
  \item $b_{11} = 0$. Тогда рассмотрим три подслучая:
    \begin{enumerate}
      \item Справа и снизу нули, значит, идем далее.
      \item $\exists i \in 2 \dots n: b_{ii} \neq 0$. Тогда поменяем 1-ую и $i$-ую строку местами, а затем 1 и $i$-ый столбец.  Потом переходим к первому случаю $b_{11} \neq 0$.
      \item $\exists i \in 2 \dots n: b_{i1} \neq 0$. Тогда добавим $i$-ую строку к первой, а затем $i$-ый столбец к первому. Потом переходим к первому случаю $b_{11} \neq 0$.
    \end{enumerate}
\end{enumerate}

Заметим, что после выполнения алгоритма выше матрица примет следующий вид:
\begin{equation*}
  \begin{pmatrix}
    \pm 1 & 0 & \dots & 0 \\
    0 & & & \\
    \vdots & & C & \\
    0 & & &
  \end{pmatrix}
\end{equation*}
Ясно, что $C$ симметричная матрица в силу операций выше, поэтому применим алгоритм рекурсивно к ней.
Если мы будем делать аналогичные действия над строками единичной матрицы, то получим матрицу перехода к базису (она невырождена в силу того, что мы выполняли только лишь
элементарные преобразования), в котором форма имеет канонический вид.
Заметим, что в силу того, что мы выполняли элементарные преобразования, матрица перехода невырождена

\Endproof

\textbf{Положительно определенные квадратичные формы}

\Def Ранг квадратичной формы - это количество $\pm 1$ на диагонали матрицы в каноническом виде. Далее будем обозначать $\Rg b$.

\Statement Ранг квадратичной формы равен рангу матрицы мквадратичной формы.

\Def Пусть $b$ - КФ. Тогда будем говорить, что она положительно определена, если $\forall x \in \Lc \hookrightarrow b(x) > 0, x \neq 0$ (отрицательно аналогично).

\Def Пусть $b$ - КФ. Тогда будем говорить, что она положительно полуопределена, если $\forall x \in \Lc \hookrightarrow b(x) \geq 0, x \neq 0$ (отрицательно аналогично).

\Th Во всяком базисе, в котором КФ имеет канонический вид, количество $\pm 1$ постоянно.

\Def Пусть $\Lc^{(-)} \subset \Lc$ - максимальное по размерности подпространство, на котором КФ $b$ отрицательно определена. Тогда $\dim \Lc^{(-)}$ - отрицательный
индекс инерции. Аналогично определяется положительный индекс инерции.

\Th Отрицательный индекс инерции равен количеству $(-1)$, а положительный равен количеству $(+1)$.

\Proof

Пусть $(e_1, \dots e_n)$ - базис. Без ограничения общности будем считать, что:
\begin{equation*}
  b(x) = -x_1^2 - x_2^2 - \dots - x_k^2 + x_{k + 1}^2 + \dots + x_{\Rg b}^2 + 0 x_{\Rg b + 1}^2 + \dots 0 x_n^2
\end{equation*}
Тогда $b$ отрицательно определена на $\Lc_1 = \langle e_1 \dots e_k \rangle$, а на $\Lc_2 = \langle e_{k + 1} \dots e_n \rangle$ положительно полуопределена.

Ясно, что $\Lc = \Lc_1 \oplus \Lc_2$. Значит $\dim \Lc^{(-)} \geq \dim \Lc_1$. Предположим, что $\dim \Lc^{(-)} > \dim \Lc_1$.
Но тогда $\dim \Lc^{(-)} + \dim \Lc_2 > n \Rightarrow \exists x \in \Lc^{(-)}:
\begin{cases}
  b(x) \geq 0 \\
  b(x) < 0
\end{cases}$. Получили противоречие.

Аналогично показывается для положительного индекса инерции.

\Endproof

\Consequence $\Rg b = \dim \Lc^{(-)} + \dim \Lc^{(+)}$.

\textbf{Критерий Сильвестра}

\Def Сигнатурой КФ назовём разность положительных и отрицательных индексов инерции. Обозначать будем $\sign b$.

\Def Пусть $A$ матрица размера $n \times n$. Тогда положим $\Delta_k (A) \overset{def}{=} \det
\begin{pmatrix}
  a_{11} & \cdots & a_{1k} \\
  \vdots & \ddots & \vdots \\
  a_{k1} & \cdots & a_{kk}
\end{pmatrix}$

\Lemma Пусть $k$ - положительноопредленная КФ порожденная $b$ БФ, а $A$ - ее матрица в произвольном базисе $(e_1, \dots, e_n)$. Тогда $\forall i \in 1 \dots n
\hookrightarrow a_{ii} > 0$.

\Proof

\begin{equation*}
  0 < k(e_i) = b(e_i, e_i) = a_{ii}
\end{equation*}

\Endproof

\Theor{Критерий Сильвестра} Пусть $k$ - КФ. Пусть $B$ - её матрица. Тогда верно следующее.
\begin{enumerate}
  \item $k$ - положительно определена $\iff$ $\forall k \in 1 \dots n \hookrightarrow \Delta_k(B) > 0$.
  \item $k$ - отрицательно определена $\iff \forall k \in 1 \dots n \hookrightarrow \sign(\Delta_k) = (-1)^k$.
\end{enumerate}

\Proof

Покажем (1).

Пусть $k$ положительно определена. Перейдем к каноническому виду с помощью алгоритма из теоремы о приведении КФ к каноническому виду.
В силу того, что во процессе перехода к всякой строке может быть только добавлена строка выше, а ко всякому столбцу -- столбец левее, значит по сути с точки зрения
подматрицы $B[1:k][1:k]$ \textit{мы делаем элементарные преобразования, примененные к строкам и столбцам одновременно, которые не меняют знак определителя}.
Пусть мы привели $B$ к каноническому виду $B^K$. Тогда, в силу положительной определенности, $\Delta_n\left(B^K\right) = \prod\limits_{k=1}^n \underbrace{b_k}_{>0} > 0
\Rightarrow \Delta_n(B) > 0$.

В другую сторону, пусть все миноры матрицы положительны. Тогда в частности $\Delta_1(B) > 0$. Опять будем использовать тот же алгоритм из теоремы о приведении КФ к
каноническому виду.
После первого шага, мы будем иметь матрицу вида:
\begin{equation*}
  \begin{pmatrix}
    1 & 0 & \dots & 0 \\
    0 & & & \\
    \vdots & & C & \\
    0 & & &
  \end{pmatrix}
\end{equation*}
В силу того, что мы применили элементарные преобразования парно, знак определителя не поменялся. Тогда $\Delta_2 (B') = 1 \cdot b_2' > 0 \Rightarrow b_2 ' > 0$, т.к.
$\Delta_2 (B) > 0$. Значит рекурсивно применим рассуждения и для подматрицы $C$.
В итоге мы получим единичную матрицу, значит исходная форма положительно определена.

Покажем (2). Пусть $k(x)$ отрицательно определена. Значит $-k(x)$ - положительно определена. Применим утверждение $(1)$ вместе с тем фактом, что $\det(-A) = (-1)^n \det
A$, и получим требуемое.

\Endproof
