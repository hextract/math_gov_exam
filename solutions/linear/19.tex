\subsection{Приведение квадратичных форм в линейном пространстве к каноническому виду. Положительно определенные квадратичные формы. Критерий Сильвестра.}

\textbf{Приведение квадратичных форм в линейном пространстве к каноническому виду}

\Def Пусть $\Lc$ - ЛП. Пусть $b: \Lc^2 \to \R$. $b$ будем называть биленейной формой (БФ), если:
\begin{enumerate}
  \item $\forall x, y, z \in \Lc \hookrightarrow b(x + y, z) = b(x, z) + b(y, z)$
  \item $\forall x, y, z \in \Lc \hookrightarrow b(x, y + z) = b(x, y) + b(x, z)$
  \item $\forall \alpha \in \R, \forall x, y \in \Lc \hookrightarrow b(\alpha x, y) = \alpha b(x, y) = b(x, \alpha y)$
\end{enumerate}

\Def Пусть $\Lc$ - ЛП. Пусть $b: \Lc^2 \to \R$. $b$ - БФ. Отображение $b(x, x)$ будем называть квадратичной формой (КФ).

\Def Пусть $\Lc$ - ЛП. Пусть $b: \Lc \to \R$. $b$ - КФ. $b$ имеет диагональный вид в базисе $(e_1, \dots, e_n)$, если
\begin{equation*}
  b(x) = \sum\limits_{k = 1}^n \alpha_k x_k^2, \alpha_k \in \R
\end{equation*}

\Def Пусть $\Lc$ - ЛП. Пусть $b: \Lc \to \R$. $b$ - КФ. $b$ имеет канонический вид в базисе $(e_1, \dots, e_n)$, если
\begin{equation*}
  b(x) = \sum\limits_{k = 1}^n \alpha_k x_k^2, \alpha_k \in \{\pm 1, 0\}
\end{equation*}

Далее сформулируем метод Лангранжа.

\Th Для всякой квадратичной формы существует базис, где она имеет канонический вид.

\Proof

Зафиксируем произвольный базис $(e_1, \dots e_n)$. Без ограничения общности считаем, что коэфициент $\alpha_{11}$ перед $x_1^2$ ненулевой.

Вынесем $x_1$ за скобки.

\begin{equation*}
  b(x) = \alpha_{11} x_1^2 + x_1 (\dots) + \dots
\end{equation*}

Поделим всё на $\alpha_{11}$:

\begin{equation*}
  b(x) = \alpha_{11} \left(x_1^2 + x_1 \frac{\dots}{\alpha_{11}}\right) + \dots
\end{equation*}

Добавим и вычтем необходимое, чтобы скобку дополнить до полного квадрата:
\begin{equation*}
  b(x) = \alpha_{11} (x_1 + \dots)^2 + \alpha_{11} \underbrace{(\dots)}_{\text{Здесь нет $x_1$}} + \dots
\end{equation*}

Далее повторяем также для $x_2, x_3, \dots x_n$. Итого на выходе получим:
\begin{equation*}
  b(x) = \alpha_{11}(x_1 + \dots)^2 + \dots + \alpha_{nn} (x_n + \dots)^2
\end{equation*}

Далее делаем замену переменных вида:
\begin{equation*}
  x_k' = \sqrt{|\alpha_{kk}|} (x_k + \dots)
\end{equation*}

Получили базис где у нас КФ имеет канонический вид.

\Endproof

И еще один метод приведения КФ на каноническому виду.

\Th Для всякой квадратичной формы существует базис, где она имеет канонический вид.

\Proof

Пусть дана матрица квадратичной формы.

\begin{enumerate}
  \item $b_{11} \neq 0 \Rightarrow$ Обнуляем все, что под ней и справа от нее.
  \item $b_{11} = 0$
    \begin{enumerate}
      \item Справа и снизу ноль $\Rightarrow$ идем далее.
      \item
        \begin{enumerate}
          \item $\exists i \in 2 \dots n: b_{ii} \neq 0 \Rightarrow$ меняем местами 1 и $i$ строку и столбец.
          \item $\exists i \in 2 \dots n: b_{i1} \neq 0 \Rightarrow$ прибавляем $i$-ую строку к 1 и $i$-ый столбец к первому.
        \end{enumerate}
    \end{enumerate}
\end{enumerate}

Рекурсивно продолжаем выполнять операции для подматрицы. Если также мы параллельно будем у матрицы $E$ соответственно менять строки, то получим матрицу перехода.

\Endproof

\textbf{Положительно определенные квадратичные формы}

\Def Ранг квадратичной формы - это количество $\pm 1$ на диагонали матрицы в каноническом виде. Далее будем обозначать $\Rg b$.

\Statement Ранг квадратичной формы равен рангу матрицы мквадратичной формы.

\Def Пусть $b$ - КФ. Тогда будем говорить, что она положительно определена, если $\forall x \in \Lc \hookrightarrow b(x) > 0, x \neq 0$ (отрицательно аналогично).

\Def Пусть $b$ - КФ. Тогда будем говорить, что она положительно полуопределена, если $\forall x \in \Lc \hookrightarrow b(x) \geq 0, x \neq 0$ (отрицательно аналогично).

\Th Во всяком базисе, в котором КФ имеет канонический вид, количество $\pm 1$ постоянно.

\Def Пусть $\Lc^{(-)} \subset \Lc$ - максимальное по размерности подпространство, на котором КФ $b$ отрицательно определена. Тогда $\dim \Lc^{(-)}$ - отрицательный
индекс инерции. Аналогично определяется положительный индекс инерции.

\Th Отрицательный индекс инерции равен количеству $(-1)$, а положительный равен количеству $(+1)$.

\Proof

Пусть $(e_1, \dots e_n)$ - базис. Без ограничения общности будем считать, что:
\begin{equation*}
  b(x) = -x_1^2 - x_2^2 - \dots - x_k^2 + x_{k + 1}^2 + \dots + x_{\Rg b}^2 + 0 x_{\Rg b + 1}^2 + \dots 0 x_n^2
\end{equation*}
Тогда $b$ отрицательно определена на $\Lc_1 = \langle e_1 \dots e_k \rangle$, а на $\Lc_2 = \langle e_{k + 1} \dots e_n \rangle$ положительно полуопределена.

Ясно, что $\Lc = \Lc_1 \oplus \Lc_2$. Значит $\dim \Lc^{(-)} \geq \dim \Lc_1$. Предположим, что $\dim \Lc^{(-)} > \dim \Lc_1$.
Но тогда $\dim \Lc^{(-)} + \dim \Lc_2 > n \Rightarrow \exists x \in \Lc^{(-)}:
\begin{cases}
  b(x) \geq 0 \\
  b(x) < 0
\end{cases}$. Получили противоречие.

Аналогично показывается для положительного индекса инерции.

\Endproof

\Consequence $\Rg b = \dim \Lc^{(-)} + \dim \Lc^{(+)}$.

\textbf{Критерий Сильвестра}

\Def Сигнатурой КФ назовём разность положительных и отрицательных индексов инерции. Обозначать будем $\sign b$.

\Def Пусть $A$ матрица размера $n \times n$. Тогда положим $\Delta_k (A) \overset{def}{=} \det
\begin{pmatrix}
  a_{11} & \cdots & a_{1k} \\
  \vdots & \ddots & \vdots \\
  a_{k1} & \cdots & a_{kk}
\end{pmatrix}$

\Lemma Пусть $k$ - положительноопредленная КФ порожденная $b$ БФ, а $A$ - ее матрица в произвольном базисе $(e_1, \dots, e_n)$. Тогда $\forall i \in 1 \dots n
\hookrightarrow a_{ii} > 0$.

\Proof

\begin{equation*}
  0 < k(e_i) = b(e_i, e_i) = a_{ii}
\end{equation*}

\Endproof

\Theor{Критерий Сильвестра} Пусть $k$ - КФ. Пусть $B$ - её матрица. Тогда верно следующее.
\begin{enumerate}
  \item $k$ - положительно определена $\iff$ $\forall k \in 1 \dots n \hookrightarrow \Delta_k(B) > 0$.
  \item $k$ - отрицательно определена $\iff \forall k \in 0 \dots \left\lfloor \frac{n}{2} \right\rfloor \hookrightarrow \Delta_{2k + 1}(B) < 0, \Delta_{2k + 2}(A) > 0$.
\end{enumerate}

\Proof

Покажем (1).

Пусть $k$ положительно определена. Значит, по вышедоказанной лемме $b_{11} > 0 \Rightarrow$ обнулим все, что под ним и справа. Получим матрицу следующего вида:

\begin{equation*}
  \begin{pmatrix}
    b_{11} & 0      & \cdots & 0      \\
    0      & b'_{22} &        &        \\
    \vdots &        & \ddots &        \\
    0      &        &        & b'_{nn}
  \end{pmatrix}
  \quad \Delta_1 > 0
\end{equation*}

Коль скоро $k$ положительно определена, значит всякое сужение это квадратичной формы положительно определенно. Значит, используя вышедоказанную лемму можно утверждать
$\forall i \in 2 \dots n \hookrightarrow b'_{ii} > 0$.
Значит имеем право проделывать эти шаги для поматрицы. Да $i$-ом шаге получим матрицу следующего вида:

\begin{equation*}
  \begin{pmatrix}
    b_{11} > 0 & 0      & \cdots & 0      & \cdots & 0      \\
    0      & b'_{22} > 0 &        &        &        &        \\
    \vdots &        & \ddots &        &        & \vdots \\
    0      &        &        & b'_{ii} > 0 &        & 0      \\
    \vdots &        &        &        & \ddots & \vdots \\
    0      &        & \cdots & 0      & \cdots & b_{nn}
  \end{pmatrix}
\end{equation*}

Значит $\Delta_{ii}(B) > 0$.

В другую сторону. Будем выполнять все те же операции и будем получать следующие оценки.
\begin{equation*}
  b_{11} = \Delta_1(B) > 0
\end{equation*}
\begin{equation*}
  b_{22}' b_{11} = \Delta_2(B) > 0 \Rightarrow  b_{22}' > 0
\end{equation*}
И так далее. В конце получим, что все диагональные элементы строго больше нуля, а, значит, имеем положительную определенность.

Покажем (2). Пусть $k(x)$ отрицательно определена. Значит $-k(x)$ - положительно определена. Применим утверждение $(1)$ вместе с тем фактом, что $\det(-A) = (-1)^n \det
A$, и получим требуемое.
\Endproof
