\subsection{Прямые и плоскости в пространстве. Формулы расстояния от точки до прямой и до плоскости, между прямыми в пространстве. Углы между прямыми и плоскостями.}

Прямая на плоскости может быть задана:
\begin{enumerate}
    \item Векторным уравнением в параметрической форме:
    \begin{equation*}
        \mathbf{r} = \mathbf{r}_0 + \mathbf{a} t \quad (\mathbf{a} \neq \mathbf{0})
    \end{equation*}
    где $\mathbf{a}$ -- направляющий вектор, $\mathbf{r}_0$ -- радиус-вектор фиксированной точки на прямой.
    
    \item Нормальным векторным уравнением
    \begin{equation*}
        ( \mathbf{r} - r_0, \mathbf{n}) = 0 \quad (\mathbf{n} \neq \mathbf{0})
    \end{equation*}
    где $n$ -- нормальный вектор прямой.

    \item Общим уравнением в декартовой системе координат
    \begin{equation*}
        Ax + By + C = 0 \quad (A^2 + B^2 \neq 0)
    \end{equation*}
\end{enumerate}

Второе уравнение можно записать в общей декартовой системе координат в виде $(\mathbf{r}, \mathbf{n}) = D$.

Если первое уравнение записать в общей декартовой системе координат, то можно получить параметрическое уравнение прямой на плоскости.
\begin{equation*}
    x = x_0 + \alpha t, \quad y = y_0 + \beta t
\end{equation*}
Или в канонической форме:
\begin{equation*}
    \frac{x - x_0}{\alpha} = \frac{y - y_0}{\beta}
\end{equation*}

Уравнение прямой, проходящей через 2 различные точки может быть записано в векторной форме
\begin{equation*}
    \mathbf{r} = \mathbf{r}_1 + (\mathbf{r_2} - \mathbf{r}_1)t
\end{equation*}
И в координатной форме:
\begin{equation*}
    \frac{x - x_1}{x_2 - x_1} = \frac{y - y_1}{y_2 - y_1}
\end{equation*}

Расстояние от точки с радиус-вектором $\mathbf{r}_1$ до прямой, заданной векторным уравнением $2$, равно
\begin{equation*}
    |Ax_1+By_1+C|/\sqrt{A^2+B^2}
\end{equation*}

Плоскость может быть задана
\begin{enumerate}
    \item векторным параметрическим уравнением 
    \begin{equation*}
        \mathbf{r} = \mathbf{r}_0+\mathbf{a}u + \mathbf{b}v
    \end{equation*}
    где $\mathbf{a, b}$ -- направляющие векторы плоскости, $\mathbf{r}_0$ -- радиус вектор фиксированной точки плоскости.
    \item Нормальным векторным уравнением
    \begin{equation*}
        (\mathbf{r} - \mathbf{r}_0, \mathbf{n}) = 0 \quad (\mathbf{n} \neq 0)
    \end{equation*}
    где $\mathbf{n}$ -- нормальный вектор плоскости.
    \item Общим уравнением в декартовой системе координат
    \begin{equation*}
        Ax+By+Cz+D=0 \quad (A^2+B^2+C^2 \neq 0)
    \end{equation*}
\end{enumerate}

При этом уравнение $2$ можно записать в виде $(\mathbf{r}, \mathbf{n})=D$, а уравнение $1$ в виде $(\mathbf{r} - \mathbf{r}_0, \mathbf{a}, \mathbf{b}) = 0$. Последнее уравнение в координатной форме равносильно уравнению

\begin{equation*}
    \begin{vmatrix}
        x-x_0 & y-y_0 & z-z_0 \\
        \alpha_1 & \beta_1 & \gamma_1 \\
        \alpha_2 & \beta_2 & \gamma_2 \\
    \end{vmatrix}
    =0
\end{equation*}

Уравнение плоскости через три точки можно записать как
\begin{equation*}
    (\mathbf{r} - \mathbf{r}_0, \mathbf{r}_1 - \mathbf{r}_0, \mathbf{r}_2 - \mathbf{r}_0) = 0
\end{equation*}
И в координатной форме
\begin{equation*}
        \begin{vmatrix}
        x-x_0 & y-y_0 & z-z_0 \\
        x_1 - x_0 & y_1 - y_0 & z_1 - z_0 \\
        x_2 - x_0 & y_2 - y_0 & z_2 - z_0 \\
    \end{vmatrix}
    =0
\end{equation*}
Здесь $x_i, y_i, z_i$ - декартовы координаты точек, а $\mathbf{r}_i$ - соответствующие радиус-векторы.

Расстояние от точки с радиус-вектором $\mathbf{r}_1$ до плоскости, заданной уравнением $2$ равно
\begin{equation*}
    \frac{|(\mathbf{r}_1-\mathbf{r}_0, \mathbf{n})|}{|\mathbf{n}|}
\end{equation*}

Расстояние от точки $M(x_1, y_1, z_1)$ до плоскости, заданной в прямоугольной системе координат уравнением $3$ равно
\begin{equation*}
    \frac{|Ax_1+By_1+Cz_1|}{\sqrt{A^2+B^2+C^2}}
\end{equation*}

Прямая в пространстве может быть задана:
\begin{enumerate}
    \item Векторным уравнением в параметрической форме:
    \begin{equation*}
        \mathbf{r} = \mathbf{r}_0 + \mathbf{a} t \quad (\mathbf{a} \neq \mathbf{0})
    \end{equation*}
    где $\mathbf{a}$ -- направляющий вектор, $\mathbf{r}_0$ -- радиус-вектор фиксированной точки на прямой.
    
    \item Векторным уравнением:
    \begin{equation*}
        [ \mathbf{r} - r_0, \mathbf{a}] = 0 \quad (\mathbf{a} \neq \mathbf{0})
    \end{equation*}
    или
    \begin{equation*}
        [\mathbf{r}, \mathbf{a}] = \mathbf{b} \quad (\mathbf{a} \neq \mathbf{0}, (\mathbf{a}, \mathbf{b} = 0))
    \end{equation*}
\end{enumerate}

Если уравнение $1$ записать в общей декартовой системе координат, то получим параметрические уравнения прямой линии:
\begin{equation*}
    x=x_0+\alpha t, \quad y=y_0+\beta t, \quad z = z_0 + \gamma t
\end{equation*}
Исключением параметра $t$ параметрические уравнения приводятся к канонической форме:
\begin{equation*}
    \frac{x-x_0}{\alpha} = \frac{y - y_0}{\beta} = \frac{z-z_0}{\gamma}
\end{equation*}

Уравнение прямой через две точки задается в векторной форме
\begin{equation*}
    \mathbf{r} = \mathbf{r}_1 + (\mathbf{r}_2 - \mathbf{r}_1)t
\end{equation*}
и в координатной форме
\begin{equation*}
    \frac{x-x_1}{x_2-x_2} = \frac{y-y_1}{y_2-y_1} = \frac{z-z_1}{z_2-z_1}
\end{equation*}

Формула расстояния между прямыми может быть получена из формулы расстояния от точки до плоскости. Эти операции оставляются читателю в качестве несложного домашнего упражнения.

Угол между прямыми в пространстве с направляющими векторами $\mathbf{a}$ и $\mathbf{b}$ выражается как
\begin{equation*}
    \cos(\phi) = \left| \frac{(\mathbf{a}, \mathbf{b})}{|\mathbf{a}| \cdot |\mathbf{b}|} \right|
\end{equation*}

Угол между плоскостями -- угол между векторами нормали.
