\subsection{Собственные значения и собственные векторы линейных преобразований. Диагонализируемость линейных преобразований.}
\textbf{Собственные значения и собственные векторы линейных преобразований}
Собственным вектором линейного оператора $\varphi: V \rightarrow V$ называется вектор $\boldsymbol{v} \in V, \boldsymbol{v} \neq 0 $ такой, что $\varphi(\boldsymbol{v})
= \lambda \boldsymbol{v}$ для некоторого $\lambda \in \mathbb{K}$. В этом случае $\lambda$ называется собственным значением.

Заметим, что
\begin{equation*}
  \varphi(\boldsymbol{v}) = \lambda \boldsymbol{v} \Longleftrightarrow \varphi(\boldsymbol{v}) - \lambda \boldsymbol{v} = \boldsymbol{0} \Longleftarrow (\varphi -
  \lambda \text{Id})(\boldsymbol{v}) = \boldsymbol{0}
\end{equation*}

Ненулевой вектор $\boldsymbol{v}$ -- собственный с собственным значением $\lambda$ тогда и только тогда, когда он лежит в ядре преобразования $\varphi - \lambda
\text{Id}$. Пространство $Ker(\varphi - \lambda \text{Id})$ называется собственным подпространством, соответствующим собственному значению $\lambda$ и обозначается
$V_\lambda$. Размерность этого пространства называется геометрической кратностью собственного значения $\lambda$ и обозначается иногда $\gamma_\varphi(\lambda)$:
\begin{equation*}
  \gamma_\varphi(\lambda) := \text{dim}(V_\lambda)
\end{equation*}
Геометрическая кратность любого собственного значения не может быть нулевой.

$\lambda \in \mathbb{K}$ является собственным значением $\varphi$ тогда и только тогда, когда подпространство $V_\lambda$ нетривиально, т.е. состоит не только из
нуль-вектора. Это эквивалентно вырожденности $\varphi - \lambda \text{Id}$. Это эквивалентно условию $\text{det}(\varphi - \lambda \text{Id}) = 0$. Это эквивалентно
тому, что $\lambda \in \mathbb{K}$ --- корень характеристического многочлена $\chi_\varphi(t)$.

\Theor{оценка размерности собственного подпространства}

Если $ \lambda $ - корень харакеристического многочлена кратности $ k $ ( $ \chi(\lambda) = (\lambda_0 - \lambda)^k Q(\lambda), Q(\lambda_0) \neq 0 $ ), то размерность
собственного подпространства не превосходит $ k $

\Proof

Пусть $ \exists S $ ЛНЗ собственных векторов соответствующим СЗ $ \lambda_0 $: $ e_1, e_2 \dots e_S $. Дополним их до базиса $ e_1, e_2 \dots e_n $. Тогда матрица
преобразования примет следующий вид в этом базисе:

\begin{equation*}
  \begin{gathered}
    A_\varphi =
    \left(
      \begin{array}{ccc|c}
        \lambda_0 & \cdots  & 0         &   \\
        \vdots    & \ddots  & \vdots    & B \\
        0         &  \cdots & \lambda_0 &   \\
        \hline
        & 0& & C
      \end{array}
    \right)
    \\
    \Rightarrow \det(A_\varphi - \lambda E) = (\lambda_0 - \lambda)^S \cdot det(C - \lambda E) = (\lambda_0 - \lambda)^S Q(\lambda)
  \end{gathered}
\end{equation*}

$ S $ не превосходит $ k = $ кратность $ \lambda $ в $ (\lambda_0 - \lambda)^S Q(\lambda) \Rightarrow $ существует не более $ k $ ЛНЗ собственных векторов в собственном
подпространстве $ \lambda \Rightarrow \dim V_\lambda \leq k $.

\Endproof

Пусть $\lambda \in \mathbb{K}$ -- собственное значение $\varphi$. Алгебраическая кратность $\lambda$ -- это кратность $\lambda$ как корня $\chi_\varphi(t)$. Т.е. это
максимальная степень $k$, такая что $\chi_\varphi(t)$ делится на $(t-\lambda)^k$. Обозначается иногда $\mu_\varphi(\lambda)$. $ \gamma_\varphi(\lambda) \leq
\mu_\varphi(\lambda) $ (как следствие предыдущей теоремы).

\textbf{Диагонализируемость}
Оператор $\varphi: V \rightarrow V$ называется диагонализируемым, если в каком-то базисе его матрица диагональна. Т.е. для $\varphi$ можно найти базис, состоящий из его
собственных векторов. Это эквивалентно тому, что все пространство $V$ распалось в прямую сумму собственных подпространств всех собственных значений. Это эквивалентно
тому, что сумма геометрических кратностей все собственных значений оператора равна размерности всего подпространства:
\begin{equation*}
  n = \sum_{i=1}^{\sigma(\varphi)} \gamma_\varphi(\lambda_i)
\end{equation*}

\Th Оператор $\varphi: V \rightarrow V$ диагонализируем тогда и только тогда, когда сумма геометрических кратностей всех собственных значений $\varphi$
оказалась равна размерности $V$ (Или раскадывается в прямую сумму собственных подпространств: $V = V_1 \oplus \cdots \oplus V_k$, $V_i$ - собственное
подпространство, соответствующее собственному значению $\lambda_i$). В этом случае диагонализирующий базис можно получить, выбрав в каждом собственном подпространстве
$V_{\lambda_i}$ базис. В этом базисе (из собственных векторов) оператор имеет диагональную матрицу с собственными значениями на диагонали.

\Proof

\begin{enumerate}
  \item $\Rightarrow$

    Пусть $\varphi$ диагонализируем. Тогда по аналогично доказательству предыдущей теоремы, запишем его матрицу в диагональном виде

    \begin{equation*}
      A =
      \left(
        \begin{array}{ccccccc}
          \lambda_1^1 &  &  &  &  & & \mathbf{0} \\
          & \ddots  \\
          &  & \lambda_m^1 \\
          &  &  & \ddots \\
          &  &  &  & \lambda_1^k \\
          &  &  &  &  & \ddots \\
          \mathbf{0}  &  &  &  &  &  & \lambda_p^k \\
        \end{array}
      \right)
    \end{equation*}
    В таком случае $\sum \deg(\lambda^i) = \dim(V)$.

  \item $\Leftarrow$

    Если матрица преобразования $A$ раскладывается в прямую сумму собственных подпространств (в этом случае, очевидно, сумма всех геометрических кратностей будет равна
    размерности $V$), то его матрица имеет блочно-диагональный вид, где каждый блок представляет соответствующее собственное подпространство. Блочно-диагональные матрицы
    являются диагонализируемыми.
\end{enumerate}

\Endproof

\Theor{Необходимое условие диагонализируемости} $A$ -- диагонализируемо, $\Rightarrow$ все корни характеристического многочлена вещественные.

\Proof

Предположим, что $\exists \lambda_i \in \mathbb{C}$, тогда сумма кратностей вещественных собственных значений будет меньше размерности $V$, что
противоречит предыдущей теореме, значит предположение неверно, условие теоремы выполняется.

\Endproof
