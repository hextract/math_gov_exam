\subsection{Линейное пространство, базис и размерность. Линейное отображение конечномерных пространств, его матрица. Ядро и образ линейного отображения.}

\textbf{Линейное пространство}

\Def Пусть $\mathcal{L} \neq \varnothing, "+": \mathcal{L}^2 \to \mathcal{L}, "\cdot": \K \times \mathcal{L} \to \mathcal{L}$ ($\K$ здесь произвольное поле чисел, например $\R$).
Будем говорить, что тройка $(\mathcal{L}, +, \cdot)$ является линейным пространством, если выполнены следующие условия:
\begin{itemize}
  \item $\forall x, y \in \mathcal{L} \hookrightarrow x + y = y + x$
  \item $\forall x, y, z \in \mathcal{L} \hookrightarrow (x + y) + z = x + (y + z)$
  \item $\exists 0 \in \mathcal{L}: \forall x \in \mathcal{L} \hookrightarrow x + 0 = 0 + x$
  \item $\forall x \in \mathcal{L} \, \exists (-x) \in \mathcal{L}: x + (-x) = 0$
  \item $\forall \alpha, \beta \in \K, \forall x \in \mathcal{L} \hookrightarrow \alpha (\beta x) = (\alpha \beta) x$
  \item $\forall \alpha, \beta \in \K, \forall x \in \mathcal{L} \hookrightarrow (\alpha + \beta) x = \alpha x + \beta x$
  \item $\forall \alpha \in \K, \forall x, y \in \mathcal{L} \hookrightarrow \alpha (x + y) = \alpha x + \alpha y$
  \item $\forall x \in \mathcal{L} \hookrightarrow 1 \cdot x = x$
\end{itemize}

\Note Везде далее $\K = \R$ для удобства.

Элементы $\mathcal{L}$ называют векторами.

\Note Линейное пространство будем иногда сокращать до ЛП.

\Note Всюду, где не возникает двойственных ситуаций, линейное пространство $(\mathcal{L}, +, \cdot)$ будем обозначать $\mathcal{L}$.

\Def Пусть $\mathcal{L}$ - ЛП. Пусть $x_1, \dots x_n \in \mathcal{L}, \alpha_1, \dots \alpha_n \in \R$. Линейной комбинацией назовём $\sum\limits_{k=1}^n \alpha_k x_k$.

\Def Линейную комбинацию будем называть тривиальной если $\alpha_1 = \alpha_2 = \dots = \alpha_n = 0$.

\Def Пусть $\mathcal{L}$ - ЛП. Пусть $x_1, \dots x_n \in \mathcal{L}$. Будем говорить, что система векторов $x_1, \dots x_n$ линейно-независима (далее будем использовать
сокращение ЛНЗ), если только лишь тривиальная их линейная комбинация дает ноль.

\Th Пусть $\mathcal{L}$ - ЛП. Пусть $x_1, \dots x_n \in \mathcal{L}$. $x_1, x_2 \dots x_n$ ЛЗ $\iff$ один из векторов является линейной комбинацией других.

\Proof

Пусть $x_1, \dots x_n \in \mathcal{L}$. $x_1, x_2 \dots x_n$ ЛЗ. Тогда существует нетривиальная линейная комбинация $\sum\limits_{k = 1}^n \alpha_k x_k = 0$.

Без ограничения общности будем считать, что $\alpha_1 \neq 0$ в силу нетривиальности комбинации. Тогда $x_1 = \sum\limits_{k = 2}^n \left(-\frac{\alpha_k}{\alpha_1} x_k \right)$.

Пусть теперь один векторов является линейной комбинацией других. Будем считать, что это вектор $x_1 = \sum\limits_{k = 2}^n \alpha_k x_k$. Но тогда:
\begin{equation*}
  x_1 - \alpha_2 x_2 - \alpha_3 x_3 - \dots - \alpha_n x_n = 0
\end{equation*}
Получили нетривальную линейную комбинацию нуля, значит, по определению, система ЛЗ.

\Endproof

\Th Пусть $\mathcal{L}$ - ЛП и $x_1, \dots x_n \in \mathcal{L}$. Пусть $x \in \mathcal{L}$ может быть выражен как $x = \sum\limits_{k = 1}^n \alpha_k x_k$.

Тогда набор $x_1, \dots, x_n$ -- ЛНЗ $\iff$ коэффициенты $\alpha_1, \dots, \alpha_n$ определяются единственным образом.

\Proof

Пусть набор ЛНЗ. Предположим, что разложение не единственно, тогда $\exists (\gamma_1, \dots \gamma_n) \neq (\alpha_1, \dots, \alpha_n): x =  \sum\limits_{k = 1}^n \gamma_k x_k$.

Тогда имеем право записать следующее соотношение:
\begin{equation*}
  \sum\limits_{k = 1}^n \gamma_k x_k = x = \sum\limits_{k = 1}^n \alpha_k x_k
\end{equation*}
Тогда:
\begin{equation*}
  \sum\limits_{k = 1}^n (\gamma_k - \alpha_k) x_k = 0
\end{equation*}
Заметим, что мы получили нетривиальную линейную комбинацию нуля, ведь наборы коэффицентов различны и $\exists k' \in 1 \dots n: \gamma_{k'} - \alpha_{k'} \neq 0$.
Получили противоречие с ЛНЗ.

Пусть теперь разложение единственно. Предположим, что набор ЛЗ. Тогда есть нетривиальная линейная комбинация нуля: $\sum\limits_{k = 1}^n \gamma_k x_k = 0$.

Тогда:
\begin{equation*}
  x = \sum\limits_{k = 1}^n \alpha_k x_k = \sum\limits_{k = 1}^n \alpha_k x_k + 0 = \sum\limits_{k = 1}^n \alpha_k x_k + \sum\limits_{k = 1}^n \gamma_k x_k =
  \sum\limits_{k = 1}^n (\alpha_k + \gamma_k) x_k
\end{equation*}

Получили второе разложение $x$ - противоречие.

\Endproof

\textbf{Базисы}

\Def Пусть $\mathcal{L}$ - ЛП. Базисом $\mathcal{L}$ будем называть упорядоченный конечный набор ЛНЗ векторов, через которые выражается всякий вектор из $\mathcal{L}$.

\Statement Пусть $\mathcal{L}$ - ЛП. Всякий вектор раскладывается единственным образом по базису.

\Proof

Применим теорему выше и получим требуемое.

\Endproof

\Def Пусть $\mathcal{L}$ - ЛП. Пусть $e_1, e_2, \dots e_n$ - базис $\mathcal{L}$. Пусть $x = x_1 e_1 + x_2 e_2 + \dots + x_n e_n$. Тогда $
\begin{pmatrix}
  x_1 & x_2 & \hdots & x_n
\end{pmatrix}^T$ будем называть координатами $x$ в базисе $e_1, e_2, \dots e_n$.

\Statement Вектора ЛЗ $\iff$ их координатные столбцы ЛЗ.

\textbf{Размерность}

\Th Пусть $\mathcal{L}$ - ЛП и $e_1, \dots e_n$ его базис $\Rightarrow$ всякий базис состоит ровно из $n$ векторов.

\Proof

Покажем в начале, что больше быть не может. Предположим нашелся базис $f_1, f_2 \dots f_m, m > n$. Будем считать одновременно, что $f_1, f_2 \dots f_m$ - координатные
столбцы в базисе $e_1, e_2, \dots e_n$.

Тогда заметим, что $\operatorname{rg}
\begin{pmatrix}
  f_1 & f_2 & \hdots & f_m
\end{pmatrix} = \min \{ n, m \} = n \Rightarrow$ $f_1, \dots f_n$ - ЛЗ. Противоречие.

Покажем, что и меньше быть не может. Опять предположим обратное, что нашелся базис $f_1, f_2 \dots f_m, m < n$. Но тогда по все той же логике, $\operatorname{rg}
\begin{pmatrix}
  f_1 & f_2 & \hdots & f_m
\end{pmatrix} = \min \{ n, m \} = m \Rightarrow$ $e_1, \dots e_n$ - ЛЗ. Противоречие.

\Endproof

\Def Пусть $\mathcal{L}$ - ЛП и $e_1, \dots e_n$ его базис. Тогда положим $\dim \mathcal{L} \overset{def}{=} n$, в частности $\dim \{0\} \overset{def}{=} 0$. $\dim
\mathcal{L}$ называют размерностью ЛП $\mathcal{L}$.

\Statement $\dim \mathcal{L}$ определяется однозначно.

\Proof

По теореме выше, если нашелся базис из $n$ векторов, то всякий другой базис тоже состоит из $n$ векторов, поэтому определение $\dim \mathcal{L}$ корректно.

\Endproof

\Th Пусть $\mathcal{L}$ - ЛП, $\dim \mathcal{L} = n$ и $e_1, \dots e_n$ - ЛНЗ. Тогда $e_1, \dots e_n$ - базис.

\Proof

Пусть $x \in \mathcal{L}$. По теореме выше, $x, e_1, \dots e_n$ - ЛЗ. Значит, существует нетривиальная комбинация нуля:

\begin{equation*}
  \alpha x + \alpha_1 e_1 + \dots + \alpha_n e_n = 0
\end{equation*}

Заметим, что $\alpha \neq 0$, ведь иначе бы $e_1, \dots e_n$ - ЛЗ. Поэтому:
\begin{equation*}
  x = \left( - \frac{\alpha_1}{\alpha} \right) e_1 + \left( - \frac{\alpha_2}{\alpha} \right) e_2 + \dots \left( - \frac{\alpha_n}{\alpha} \right) e_n
\end{equation*}

\Endproof

\textbf{Линейные отображения}

\Def Пусть $\mathcal{L}_1, \mathcal{L}_2$ - ЛП. Пусть $\varphi: \mathcal{L}_1 \to \mathcal{L}_2$. Будем говорить, что $\varphi$ - линейное отображение (ЛО), если
выполнены следующие условия:
\begin{enumerate}
  \item $\forall x, y \in \mathcal{L}_1 \hookrightarrow \varphi(x + y) = \varphi(x) + \varphi(y)$
  \item $\forall \alpha \in \R, \forall x \in \mathcal{L}_1 \hookrightarrow \varphi(\alpha x) = \alpha \varphi(x)$
\end{enumerate}

\Def Пусть $\mathcal{L}_1, \mathcal{L}_2$ - ЛП. Пусть $\varphi: \mathcal{L}_1 \to \mathcal{L}_2$ - линейное отображение. Если $\mathcal{L}_1 = \mathcal{L}_2$, то будем
говорить, что $\varphi$ - линейное преобразование (ЛП).

\Def Пусть $(\mathcal{L}, +, \cdot)$ - ЛП. Пусть $\mathcal{L}' \subset \mathcal{L}$. Будем говорить, что $(\mathcal{L}', +, \cdot)$ - линейное подпространство
$(\mathcal{L}, +, \cdot)$, если выполнены следующие условия:
\begin{enumerate}
  \item $\forall x, y \in \mathcal{L}' \hookrightarrow x + y \in \mathcal{L}'$
  \item $\forall \alpha \in \R, \forall x \in \mathcal{L}' \hookrightarrow \alpha x \in \mathcal{L}'$
\end{enumerate}

\Note Если $\mathcal{L}'$ - линейное подпространство $\mathcal{L}$, если не возникает двойственных ситуаций, будем обозначать это $\mathcal{L}' \subset \mathcal{L}$.

\Statement Пусть $\mathcal{L}$ - ЛП, $\mathcal{L}' \subset \mathcal{L}$ - линейное подпространство. Тогда $\mathcal{L}'$ - ЛП.

\Statement Пусть $\mathcal{L}_1, \mathcal{L}_2$ - ЛП, $e_1, e_2, \dots e_n \in \mathcal{L}_1$ - ЛЗ, $\varphi: \mathcal{L}_1 \to \mathcal{L}_2$ - ЛО $\Rightarrow$
$\varphi(e_1), \dots, \varphi(e_n)$ - ЛЗ

\Proof

$e_1, e_2, \dots e_n \in \mathcal{L}$ - ЛЗ $\Rightarrow e_1 = \alpha_2 e_2 + \alpha_3 e_3 + \dots + \alpha_n e_n$ (без ограничения общности будем считать, что именно
$e_1$ выразим).

Но тогда $\varphi(e_1) = \varphi(\alpha_2 e_2 + \dots + \alpha_n e_n) = \alpha_2 \varphi(e_2) + \dots + \alpha_n \varphi(e_n)$. Получили требуемое.

\Endproof

\Statement Пусть $\mathcal{L}_1, \mathcal{L}_2$ - ЛП, $\varphi: \mathcal{L}_1 \to \mathcal{L}_2$ - ЛО, $\mathcal{L}' \subset \mathcal{L}_1 \Rightarrow
\varphi(\mathcal{L}') \subset \mathcal{L}_2$.

\Proof

Пусть $e_1, \dots e_n$ - базис $\mathcal{L}'$. Заметим, что $0 \in \varphi(\mathcal{L}')$. Действительно, $\varphi(0 e_1 + \dots + 0 e_n) = 0 \varphi(e_1) + \dots + 0
\varphi(e_n) = 0$

Далее:
\begin{equation*}
  \forall \alpha, \beta \in \R, \forall x, y \in \mathcal{L}' \hookrightarrow \varphi(\alpha x) + \varphi(\beta y) = \varphi(\alpha x + \beta y) \in \mathcal{L}_2
\end{equation*}

\Endproof

\Statement. Пусть $\mathcal{L}_1, \mathcal{L}_2$ - ЛП, $\varphi: \mathcal{L}_1 \to \mathcal{L}_2$ - ЛО, $\mathcal{L}' \subset \mathcal{L}_1 \Rightarrow \dim
\varphi(\mathcal{L}') \leq \dim \mathcal{L}'$

\Proof

Пусть $e_1, \dots e_n$ - базис $\mathcal{L}_1$. Пусть $x \in \mathcal{L}' \Rightarrow \varphi(x) = x_1 \varphi(e_1) + \dots + x_n \varphi(e_n) \Rightarrow \dim
\varphi(\mathcal{L}') \leq \dim \mathcal{L}'$.

\Endproof

\Def Пусть $\mathcal{L}_1, \mathcal{L}_2$ - ЛП, $\varphi: \mathcal{L}_1 \to \mathcal{L}_2$ - ЛО. Пусть $(e_1, \dots, e_n)$ - базис $\mathcal{L}_1$. Тогда матрицей
$\varphi$ в базисе $(e_1, \dots e_n)$ будем называть матрицу $A_\varphi$, которая получена следующим образом:
\begin{equation*}
  A_\varphi =
  \begin{pmatrix}
    \varphi(e_1) & \varphi(e_2) & \hdots & \varphi(e_n)
  \end{pmatrix}
\end{equation*}

\Statement Пусть $\mathcal{L}_1, \mathcal{L}_2$ - ЛП, $\varphi: \mathcal{L}_1 \to \mathcal{L}_2$ - ЛО. Пусть $(e_1, \dots, e_n)$ - базис $\mathcal{L}_1$. Тогда $\forall
x \in \mathcal{L}_1 \hookrightarrow \varphi(x) = A_\varphi x$, если отождествить $x$ с его координатным столбцом в данном базисе.

\Proof

Пусть $x \in \mathcal{L}_1$. Так как $(e_1, \dots, e_n)$ -- базис $\mathcal{L}_1$, то существует единственное разложение:
\begin{equation*}
  x = x_1 e_1 + x_2 e_2 + \dots + x_n e_n,
\end{equation*}
где $x_i \in \mathbb{R}$ (или $\mathbb{C}$) -- координаты вектора $x$ в данном базисе. Отождествим $x$ с координатным столбцом:
\begin{equation*}
  x \leftrightarrow
  \begin{pmatrix} x_1 \\ x_2 \\ \vdots \\ x_n
  \end{pmatrix}.
\end{equation*}

Рассмотрим образ $\varphi(x)$. По линейности оператора $\varphi$:
\begin{equation*}
  \varphi(x) = \varphi\left( \sum_{j=1}^n x_j e_j \right) = \sum_{j=1}^n x_j \varphi(e_j).
\end{equation*}

Для каждого базисного вектора $e_j$ образ $\varphi(e_j) \in \mathcal{L}_2$ можно разложить по некоторому фиксированному базису $(f_1, \dots, f_m)$ пространства $\mathcal{L}_2$:
\begin{equation*}
  \varphi(e_j) = \sum_{i=1}^m a_{ij} f_i, \quad j = 1, \dots, n,
\end{equation*}
где $a_{ij}$ -- коэффициенты разложения, образующие матрицу $A_\varphi = (a_{ij})_{m \times n}$.

Подставим это в выражение для $\varphi(x)$:
\begin{align*}
  \varphi(x) &= \sum_{j=1}^n x_j \left( \sum_{i=1}^m a_{ij} f_i \right) \\
  &= \sum_{i=1}^m \left( \sum_{j=1}^n a_{ij} x_j \right) f_i.
\end{align*}

Координатный столбец $\varphi(x)$ в базисе $(f_1, \dots, f_m)$ пространства $\mathcal{L}_2$ имеет вид:
\begin{equation*}
  \begin{pmatrix}
    \sum_{j=1}^n a_{1j} x_j \\
    \sum_{j=1}^n a_{2j} x_j \\
    \vdots \\
    \sum_{j=1}^n a_{mj} x_j
  \end{pmatrix}
  = A_\varphi
  \begin{pmatrix}
    x_1 \\
    x_2 \\
    \vdots \\
    x_n
  \end{pmatrix}.
\end{equation*}

Таким образом, при отождествлении векторов с их координатными столбцами, действие линейного оператора $\varphi$ выражается матричным умножением: $\varphi(x) = A_\varphi x$.

\Endproof

\textbf{Ядро и образ линейного отображения}

\Def Пусть $\mathcal{L}_1, \mathcal{L}_2$ - ЛП, $\varphi: \mathcal{L}_1 \to \mathcal{L}_2$ - ЛО.
\begin{equation*}
  \Ima \varphi \overset{def}{=} \condset{\varphi(x)}{x \in \mathcal{L}_1}
\end{equation*}
\begin{equation*}
  \Ker \varphi \overset{def}{=} \condset{x}{x \in \mathcal{L}_1: \varphi(x) = 0}
\end{equation*}

\Def Пусть $\mathcal{L}_1, \mathcal{L}_2$ - ЛП, $\varphi: \mathcal{L}_1 \to \mathcal{L}_2$ - ЛО. $\Rg \varphi \overset{def}{=} \dim \Ima \varphi$

\Th  Пусть $\mathcal{L}_1, \mathcal{L}_2$ - ЛП, $\varphi: \mathcal{L}_1 \to \mathcal{L}_2$ - ЛО. $\Ker \varphi$ - линейное подпространство $\mathcal{L}_1$.

\begin{center}
  $\boxed{\text{Всюду далее $\mathcal{L}_1, \mathcal{L}_2$ - ЛП, $\varphi: \mathcal{L}_1 \to \mathcal{L}_2$ - ЛО.}}$
\end{center}

\Def $\varphi$ - сюръекция $\overset{def}{\iff} \forall y \in \Lc_2 \, \exists x \in \Lc_1: \varphi(x) = y$.

\Th $\Rg \varphi = \dim \Lc_2 \iff \varphi \text{ - сюръекция}$.

\Proof

$\Rg \varphi = \dim \Lc_2 \iff \dim \Ima \varphi = \dim \Lc_2 \overset{\Ima \varphi \subset \Lc_2}{\iff} \Ima \varphi = \Lc_2$

\Endproof

\Def $\varphi$ - инъекция $\overset{def}{\iff} \forall x, y \in \Lc_1: x \neq y \Rightarrow \varphi(x) \neq \varphi(y)$.

\Th $\Ker \varphi = \{ 0 \} \iff \varphi \text{ - инъекция}$

\Proof

Пусть $\varphi$ - инъекция. Пусть $\Ker \varphi \neq \{ 0 \}$. Т.е. $\exists x \in \Ker \varphi: x \neq 0$. Тогда $\varphi(x) = \varphi(0) = 0$ - противоречие с определением.

Пусть $\Ker \varphi = \{ 0 \}$. Предположим $\varphi$ - не инъекция. Тогда $\exists x \neq y \in \Lc_1: \varphi(x) = \varphi(y) \Rightarrow \varphi(x - y) = 0
\Rightarrow x - y \in \Ker \varphi$. Противоречие.

\Endproof

\Th Пусть $\varphi$ - инъекция. Пусть $e_1, \dots e_n$ - базис в $\Lc_1 \Rightarrow \varphi(e_1), \dots, \varphi(e_n)$ - ЛНЗ.

\Proof

Предположим противное, тогда существует нетривиальная комбинация $\sum\limits_{k = 1}^n \alpha_k \varphi(e_k) = 0 \iff \varphi\left(\sum\limits_{k=1}^n \alpha_k e_k
\right) = 0 \overset{\Ker \varphi = \{0\}}{\iff} \sum\limits_{k=1}^n \alpha_k e_k = 0$. Противоречие.

\Endproof

\Def $\varphi$ - биекция $\iff$ $\varphi$ - сюръекция и инъекция.

\Th $\dim \Lc_1 = \dim \Ker \varphi + \dim \Ima \varphi$

\Proof

Очевидно, что $\Ker \varphi$ задается системой уравнений $A_\varphi x = 0$. А значит размерность ядра есть размерность ФСР данного уравнения, т.е.:
\begin{equation*}
  \dim \Ker \varphi = \Rg F(A_\varphi) = \dim \Lc_1 - \Rg A_\varphi = n - \Rg \varphi = n - \dim \Ima \varphi
\end{equation*}

Тогда:
\begin{equation*}
  \dim \Lc_1 = \underbrace{\dim \Lc_1 - \dim \Ima \varphi}_{\dim \Ker \varphi} + \dim \Ima \varphi = \dim \Ker \varphi + \dim \Ima \varphi
\end{equation*}

\Endproof

\Th $\varphi$ - биекция $\Rightarrow \dim \Lc_1 = \dim \Lc_2$.

\Proof

Действительно, коль скоро отображение биективно, по критерию выше $\dim \Ker \varphi = 0$, а также $\dim \Ima \varphi = \dim \Lc_2$. А значит:
\begin{equation*}
  \dim \Lc_1 = \dim \Ker \varphi + \dim \Ima \varphi = 0 + \dim \Lc_2 = \dim \Lc_2
\end{equation*}

\Endproof
