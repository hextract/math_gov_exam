\subsection{Самосопряженные преобразования евклидовых пространств, свойства их собственных значений и собственных векторов.}

Пусть $V$ --- евклидово пространство со скалярным произведением $\langle \cdot, \cdot \rangle, \varphi: V \rightarrow V$ -- линейный оператор. Линейный оператор $\varphi^*: V \rightarrow V$ называют сопряженным к $\varphi$, если
\begin{equation*}
    \langle \varphi(\boldsymbol{u}), \boldsymbol{v} \rangle = \langle \boldsymbol{u}, \varphi^*(\boldsymbol{v}) \rangle
\end{equation*}

\textbf{Теорема} Для любого линейного оператора $\varphi$ над $V$ существует единственный сопряженный ему линейный оператор $\varphi^*$.

\textbf{Доказательство} Запишем в матричном виде
$A$ -- матрица $\varphi$, $A^*$ -- матрица $\varphi^*$.
\begin{equation*}
    \begin{gathered}
        (A(x), y) = (x, A^*(y))
        \\
        x^TA^T\text{Г}y = x^T\text{Г}A^*y \Rightarrow A^T\text{Г}=ГA^T, \quad A^* = \text{Г}^{-1}A^T\text{Г}
    \end{gathered}
\end{equation*}
ЧТД
\\


Если $\varphi = \varphi^*$, то оператор $\varphi$ называют самосопряженным, или симметричным. Если $\varphi = -\varphi^*$, то оператор $\varphi$ называют кососимметричным.
\\

\textbf{Свойства}

Матрица $A$ самосопряженного преобразования удовлетворяет тождеству $A^T\text{Г} = \text{Г}A$, где $Г$ --- матрица Грама ($\text{Г}_{i, j}=\langle \boldsymbol{e}_i, \boldsymbol{e}_j \rangle$, $(\boldsymbol{e_{i}})_1^n$ -- базис). Отсюда следует, что если базис ортонормированный, то $A^T = A$, и обратное: если матрица оператора в ортонормированном базисе симметрична, то он самосопряжен.

\textbf{Теорема} Все корни характеристического многочлена самосопряженного преобразования -- вещественные числа

\textbf{Доказательство} Предположим противное, $\exists \lambda, \overline{\lambda} \in \mathbb{C}$ -- корни $\Rightarrow \exists$ инвариантное двумерное подпространство, в котором нет собственных векторов $(\varepsilon')$

Сузим $A$ на него и получим $A': \varepsilon' \mapsto \varepsilon$. Перейдем в ОНБ и получим симметричную матрицу $A'$. Рассмотрим ее характеристический многочлен:
\begin{equation*}
    det \left(
    \begin{matrix}
        \alpha - \lambda & \sigma \\
        \sigma & \beta - \lambda \\
    \end{matrix}
    \right)
    = (\alpha - \lambda)(\beta - \lambda) - \sigma^2 = 0
\end{equation*}
При этом $\exists \lambda' \in \mathbb{R}$ -- корень, т.к. $D > 0$, но тогда в $\varepsilon'$ существуют собственные вектора, противоречие. ЧТД.
\\

\textbf{Теорема} Собственные вектора самосопряженного преобразования, отвечающие попрано различным значениям ортогональны.

\textbf{Доказательство} Пусть $A$ -- самосопряженное преобразование, $\lambda_1, \lambda_2 \in \mathbb{R}, \quad \lambda_1 \neq \lambda_2$ - собственные значения, $h_1, h_2$ -- соответсвующие им собственные вектора.
\begin{equation*}
\begin{gathered}
\lambda_1(h_1, h_2) = (A(h_1), h_2) = (h_1, A(h_2)) = \lambda_2(h_1, h_2) \Rightarrow \lambda_1(h_1, h_2) = \lambda_2(h_1, h_2)
\\
(\lambda_1 - \lambda_2)(h_1, h_2) = 0
\\
\lambda_1 \neq \lambda_2 \Rightarrow (h_1, h_2) = 0 \Rightarrow h1 \bot h_2
\end{gathered}
\end{equation*}
ЧТД
\\

\textbf{Теорема} Если преобразование $A$ самосопряженное и $\varepsilon_1 \subset \varepsilon$ -- инвариант относительно $A$, то $\varepsilon_1^\bot$ тоже инвариант.

\textbf{Доказательство} $\forall x \in \varepsilon_1, \forall y \in \varepsilon_1^\bot \hookrightarrow A(x) \in \varepsilon_1 \Rightarrow (A(x), y) = 0$, т.к. $A(x) \bot y$

$\Rightarrow (x, A(y)) = 0 \Rightarrow A(y) \in \varepsilon_1^\bot \Rightarrow \varepsilon_1^\bot$ -- инвариант, ЧТД.
\\

\textbf{Теорема} Преобразование $\varphi: V \rightarrow V$ самосопряжено $\Longleftrightarrow$ существует ортонормированный базис в $V$, состоящий из его собственных векторов.

\textbf{Доказательство $\Leftarrow$} В одну сторону очевидно, (и в другую тоже очевидно), в базисе из собственных векторов матрица $A$ симметрична, а значит, преобразование самосопряжено.

\textbf{Доказательство $\Rightarrow$} $A$ -- самосопряженное преобразование $\Rightarrow$ все корни его характеристического многочлена вещественные и можно представить $V = V_1 \bigoplus \cdots \bigoplus V_n$, где $V_i$ -- собственные подпространства $\Rightarrow V - $ инвариант как сумма инвариантных подпространств.

Предположим $\varepsilon \neq V \Rightarrow dim V < dim \varepsilon \Rightarrow \Rightarrow dim V^\bot >= 1$ и $V^\bot$ -- инвариант.

В силу того, что все собственные значения вещественны и т.к. $V^\bot$ -- инвариант, то существует собственный вектор из $V^\bot$, но $V$ -- прямая сумма подпространств, и такого вектора нет, противоречие. Т.к. собственные вектора, соответствующие различным собственным значениям попарно ортогональны, то базис можно отнормировать и получить ОНБ, ЧТД.
