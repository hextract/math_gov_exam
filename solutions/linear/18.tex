\subsection{Самосопряженные преобразования евклидовых пространств, свойства их собственных значений и собственных векторов.}
Пусть $V$ --- евклидово пространство со скалярным произведением $\langle \cdot, \cdot \rangle, \varphi: V \rightarrow V$ -- линейный оператор. Линейный оператор $\varphi^*: V \rightarrow V$ называют сопряженным к $\varphi$, если
\begin{equation*}
    \langle \varphi(\boldsymbol{u}, \boldsymbol{v}) = \langle \boldsymbol{u}, \varphi^*(\boldsymbol{v})
\end{equation*}

\textbf{Теорема} Для любого линейного оператора $\varphi$ над $V$ существует единственный сопряженный ему линейный оператор $\varphi^*$.

Если $\varphi = \varphi^*$, то оператор $\varphi$ называют самосопряженным, или симметричным. Если $\varphi^* = \varphi$, то оператор $\varphi$ называют кососимметричным.

\textbf{Свойства}
Матрица $A$ самосопряженного преобразования удовлетворяет тождеству $A^T\text{Г} = \text{Г}A$, где $Г$ --- матрица Грама ($\text{Г}_{i, j}=\langle \boldsymbol{e}_i, \boldsymbol{e}_j \rangle$, $(\boldsymbol{e_{i}})_1^n$ -- базис). Отсюда следует, что если базис ортонормированный, то $A^T = A$, и обратное: если матрица оператора в ортонормированном базисе симметрична, то он самосопряжен.

\textbf{Теорема} Преобразование $\varphi: V \rightarrow V$ самосопряжено $\Longleftrightarrow$ существует ортонормированный базис в $V$, состоящий из его собственных векторов.
\textit{Доказательство будет когда заботаю остальное, или если появится issue}
