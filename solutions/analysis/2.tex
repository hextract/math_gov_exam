\subsection{Ограниченность функции, непрерывной на отрезке, достижение точных верхней и нижней граней.}

\Def $\hat \R := \R \cup \{ -\infty, +\infty, \infty \}$

\Def Пусть $x_0, A \in \hat\R, \delta_0 > 0, f: \mathring U_{\delta_0}(x_0) \to \R$. Будем говорить, что $A$ - предел функции $f$ в точке $x_0$ и записывать $\lim\limits_{x \to x_0} f(x) = A$, если
\begin{equation*}
    \forall \varepsilon > 0 \; \exists \delta_\varepsilon < \delta_0: \forall x \in \mathring U_{\delta_\varepsilon}(x_0) \hookrightarrow f(x) \in U_\varepsilon(A)
\end{equation*}

\Def Последовательностью Гейне в точке $x_0 \in \hat\R$ называется такая последовательность $\{x_n\} \subset \R$:
\begin{enumerate}
    \item $x_n \to x_0, n \to \infty$
    \item $\forall n \in \N \hookrightarrow x_n \neq x_0$
\end{enumerate}

\Def Пусть $x_0 \in \hat\R, A \in \hat\R$. Пусть $f: \mathring U_{\delta_0}(x_0) \to \R$. 
Будем говорить, что $A$ - предел функции $f$ в точке $x_0$ и записывать $\lim\limits_{x \to x_0} f(x) = A$, если для всякой последовательности Гейне $\{x_n\} \subset \mathring U_{\delta_0}(x_0)$ в точке $x_0$ выполнено, что $f(x_n) \to A, n \to \infty$.

\Th Определения пределов функций выше эквивалентны.

\Def Пусть $\delta_0 > 0$, $x_0 \in \mathbb{R}$, $f: U_{\delta_0} (x_0) \to \R$. Будем говорить, что $f$ непрерывна в точке $x_0$, если $\exists \lim\limits_{x \to x_0} f(x) = f(x_0)$.

\Def Пусть $\delta_0 > 0$, $x_0 \in \mathbb{R}$, $f: U_{\delta_0} (x_0) \to \R$. Будем говорить, что $f$ разрывна в точке $x_0$, если она не непрерывна в точке $x_0$.

\Def Пусть $X \subset \R, f: X \to \R$. Будем говорить, что $f$ непрерывна на $X$, если она непрерывна в каждой точке $X$.

\Lemma

Множество значений \(
  f([a,b])=\{f(x)\mid x\in[a,b]\}
\) ограничено

\Proof

Предположим, что функция не ограничена. Для определенности, сверху (снизу аналогично).

Для каждого
$n\in\mathbb{N}$ существует $a_n\in[a,b]$ такое, что
\(
  f(a_n) > n.
\)

По теореме Больцано-Вейерштрасса последовательность $a_n$ имеет конечный частичный предел $x_0 \in [a,b]$

Так как $f$ непрерывна в точке $x_0$, то из $a_{n_k}\to x_0$ следует
\(
  f(a_{n_k})\to f(x_0)
\)
при $k\to\infty$.
Но по построению $f(a_{n_k})>n_k\ge k$, то есть $f(a_{n_k})\to+\infty$,
что противоречит существованию конечного предела $f(x_0)$.
Следовательно, наше предположение неверно, и множество $f([a,b])$
ограничено сверху.

\Endproof

\Theor{Вейерштрасса}

Пусть $f\colon [a,b]\to\mathbb{R}$ непрерывна на отрезке $[a,b]$.
Тогда существуют точки $x_m,x_M\in[a,b]$ такие, что
\[
  f(x_M)=\sup_{x\in[a,b]} f(x),\qquad
  f(x_m)=\inf_{x\in[a,b]} f(x).
\]

\Proof

Покажем для супремума, для инфимума аналогично

Так как $f([a,b])$ ограничено сверху, существует
его точная верхняя грань
\[
  M = \sup_{x\in[a,b]} f(x)\in\mathbb{R}.
\]

По определению супремума для каждого $n\in\mathbb{N}$ найдётся такая
точка $x_n\in[a,b]$, что
\(
  M - \frac1n < f(x_n) \le M.
\).
То есть получили $f(x_n) \to M$. Выберем из них сходящуюся подпоследовательность по теореме Больцано-Вейерштрасса $x_{n_k} \to x_M$. По непрерывности функции $f(x_{n_k}) \to f(x_M)$.
Но по построению $f(x_{n_k}) \to M$. В силу единственности предела $f(x_M) = M$. Что и требовалось.

\Endproof