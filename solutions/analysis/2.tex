\subsection{Ограниченность функции, непрерывной на отрезке, достижение точных верхней и нижней граней.}

\Def $\hat \R := \R \cup \{ -\infty, +\infty, \infty \}$

\Def Пусть $x_0, A \in \hat\R, \delta_0 > 0, f: \mathring U_{\delta_0}(x_0) \to \R$. Будем говорить, что $A$ - предел функции $f$ в точке $x_0$ и записывать $\lim\limits_{x \to x_0} f(x_0) = A$, если
\begin{equation*}
    \forall \varepsilon > 0 \; \exists \delta_\varepsilon < \delta_0: \forall x \in \mathring U_{\delta_\varepsilon}(x_0) \hookrightarrow f(x) \in U_\varepsilon(A)
\end{equation*}

\Def Последовательностью Гейне в точке $x_0 \in \hat\R$ называется такая последовательность $\{x_n\} \subset \R$:
\begin{enumerate}
    \item $x_n \to x_0, n \to \infty$
    \item $\forall n \in N \hookrightarrow x_n \neq x_0$
\end{enumerate}

\Def Пусть $x_0 \in \hat\R, A \in \hat\R$. Пусть $f: \mathring U_{\delta_0}(x_0) \to \R$. 
Будем говорить, что $A$ - предел функции $f$ в точке $x_0$ и записывать $\lim\limits_{x \to x_0} f(x_0) = A$, если для всякой последовательности Гейне $\{x_n\} \subset \mathring U_{\delta_0}(x_0)$ в точке $x_0$ выполнено, что $f(x_n) \to A, n \to \infty$.

\Th Определения пределов функций выше эквивалентны.

\Def Пусть $\delta_0 > 0$, $x_0 \in \mathbb{R}$, $f: U_{\delta_0} (x_0) \to \R$. Будем говорить, что $f$ непрерывна в точке $x_0$, если $\exists \lim\limits_{x \to x_0} f(x) = f(x_0)$.

\Def Пусть $\delta_0 > 0$, $x_0 \in \mathbb{R}$, $f: U_{\delta_0} (x_0) \to \R$.Будем говорить, что $f$ разрывна в точке $x_0$, если она не непрерывна в точке $x_0$.

\Note Не входит как в таковой билет, но в лекции говорится, что надо знать как <<отче наш>>. Пусть $\delta_0 > 0$, $x_0 \in \mathbb{R}$, $f: U_{\delta_0} (x_0) \to \R$.
\begin{enumerate}
    \item \textit{Устранимый разрыв.} $\exists \lim\limits_{x \to x_0} f(x) \in \R$, но $\lim\limits_{x \to x_0} f(x) \neq f(x_0)$.
    \item \textit{Разрыв первого рода.} $\exists \lim\limits_{x \to x_0 - 0} f(x) \in \R, \exists \lim\limits_{x \to x_0 + 0} f(x) \in \R$, но $\lim\limits_{x \to x_0 - 0} f(x) \neq \lim\limits_{x \to x_0 + 0} f(x)$.
    \item \textit{Разрыв второго рода.} Хотя бы один из односторонних пределов не существует, либо бесконечен.
\end{enumerate}

\Def Пусть $X \subset \R, f: X \to \R$. Будем говорить, что $f$ непрерывна на $X$, если она непрерывна в каждой точке.

\Def Класс всех непрерывных функций на $X$ будем обозначать $C(X)$.

\Th Пусть $K \subset \R$ - компакт, $f \in C(K)$. Тогда $f(K)$ тоже компакт.

\Proof

Пусть $\{y_n\} \subset f(K) \Rightarrow \exists \{x_n\} \subset K: \forall n \in \N \hookrightarrow f(x_n) = y_n \Rightarrow \exists \{x_{n_j}\} \subset K: \lim\limits_{j \to \infty} x_{n_j} = x^* \in K \Rightarrow \lim\limits_{j \to \infty} f(x_{n_j}) = f(x^*) = y^* \in f(K)$.

То есть для любой последовательности $\{y_n\}$ в $f(K)$ найдется подпоследовательность $\{y_{n_j}\}$, такая что её предел сходится к $y^* \in f(K)$. Значит $f(K)$ - компакт, по определению.

\Endproof
