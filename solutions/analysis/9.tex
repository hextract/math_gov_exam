\subsection{Экстремумы функций нескольких переменных. Необходимые условия, достаточные условия.}

\Note Здесь и далее $E \subset \R^m$ -- непустое множество.

\Def Пусть $x^0 \in E$, $f: E \rightarrow \R$. Будем говорить, что $x^0$ -- точка строгого локального экстремума $f$ на $E$, если $\exists \delta > 0: \;\forall x \in \mathring U_\delta (x^0) \cap E \hookrightarrow f(x) > \textit{(<)} f(x^0)$.

В случае нестрогих неравенств, определяется точка нестрого локального экстремума.

\textbf{Необходимое условие безусловного эктремума}

\Th Пусть $E$ -- открыто, $f: E \rightarrow \R$; $\forall i \in \{1, \dots, m\} \;\exists \frac{\partial f}{\partial x_i}(x^0) \in \R$. 

Тогда если $x^0$ -- точка локального экстремума, то $\frac{\partial f}{\partial x_i}(x^0) = 0 \;\forall i \in \{1, \dots, m\}$.

\Proof

Так как $E$ открыто, а $x^0$ -- точка локального экстремума (скажем, минимума), 
то $\exists \delta > 0: B_\delta(x^0) \subset E$ и $\forall x \in B_\delta(x^0) \; f(x) \geq f(x^0)$.

Фиксируем $i \in \{1, \dots, m\}$. Введем $g_i(t) = f(x_1^0, \dots, x_{i-1}^0, t, x_{i+1}^0, \dots, x_m^0)$. Эта функция имеет локальный экстремум в т. $x_i^0$ и имеет производную $\frac{\partial g_i}{\partial t}(x_i^0) = \frac{\partial f}{\partial x_i}(x^0)$, которая в силу теоремы для одномерного случая равна 0. Это верно для любого $i$.

\Endproof

\textbf{Достаточное условие безусловного эктремума}

\Def Квадратичной формой в $\R^m$ назовем функцию $K: \R^m \rightarrow \R$

\begin{equation*}
    K(h) \coloneqq \sum_{i=1}^m \sum_{j=1}^m  a_{ij} h_i h_j = h^T A h.
\end{equation*}

Так вот: если $f$ дважды дифференцируема в $x^0$ и все частные производные 2-го порядка непрерывны в точке $x^0$, то второй дифференциал, который по определению записывается как

\begin{equation*}
    d^2 f(x^0) \coloneqq \sum_{i=1}^m \sum_{j=1}^m \frac{\partial^2 f}{\partial x_i \partial x_j}(x^0) dx_i dx_j,
\end{equation*}

представляет собой квадратичную форму с матрицей вторых производных (которая в силу их непрерывности симметрична), которая называется матрицей Гессе.

\Def $K(h)$ 

\begin{itemize}
    \item положительно определена, если $\forall h \neq 0 \hookrightarrow K(h) > 0$;
    \item отрицательно определена, если $\forall h \neq 0 \hookrightarrow K(h) < 0$;
    \item знаконеопределена, если $\exists h_1, h_2 \neq 0: \; K(h_1) > 0,\; K(h_2) < 0$;
\end{itemize}

\Theorbd{Критерий Сильвестра.}

Симметричная квадратичная форма положительно определена $\iff$ все главные миноры ее матрицы положительны.

Симметричная квадратичная форма отрицательно определена $\iff$ знаки главных миноров чередуются, начиная с минуса.

\textit{Миноры -- это определители квадратных подматриц, расположенных в левом верхнем углу.}

\Theor{Достаточное условие безусловного экстремума}

Пусть $f \in C^2(B_{\delta_0}(x^0))$. Пусть $\forall i \in \{1, \dots, m\}\; \frac{\partial f}{\partial x_i}(x^0) = 0$.

Тогда 

\begin{itemize}
    \item Если квадратичная форма $d^2 f(x^0)$ положительно/отрицательно определена, то $x^0$ -- точка локального минимума/максимума соответственно.
    \item Если квадратичная форма $d^2 f(x^0)$ знаконеопределена, то экстремума нет.
    \item Иначе ничего сказать нельзя.
\end{itemize}

\Proof

\begin{gather*}
    f(x) = f(x^0) + d_{x^0} f(dx) + \frac{1}{2} d^2_{x^0} f(dx) + o(||dx||^2) = \\
    f(x^0) + \frac{d^2_{x^0} f(dx)}{2} + o(||dx||^2), x \rightarrow x^0 .
\end{gather*}

Рассмотрим случай, где квадратичная форма положительно определена. Тогда, как непрерывная функция, на единичной сфере $S_1^{m-1}$ (компакт) она достигает своего инфимума $m>0$:

\begin{equation*}
    \forall dx \in S_1^{m-1}(0) \; d^2_{x^0} f(dx) \geq m.
\end{equation*}

\begin{equation*}
    \forall dx \neq 0 \; d^2_{x^0} f(dx) = ||dx||^2 \sum_i \sum_j \frac{\partial^2 f}{\partial x_i \partial x_j}(x^0) \frac{dx_i}{||dx||} \frac{dx_j}{||dx||} \geq m ||dx||^2.
\end{equation*}

Хотим дать оценку снизу на 

\begin{equation*}
    f(x) = f(x^0) + \frac{d^2_{x^0} f(dx)}{2} + o(||dx||^2)
\end{equation*}

$\frac{d^2_{x^0} f(dx)}{2}$, как мы выяснили, можно оценить снизу $\frac{m}{2}||dx||^2$

Третье слагаемое в достаточно малом шаре $\mathring{B}_{\hat{\delta}}(x^0)$ можно оценить снизу как $-m/4 \cdot ||dx||^2$ (оно о-малое, константа может быть любой). Тогда итоговая оценка будет выглядеть так:

\begin{equation*}
\forall x \in \mathring{B}_{\hat{\delta}}(x^0) \hookrightarrow f(x) \geq f(x^0) + \frac{1}{2} m ||dx||^2 - \frac{1}{4} m ||dx||^2 = f(x^0) + \frac{1}{4} m ||x-x^0||^2
\end{equation*}

А это уже означает то, что мы получили минимум.

Случай отрицательно определенной формы доказывается в точности аналогично, а также может быть сведен заменой $g = -f$ к доказанному выше.

Для случая знаконеопределенности ($\exists dx^1, dx^2 \neq 0: d^2_{x^0} f(dx^1) > 0, d^2_{x^0} f(dx^2) < 0$) достаточно расписать 

\begin{gather*}
    f(x^0 + t dx^1) - f(x^0) = \frac{t^2}{2} d^2_{x^0} f(dx^1) + o(t^2), t \rightarrow 0; \\
    f(x^0 + t dx^2) - f(x^0) = \frac{t^2}{2} d^2_{x^0} f(dx^2) + o(t^2), t \rightarrow 0;
\end{gather*}

\begin{gather*}
    \exists \delta>0: \;\forall t \in (0, \delta) \hookrightarrow f(x^0 + t dx^1) < f(x^0) < f(x^0 + t dx^2).
\end{gather*}

\Endproof

\textit{Примеры по 3-му пункту: $f_1(x_1, x_2) = x_1^4$, $f_2(x_1, x_2) = x_1^3$. В обоих случаях квадратичные формы зануляются в $(0, 0)$, но в одном случае это точка экстремума, а в другом нет.}


