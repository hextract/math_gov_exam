\subsection{Теоремы о среднем Ролля, Лагранжа и Коши для дифференцируемых функций.}

\Def Пусть $f: U_{\delta_0} (x_0) \to \R$. Тогда производной функции $f$ точке $x_0$ будем называть следующий предел:
\begin{equation*}
    f'(x_0) = \lim\limits_{x \to x_0} \frac{f(x) - f(x_0)}{x - x_0}
\end{equation*}

\Def Будем говорить, что $f: U_{\delta_0} (x_0) \to \R$ дифференцируема в точке $x_0$ $\iff$ $\exists f'(x_0) \in \R$.

\Def Пусть $f: X \to \R$. Будем говорить, что $x_0 \in X$ - точка локального минимума, если выполнены следующие условия:
\begin{equation*}
    \exists \delta_0 > 0: \forall x \in U_{\delta_0}(x_0) \hookrightarrow f(x_0) \leq f(x)
\end{equation*}

Аналогично определяется локальный максимум, строгий локальный минимум/максимум.

\Lemma Пусть $f: [a, b] \to \R$. Пусть $x_0$ - точка локального минимума.
\begin{itemize}
    \item $\exists f'_+ (x_0) \in \R \Rightarrow f'_+(x_0) \geq 0$
    \item $\exists f'_- (x_0) \in \R \Rightarrow f'_-(x_0) \leq 0$
\end{itemize}

\Proof

Покажем первый пункт, т.к. второй доказывается аналогично.

По определению локального минимума, $\exists \delta_0 > 0: \forall x \in U_{\delta_0}(x_0) \hookrightarrow f(x) \geq f(x_0) \iff f(x) - f(x_0) \geq 0$.

Значит верно, что:
\begin{equation*}
    \forall x \in (x_0, x_0 + \delta_0) \hookrightarrow \frac{f(x) - f(x_0)}{x - x_0} \geq 0
\end{equation*}

Перейдя к пределу в неравенстве:
\begin{equation*}
    f'_+(x_0) \overset{def}{=} \lim\limits_{x \to x_0 + 0} \frac{f(x) - f(x_0)}{x - x_0} \geq 0
\end{equation*}

\Endproof

\Def Пусть $f: U_{\delta_0} (x_0) \to \R$. Будем говорить, что $x_0$ - точка локального экстремума, если и только если она точка локального минимума, либо максимума.

\Theor{Ферма} Пусть $f: U_{\delta_0} (x_0) \to \R$ дифференцируема в $x_0$. Тогда коль скоро $x_0$ - точка локального экстремума, то $f'(x_0) = 0$.

\Proof

Для определённости будем считать, что $x_0$ - точка локального максимума (для минимума аналогично).

$f$ дифференцируема в $x_0$, тогда $0 \leq f'_-(x_0) = f'(x_0) = f'_+(x_0) \leq 0 \Rightarrow f'(x_0) = 0$.

\Endproof


Для локального максимума формулируется аналогичная теорема.

\Theor{Ролля о среднем} Пусть $f \in C([a, b], \R)$, $f$ - дифференцируема на $(a, b)$, а также $f(a) = f(b)$. 
Коль скоро это так, то $\exists \xi \in (a, b): f'(\xi) = 0$.

\Proof

Случай $f \overset{[a, b]}{\equiv} C, C \in \R$ тривиален. Поэтому будем считать функцию неконстантой.

В силу непрерывности на отрезке, $f$ достигает наибольшего и наименьшего значений на $[a,b]$. Так как $f(a)=f(b)$ и $f$ не константа, хотя бы одна из точек достижения max/min лежит внутри $(a,b)$; обозначим её $\xi\in(a,b)$. Тогда $\xi$ — точка локального экстремума и по теореме Ферма $f'(\xi)=0$.

\Endproof

\Theor{Коши о среднем} Пусть $f, g \in C([a, b], \R)$, $f, g$ - дифференцируемы на $(a, b)$. Также $g(b)\neq g(a)$ и $\forall x \in (a, b) \hookrightarrow g'(x) \neq 0$. 
Тогда $\exists \xi \in (a, b): \frac{f(b) - f(a)}{g(b) - g(a)} = \frac{f'(\xi)}{g'(\xi)}$.

\Proof

Положим $h(x) = f(x) - \frac{f(b) - f(a)}{g(b) - g(a)} g(x)$. Заметим следующее:

\begin{multline*}
    h(a) = f(a) - \frac{f(b) - f(a)}{g(b) - g(a)} g(a) = \frac{f(a) g(b) - f(a) g(a) - g(a) f(b) + g(a) f(a)}{g(b) - g(a)} = \frac{g(a)f(a) - g(a) f(b)}{g(b) - g(a)} = \\
    = \frac{f(a) g(b) - f(b) g(b) - g(a) f(b) + g(b) f(b)}{g(b) - g(a)} =
    \frac{(f(a) g(b) - f(b) g(b)) - (g(a) f(b) - g(b) f(b))}{g(b) - g(a)} = \\
    =\frac{(f(a) - f(b)) g(b) - (g(a) - g(b)) f(b)}{g(b) - g(a)} = f(b) - \frac{f(b) - f(a)}{g(b) - g(a)} g(b) = h(b)
\end{multline*}

Заметим, что выполняются условия теоремы Ролля для $h$, значит $\exists \xi \in (a, b): h'(\xi) = 0$:
\begin{equation*}
    h'(\xi) = f'(\xi) - \frac{f(b) - f(a)}{g(b) - g(a)} g'(\xi) = 0
\end{equation*}
\begin{equation*}
    f'(\xi) = \frac{f(b) - f(a)}{g(b) - g(a)} g'(\xi)
\end{equation*}
\begin{equation*}
    \frac{f'(\xi)}{g'(\xi)} = \frac{f(b) - f(a)}{g(b) - g(a)}
\end{equation*}

\Endproof

\Theor{Лагранжа о среднем} Пусть $f \in C([a, b], \R)$ и $f$ дифференцируема на $(a, b)$.
Тогда $\exists \xi \in (a, b): \frac{f(b) - f(a)}{b - a} = f'(\xi)$.

\Proof

Применяем теорему Коши о среднем для $f$ и $g(x) = x$ и получаем требуемое.

\Endproof
