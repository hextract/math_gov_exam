\subsection{Теорема о равномерной непрерывности функции, непрерывной на компакте.}

\Def Множество $A$ метрического пространства $(X, \rho_X)$ называется \textit{компактным} (или \textit{компактом}), если из любой последовательности $\{x_k\} \subset A$ можно выделить подпоследовательность $\{x_{k_j}\}$, сходящуюся к некоторой точке $c \in A$.

\Theor{Критерий компактности в $\mathbb{R}^n$ (лемма Гейне-Бореля)} 
Подмножество $A \subset \mathbb{R}^n$ является компактным тогда и только тогда, когда $A$ замкнуто и ограничено.

\Note
В произвольных метрических пространствах замкнутость и ограниченность множества недостаточны для компактности. Общий критерий гласит: 
\textit{подмножество полного метрического пространства компактно тогда и только тогда, когда оно замкнуто и вполне ограничено} (для любого $\eps > 0$ существует конечное число точек, образующих $\eps$-сеть для множества). 
В $\mathbb{R}^n$ полная ограниченность эквивалентна обычной ограниченности, что делает лемму Гейне-Бореля частным случаем общего критерия.

\Def Функция $f: X \to Y$ (где $X, Y$ --- произвольные метрические пространства) называется равномерно непрерывной на $M \subset X$, если

\begin{equation*}
  \forall \eps > 0 \;\exists\delta > 0: \;\forall x_1, x_2 \in M \hookrightarrow \left(\rho_X(x_1, x_2) < \delta\right) \implies \left(\rho_Y\left(f(x_1), f(x_2)\right)
  < \eps\right).
\end{equation*}

\Th Пусть $(X, \rho_X), (Y, \rho_Y)$ --- метрические пространства; $A \subset X$ --- компакт; $f: A \to Y$ непрерывна.

Тогда $f$ равномерно непрерывна на $A$.

\Proof

Предположим, что $f$ -- непрерывна на компакте $A$, но не равномерно непрерывна на нем. Тогда из отрицания определения

\begin{equation*}
  \exists \eps > 0 \;\forall \delta > 0 \;\exists x, y \in A: \rho_X(x, y) < \delta \text{ и } \rho_Y(f(x), f(y)) \geq \eps.
\end{equation*}

Раз это верно для любого $\delta$, то построим последовательности $\{x_k\}, \{y_k\}$, такие что $\rho_X(x_k, y_k) < \frac{1}{k}$ и $\rho_Y(f(x_k), f(y_k)) \geq \eps$ для
$\forall k \in \N$.

Раз $A$ -- компакт, из $\{x_k\}$ можно выделить сходящуюся в $A$ подпоследовательность: $x_{k_j} \to c \in A$.

А так как расстояние между соответствующими членами подпоследовательностей стремится к 0, то, воспользовавшись неравенством треугольника, получаем, что соответствующие
подпоследовательности стремятся к одной точке:

\begin{equation*}
  \rho_X(y_{k_j}, c) \leq \rho_X(y_{k_j}, x_{k_j}) + \rho_X(x_{k_j}, c)
\end{equation*}

Итого $y_{k_j} \to c$.

Так как $f$ непрерывна, $f(x_{k_j}) \to f(c), f(y_{k_j}) \to f(c) \implies \rho_Y(f(x_{k_j}), f(y_{k_j})) \to 0$, что противоречит предположению, что $\rho_Y(f(x_k),
f(y_k)) \geq \eps$.

\Endproof
