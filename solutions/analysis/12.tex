\subsection{Степенные ряды. Радиус сходимости. Бесконечная дифференцируемость суммы степенного ряда. Ряд Тейлора.}

\Note Выкладки, в общем-то аналогичны для случая $c_n, z_0 \in \R$. Редкозубов вводит вещественные, а не комплексные ряды.

\textbf{Степенные ряды и радиус сходимости}

\Def Степенной ряд -- функциональный ряд вида 

\begin{equation*}
    \sum_{n=0}^{\infty} c_n (z-z_0)^n,
\end{equation*}

где $c_n, z_0 \in \Cmp$, а $z$ -- комплексное переменное.

\Def Радиусом сходимости степенного ряда называется

\begin{equation*}
    R = \frac{1}{\limsup_{n \to \infty} \sqrt[n]{|c_n|}}, \quad 0 \leq R \leq +\infty.
\end{equation*}

При этом если предел равен $0$ или $+\infty$, то $R$ считается равным $+\infty$ или $0$ соответственно.

\Def Кругом сходимости степенного ряда называется круг 

\begin{equation*}
    \{z \in \Cmp: |z - z_0| < R\}.
\end{equation*}

При этом принимается, что при $R = +\infty$ круг сходимости равен $\Cmp$, а при $R = 0$ он является пустым множеством.

\textit{Эта формула называется формулой Коши-Адамара.}

Вспомним признак Коши сходимости ряда с неотрицательными членами.

\Theor{Признак Коши}

Пусть $\sum_{n=1}^{\infty} b_n$ -- числовой ряд, где $b_n \geq 0 \;\forall n \in \N$, а $q = \limsup_{n \to \infty} \sqrt[n]{b_n}$.

Тогда

\begin{enumerate}
    \item при $q < 1$ ряд $\sum_{n=1}^{\infty} b_n$ сходится;
    \item при $q > 1$ ряд $\sum_{n=1}^{\infty} b_n$ расходится.
\end{enumerate}

Применим этот признак для исследования абсолютной сходимости степенного ряда:

\begin{equation*}
    q = \limsup_{n \to \infty} \sqrt[n]{|c_n z^n|} = |z| \limsup_{n \to \infty} \sqrt[n]{|c_n|} = \frac{|z|}{R}.
\end{equation*}

Отсюда получаем теорему о радиусе сходимости:

\Theor{О радиусе сходимости} Если $R$ -- радиус сходимости степенного ряда, то 

\begin{enumerate}
    \item при $|z| < R$ ряд сходится, причем абсолютно;
    \item при $|z| > R$ ряд расходится.
\end{enumerate}

\Note При $z = R$, т. е. на границе круга сходимости, ряд может как сходиться, так и расходиться.

\Theor{О равномерной сходимости степенного ряда}

Пусть $R$ -- радиус сходимости степенного ряда. 

Тогда $\forall r \in (0, R)$ ряд сходится равномерно на замкнутом круге $\{z \in \Cmp: |z| \leq r\}$.

\Proof

Имеем $|c_n z^n| \leq |c_n| r^n$ при $|z| \leq r$.

Если рассмотреть числовой ряд $\sum_{n=1}^{\infty} |c_n|r^n$, то он сходится (поскольку $r < R$) по определению радиуса сходимости. То есть мы ограничили наш функциональный ряд сходящимся числовым. 

По теореме Вейершрасса, исходный ряд сходится равномерно на круге $\{z \in \Cmp: |z| \leq r\}$.

\Endproof

\textbf{Дифференцирование и интегрирование степенного ряда}

\Lemma Радиус сходимости ряда $\sum_{k=0}^{\infty} c_k z^k$ равен радиусу сходимости рядов $\sum_{k=1}^{\infty} kc_k z^{k-1}$ и $\sum_{k=0}^{\infty} \frac{c_k z^{k+1}}{k+1}$.

\Proof

Сначала покажем, что радиус сходимости $\hat{R}$ ряда $\sum_{k=1}^{\infty} k c_k z^k$ равен радиусу сходимости $R$ ряда $\sum_{k=0}^{\infty} c_k z^k$:

\begin{equation*}
    \frac{1}{\hat{R}} = \limsup_{k \to \infty} \sqrt[k]{|k c_k|} = \limsup_{k \to \infty} \sqrt[k]{k} \cdot \limsup_{k \to \infty} \sqrt[k]{|c_k|} = \limsup_{k \to \infty} \sqrt[k]{|c_k|} = \frac{1}{R},
\end{equation*}

так как $\lim\limits_{k \to \infty} \sqrt[k]{k} = \lim_{k \to \infty} e^{\frac{\ln k}{k}} = 1$.

А теперь, предполагая, что $z \neq 0$ (случай $z=0$ тривиален), рассмотрим частичные суммы $S_n = \sum_{k=1}^n c_k k z^{k-1}$ и $\hat{S}_n = \sum_{k=1}^n c_k k z^k$.

Так как $\hat{S}_n = S_n z$, то $\exists \lim\limits_{n \to \infty} S_n \iff \exists \lim\limits_{n \to \infty} \hat{S}_n$, причем если они существуют, то они $S = \hat{S} z$.

Для формального дифференцирования доказали. А для интегрирования, мы только что показали, что радиус сходимости исходного ряда обязан быть равен радиусу сходимости формально продифференцированного, откуда радиус сходимости исходного ряда обязан быть равен радиусу сходимости формально проинтегрированного.

\Endproof

Далее будем работать с вещественными степенными рядами

\Theor{О почленном интегрировании и дифференцировании степенного ряда}

Пусть $\{c_k\} \subset \R$ и $x_0 \in \R$. Пусть внутри круга сходимости степенной ряд сходится к некоторой функции $f$: $f(x)=\sum_{k=0}^{\infty} c_k (x-x_0)^k$.

Тогда внутри интервала сходимости ($\forall x \in (x_0 - R, x_0 + R)$) степенной ряд

\begin{enumerate}
    \item можно почленно интегрировать, то есть 
    \begin{equation*}
        \int_{x_0}^x f(t)dt = \sum_{k=0}^{\infty} \frac{c_k}{k+1} (x-x_0)^{k+1};        
    \end{equation*}

    \item можно сколь угодно много раз почленно дифференцировать, то есть 
    \begin{equation*}
        \forall n \in \N \; f^{(n)}(x) = \sum_{k=n}^{\infty} k(k-1)\dots(k-(n-1)) c_k (x-x_0)^{k-n};    
    \end{equation*}

    \item $c_k = \frac{f^{(k)}(x_0)}{k!}$.
\end{enumerate}

\Proof

\textbf{Интегрирование}: Так как $c_k (x-x_0)^k$ -- непрерывная функция от $x$ для любого $k \in \N$, а $\forall r \in (0, R)$ ряд сходится равномерно на $[x_0 - r, x_0 + r]$ по теореме о круге сходимости, то мы можем проинтегрировать сумму $f(x)$ в любой точке $x \in [x_0 - r, x_0 + r]$:

\begin{equation*}
    \int_{x_0}^x f(t)dt = \sum_{k=0}^{\infty} c_k \int_{x_0}^x (t - x_0)^k dt = \sum_{k=0}^{\infty} \frac{c_k}{k+1} (x-x_0)^{k+1}.
\end{equation*}


\textbf{Дифференцирование}: зафиксируем $r \in (0, R)$. По предыдущей лемме ряд $\sum_{k=1}^{\infty} c_k k (x-x_0)^{k-1}$ имеет тот же радиус сходимости, что и исходный ряд, а значит сходится равномерно на $[x_0 - r, x_0 + r]$. Также заметим, что выражение под суммой является производной по $x$ выражения под суммой в исходном ряду.

По теореме о дифференцировании функционального ряда, предельная функция дифференцируема и ряд можно почленно дифференцировать, если ряд из производных сходится равномерно, а ряд исходных функций сходится хотя бы в одной точке. Мы попадаем в эти условия. Получаем, что $\forall x \in (x_0 - R, x_0 + R) \;\exists f'(x) = \sum_{k=1}^{\infty} c_k k (x-x_0)^{k-1}$.

Далее по индукции получаем дифференцируемость сколь угодно много раз.

\textbf{Коэффициенты Тейлора}: Покажем, что $c_k = \frac{f^{(k)}(x_0)}{k!}$. 

Для $k=0$: $\frac{f(x_0)}{1!} = \sum_{k=0}^{\infty} c_k (x_0 - x_0)^k = c_k$.

Запишем $n$-ю производную:

\begin{equation*}
    f^{(n)}(x) = \sum_{k=n}^{\infty} c_k k(k-1)\dots(k-(n-1)) (x-x_0)^{k-n}
\end{equation*}

Возьмем $x=x_0$. 
Тогда для всех $k > n$, $(x-x_0)^{k-n} = 0$.
А значит 

\begin{equation*}
    f^{(n)}(x_0) = c_n n(n-1)\dots(n-(n-1)) = c_n n!
\end{equation*}

Откуда искомое $c_n = \frac{f^{(n)}(x_0)}{n!}$.

\Endproof

\textbf{Ряд Тейлора}

\Def Пусть $\exists f^{(n)}(x_0) \in \R$. Тогда

\begin{itemize}
    \item $T_{x_0}^n [f](x) \coloneqq \sum_{k=0}^n \frac{f^{(k)}(x_0)}{k!} (x-x_0)^k$ -- полином Тейлора;
    \item $r_{x_0}^n [f](x) \coloneqq f(x) - T_{x_0}^n [f](x)$ -- формальный Тейлоровский остаток, или остаточный член формулы Тейлора.
\end{itemize}

\Def Пусть $x_0 \in \R$ и $\forall n \in \N \;\exists f^{(n)}(x_0) \in \R$. Тогда ряд 

\begin{equation*}
    \sum_{n=0}^\infty \frac{f^{(n)}(x_0)}{n!} (x-x_0)^n
\end{equation*}

называется рядом Тейлора с функции $f$ с центром в точке $x_0$.

\Theor{Формула Тейлора}

Пусть $x_0, x \in \R$, $f^{(n+1)} \in C([\min(x_0, x), \max(x_0, x)])$. 

Тогда остаточный член формулы Тейлора можно представить:

\begin{itemize}
    \item \textit{в интегральной форме}
    \begin{equation*}
        r_{x_0}^n[f](x) = \frac{1}{n!} \int_{x_0}^x (x-t)^n f^{(n+1)}(t)dt,
    \end{equation*}

    \item \textit{в форме Лагранжа}
    \begin{equation*}
        r_{x_0}^n[f](x) = \frac{f^{(n+1)}(\xi)}{(n+1)!}(x-x_0)^{n+1}, \quad \xi \in (\min(x_0, x), \max(x_0, x)),
    \end{equation*}

    \item \textit{в форме Пеано}
    \begin{equation*}
        r_{x_0}^n[f](x) = o((x-x_0)^n), \quad x \to x_0.
    \end{equation*}
\end{itemize}

\Proof

Пусть для определенности $x > x_0$.

\textbf{Интегральная форма}: покажем, что 

\begin{equation*}
    f(x) = \sum_{k=0}^n \frac{f^{(k)}(x_0)}{k!} (x-x_0)^k + \frac{1}{n!}\int_{x_0}^x (x-t)^n f^{(n+1)}(t)dt.
\end{equation*}

При $n=0$ формула верна, так как совпадает с формулой Ньютона-Лейбница: $f(x) = f(x_0) + \int_{x_0}^x f'(t)dt$.

Предположим, что формула верна при $n-1$:

\begin{equation*}
    f(x) = \sum_{k=0}^{n-1} \frac{f^{(k)}(x_0)}{k!} (x-x_0)^k + \frac{1}{(n-1)!}\int_{x_0}^x (x-t)^{(n-1)} f^{(n)}(t)dt.
\end{equation*}

Проинтегрируем интеграл в правой части по частям:

\begin{gather*}
    \frac{1}{(n-1)!} \int_{x_0}^x (x-t)^{(n-1)} f^{(n)}(t)dt = \\
    = \left(-\frac{1}{n!}f^{(n)}(t) (x-t)^n\right)\Big|_{t=x_0}^{t=x} + \frac{1}{n!}\int_{x_0}^x (x-t)^n f^{(n+1)}(t)dt = \\
    = \frac{1}{n!} f^{(n)}(x_0) (x-x_0)^n + \frac{1}{n!} \int_{x_0}^x (x-t)^n f^{(n+1)}(t)dt.
\end{gather*}

Подставляя этот интеграл, получаем искомый шаг индукции.

\textbf{Форма Лагранжа}: Применим к интегралу из интегральной формы интегральную теорему о среднем:

\begin{gather*}
    r_{x_0}^n [f](x) = \frac{f^{(n+1)}(\xi)}{n!}\int_{x_0}^x (x-t)^n dt = \\
    = \frac{f^{(n+1)}(\xi)}{(n+1)!}(x-x_0)^{n+1},
\end{gather*}

где $\xi$ лежит между $x$ и $x_0$.

\textbf{Форма Пеано}: так как $f^{(n+1)}$ непрерывно в точке $x_0$, то она ограничена в некоторой окрестности этой точки: $\exists \delta > 0, M > 0: |f^{(n+1)}(t)| \leq M \;\forall x \in U_{\delta}(x_0)$.

Тогда, ограничив $f^{(n+1)}$ сверху константой $M$ в $\delta$-окрестности точки $x_0$, покажем, что остаточный член в форме Лагранжа представим в форме Пеано:

\begin{gather*}
    \frac{r_{x_0}^n [f](x)}{|x-a|^n} = \frac{f^{(n+1)}(\xi)}{(n+1)!}|x-x_0| \leq \frac{M}{(n+1)!}|x-x_0| \to 0, \quad x \to x_0.
\end{gather*}

Итого, $r_{x_0}^n [f](x) = o((x-x_0)^n), x \to x_0$.

\Endproof
