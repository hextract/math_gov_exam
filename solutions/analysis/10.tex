\subsection{Свойства интеграла с переменным верхним пределом (непрерывность, дифференцируемость). Формула Ньютона-Лейбница.}

\textbf{Определенный интеграл Римана}

\textit{Напомним вкратце самую базу про определенный интеграл Римана}.

\Def Разбиение отрезка $[a, b]$ ($a < b$) -- конечный набор точек $T=\{x_i\}_{i=0}^N$, т.ч. $a = x_0 < x_1 < \dots < x_N = b$.

\Def Мелкость разбиения $l(T) = \max_{i \in \{1, \dots, N\}} (x_i - x_{i-1})$.

\Def Для $f: [a, b] \rightarrow \R$; $T$ -- разбиение $[a, b]$

Нижняя сумма Дарбу: $s(f, T) \coloneqq \sum_{i=1}^N m_i (x_i - x_{i-1})$, где $m_i \coloneqq \inf_{x \in [x_{i-1}, x_i]} f(x)$;

Верхняя сумма Дарбу: $S(f, T) \coloneqq \sum_{i=1}^N M_i (x_i - x_{i-1})$, где $M_i \coloneqq \sup_{x \in [x_{i-1}, x_i]} f(x)$;

Нижний интеграл Дарбу: $J_* (f) \coloneqq \sup_{T} s(f, T)$.

Верхний интеграл Дарбу: $J^* (f) \coloneqq \inf_{T} S(f, T)$.

\Def $f: [a, b] \rightarrow \R$ интегрируема по Риману на отрезке $[a, b]$, если 

\begin{gather*}
    \exists J \in \R \; \forall \eps > 0 \;\exists \delta > 0 \;\forall T: l(T) < \delta \hookrightarrow 
    \begin{cases}
        |s(f, T) - J| < \eps, \\
        |S(f, T) - J| < \eps.
    \end{cases}
\end{gather*}

\Theor{Необходимое условие интегрируемости} $f \in R([a, b]) \;\Rightarrow\; f \in B([a, b])$.

\Theor{Критерий Лебега интегрируемости по Риману}. 

\begin{gather*}
    f \in R([a, b]) \iff
    \begin{cases*}
        f \in B([a, b]), \\
        \text{Множество точек разрыва $f$ на [a, b] имеет лебегову меру ноль}.
    \end{cases*}
\end{gather*}

\Theor{Критерий Дарбу интегрируемости по Риману}

$f \in R([a, b]) \iff J_*(f) = J^*(f) \in \R$.

\Def Для разбиения $T$ отрезка $[a, b]$ выборкой назовем конечный набор точек $\{\xi_i\}_{i=1}^{N_T}$, такой что $\forall i \in \{1, \dots, N_T\} \hookrightarrow \xi_i \in [x_{i-1}, x_i]$.

Интегральная сумма Дарбу: $\Sigma(f, T, \xi_T) = \sum_{i=1}^N f(\xi_i) (x_i - x_{i-1})$.

\Theor{Критерий интегрируемости по Риману в терминах интегральных сумм Римана}

$f \in R([a, b]) \iff \exists J \in \R: \;\forall \;\eps > 0 \exists \delta > 0: \forall T: l(T) < \delta \;\forall \xi_T \hookrightarrow |J - \Sigma(f, T, \xi_T)| < \eps$.

\textbf{Свойства определенного интеграла Римана}

Докажем несколько простых свойств, необходимых для доказательства целевых теорем.

\Theor{Линейность $R([a, b])$}

$R([a, b])$ линейно, причем $\int_a^b (\alpha f(x) + \beta g(x)) dx = \alpha \int_a^b f(x) dx + \beta \int_a^b g(x) dx$.

\textit{Для доказательства просто расписываем интегрируемость через интегральные суммы Римана и переходим к пределу при стремлении мелкости к нулю.}

\Theor{Интегрируемость неравенств}

Пусть $f, g \in R([a, b])$; $f(x) \leq g(x) \;\forall x\in [a, b]$. 

Тогда $\int_a^b f(x)dx \leq \int_a^b g(x) dx$.

\Proof

$\forall T \; \forall \xi \hookrightarrow \Sigma(f, T, \xi) \leq \Sigma(g, T, \xi)$.

Переходя к пределу в неравенстве по мелкости стремящейся к 0, получаем искомое.

\Endproof

\Theor{Интегрируемость модуля}

Пусть $f \in R([a, b])$;

Тогда $|f| \in R([a, b])$, и при этом $\left|\int_a^b f(x)dx\right| \leq \int_a^b |f(x)| dx$.

\Proof

По критерию Лебега, $f$ ограничена и множество точек разрыва имеет меру 0. Очевидно, то же верно для $|f|$. Тогда по критерию Лебега, $|f| \in R([a, b])$.

При этом
$\begin{cases*}
    f(x) \leq |f(x)|, \\ -f(x) \leq |f(x)|,
\end{cases*}$
откуда по предыдущей теореме
$\begin{cases*}
    \int_a^b f(x) dx \leq \int_a^b |f(x)| dx, \\
    \int_a^b -f(x) dx \leq \int_a^b |f(x)| dx.
\end{cases*}$

Используя линейность, выносим минус во втором неравенстве и получаем $\left|\int_a^b f(x) dx\right| \leq \int_a^b |f(x)| dx$.

\Endproof

\Theor{Аддитивность по отрезкам}

Пусть $f \in R([a, b])$; $f \in R([b, c]); a < b < c$.

Тогда $f \in R([a, c])$, причем $\int_a^c f(x) dx = \int_a^b f(x) dx + \int_b^c f(x) dx$.

\Theor{Связь определенного и неопределенного интегралов}

\Def $F$ -- первообразная $f$ на $([a, b])$, если $F'(x) = f(x) \;\forall x \in (a, b)$; $F'_+(a) = f(a); F'_-(b) = f(b)$.

\Note Множество функций, имеющих первообразную на $[a, b]$, и множество функций, интегрируемых на $[a, b]$, пересекаются, но не вложены:

\begin{enumerate}
    \item \textbf{Производная} функции $F(x) = \begin{cases*}
        x^2 \sin \frac{1}{x^2}, x \neq 0, \\
        0, x = 0
    \end{cases*}$ имеет первообразную на $[-a, a]$ (очевидно, $F$), но не интегрируема на любом интервале, содержащем 0, потому что неограничена в окрестности нуля.

    \item $f(x) = sign(x)$ интегрируема по Риману по критерию Лебега, но не имеет первообразной на интервале, содержащем 0 (доказывается от противного).
\end{enumerate}

\textbf{Интеграл с переменным верхним пределом}

\Def Интегралом с переменным верхним пределом для функции $f \in R([a, b])$ называют функцию $F(x) \coloneqq \int_a^x f(t) dt$.

\Theor{Непрерывность интеграла с переменным верхним пределом}

$F \in C([a, b])$.

\Proof

$|F(x_1) - F(x_2)| \leq \left|\int_{x_1}^{x_2} f(t) dt\right| \leq \int_{x_1}^{x_2} |f(t)| dt \leq M |x_1 - x_2|$, где $M = \sup_{x \in [x_1, x_2]} |f(x)|$.

\Endproof

\Theor{Непрерывная дифференцируемость интеграла с переменным верхним пределом}

Пусть $f \in C([a, b])$.

Тогда $F \in C^1([a, b])$. 

Причем $F'(x) = f(x) \;\forall x \in [a, b]$ \\(с точностью до соответствующих соглашений об односторонних производных на границах).

\Proof

\begin{gather*}
    \left|\frac{F(x) - F(x^0)}{x - x^0} - f(x^0)\right| = \left|\frac{\int_x^{x^0} f(t) dt}{x - x^0}\right| =
    \left|\frac{\int_x^{x^0} f(t) dt - \int_x^{x^0} f(x^0) dt}{x-x^0}\right| \leq \frac{\int_x^{x^0} |f(t) - f(x^0)| dt}{|x - x^0|}.
\end{gather*}

Из непрерывности, $\forall \eps > 0 \;\exists \delta > 0 \;\forall x \in \mathring U_\delta (x^0) \cap [a, b] \hookrightarrow |f(x) - f(x^0)| < \eps$.

Продолжая неравенство выше, получаем 

\begin{gather*}
    \frac{\int_x^{x^0} |f(t) - f(x^0)| dt}{|x - x^0|} \leq \frac{\eps \int_x^{x^0} dt}{|x - x^0|} = \eps.
\end{gather*}

Итого, $\forall \eps > 0 \;\exists \delta > 0 \;\forall x \in \mathring U_\delta (x^0) \cap [a, b] \hookrightarrow \left|\frac{F(x) - F(x^0)}{x - x^0} - f(x^0)\right| < \eps$, что по определению значит, что $\exists F'(x^0) = f(x^0)$ для произвольной точки $x^0 \in (a, b)$.

Аналогично расписывается для краевых точек $a, b$.

\Endproof

\Consequence Если $f \in C([a, b])$, то любая первообразная имеет вид $F(x) + C$, где $F$ -- как определена выше, а $C\in\R$.

\Consequence (Формула Ньютона-Лейбница)

Пусть $f \in C([a, b])$.

Тогда $\int_a^b f(x) dx = F(b) - F(a)$, где $F$ -- \textbf{произвольная} первообразная $f$ на $[a, b]$.

\Proof

\begin{gather*}
    F(x) = \int_a^x f(t) dt + C. \\
    F(b) - F(a) = \left(\int_a^b f(t)dt + C\right) - \left(\int_a^a f(t)dt + C\right) = \int_a^b f(t)dt.
\end{gather*}

\Endproof