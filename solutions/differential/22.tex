\subsection{Линейные обыкновенные дифференциальные уравнения с переменными коэффициентами. Фундаментальная система решений. Определитель Вронского. Формула Лиувилля -
Остроградского.}

\Def Линейным обыкновенным дифференциальным уравнением с переменными коэффициентами называется уравнение вида

\begin{equation}
  \dot{x} = A(t)x + F(t),
\end{equation}

где $t \in I \subset \R, \; F:I \rightarrow \R^n;\; A(t) \in Matr_{n \times n}(\R) \; \forall t \in I;$.

\Theorbd{О существовании и единственности решения ЗК для этого уравнения}
Пусть $a_{i,j}, F_i \in C(I)$.

Тогда $\forall (t_0, x_0) \in I \times \R^n \; \exists!$ непродолжаемое решение ЗК

\begin{align*}
  \begin{cases}
    \dot{x} = A(t)x + F(t), \\
    x(t_0) = x_0,
  \end{cases}
\end{align*}

которое определено на $I$.

\Remind Уравнение вида $y^{(n)} + a_1(t)y^{(n-1)} + \dots + a_{n-1}(t)y' + a_n(t)y = f(t)$ -- это уравнение, разрешенное относительно старшей производной, и его можно
переписать в виде $\dot{x} = A(t)x + F(t)$.

Далее рассматриваем однородное уравнение: $\dot{x} = A(t)x$, так как частное решение неоднородного находится методом вариации постоянных так же, как и с уравнениями с
постоянными коэффициентами.

\Theorbd{Об изоморфизме}

Пусть $\mathcal{L}$ -- пространство решений.

Тогда $\mathcal{L}$ -- линейное пространство, такое что отображение $\phi_{t_0}:\mathcal{L} \rightarrow \R^n$, определенное как $\phi_{t_0}(x) = x(t_0)$ (это отображение
переводит решение уравнения в значение этого решения в точке $t_0$), является изоморфизмом между $\mathcal{L}$ и $\R^n$.

\Proof

$\phi$ -- биекция, так как $\forall a\in \R^n \;\exists!$ решение $x(t)$, такое что $x(t_0) = a$, по теореме Коши.

$\phi$ -- гомоморфизм, так как сохраняет линейность:

\begin{equation*}
  \phi_{t_0}(c_1 x_1 + c_2 x_2) = (c_1 x_1 + c_2 x_2) (t_0) = c_1 x_1(t_0) + c_2 x_2(t_0) = c_1 \phi_{t_0}(x_1) + c_2\phi_{t_0}(x_2)
\end{equation*}

\Endproof

\Consequence $\dim \mathcal{L} = n$.

\Consequence В качестве ФСР можно взять любой базис в $\mathcal{L}$.

\Note Определение ФСР есть в предыдущем билете.
Кратко, это множество ЛНЗ решений, через которые линейно выражается любое решение.

\Def Вронскиан системы функций $\mathcal{F} = \{f: I \rightarrow \R^n\}$ -- это определитель матрицы, столбцами которой являются значения функций в точке $t$:

\begin{equation*}
  W_{f_1, \dots, f_n}(t) = |f_1(t) \quad f_2(t) \quad\dots\quad f_n(t)| =
  \begin{vmatrix}
    f_{11}(t)& f_{21}(t) & \cdots & f_{n1}(t) \\
    f_{12}(t) & f_{22}(t) & \cdots & f_{n2}(t) \\
    \vdots & \vdots & \ddots & \vdots \\
    f_{1n}(t) & f_{2n}(t) & \cdots & f_{nn}(t)
  \end{vmatrix}
\end{equation*}

\Statement Если $f_1, \dots, f_n$ - ЛЗ, то $W_{f_1, \dots, f_n}(t) \equiv 0$.

Это верно, так как определитель матрицы, имеющей ЛЗ столбцы, равен 0.

\Th Пусть $x_1, \dots, x_n$ - решения уравнения $\dot{x}=A(t)x$. Тогда следующие утверждения эквивалентны:

\begin{enumerate}
  \item $x_1, \dots, x_n$ - ЛЗ;
  \item $W_{x_1, \dots, x_n}(t) \equiv 0$;
  \item $\exists t_i \in I: W_{x_1, \dots, x_n}(t_0) = 0$.
\end{enumerate}

\Proof

$(1\rightarrow 2)$: Из утверждения выше.

$(2\rightarrow 3)$: Тривиально верно.

$(3\rightarrow 1)$: Пусть $\exists t_i \in I: W_{x_1, \dots, x_n}(t_0) = 0$.
Тогда $\exists c_1, \dots, c_n \left(\sum c_i^2 > 0\right): \sum_{k=1}^n c_k x_k(t_0) = 0$.

Рассмотрим $x(t) = \sum_{k=1}^n c_k x_k(t)$ и $\hat{x}(t) \equiv 0$.
Обе функции являются решениями $\dot{x} = A(t)x$, причем $x(t_0) = \hat{x}(t_0) = 0$, а следовательно они совпадают.
Раз $x \equiv 0$, значит $x_1, \dots, x_n$ - ЛЗ.

\Lemma
$\hat{W}(t) = \tr A(t) \cdot W(t)$

\Proof

Пусть $\mathcal{X}(t)$ - матрица, столбцами которой являются $x_1, \dots, x_n$.
Тогда $W(t) = \det \mathcal{X}(t)$.

Помимо этого $\dot{\mathcal{X}} = A\mathcal{X}$.
Обозначив $\xi_k$ -- $k$-ю строку $\mathcal{X}$, а $a^k$ -- $k$-ю строку $A$, это же равенство можно записать как $\dot{\xi_k} = a^k \mathcal{X}$.

По формуле Лейбница для определителя матрицы, его можно расписать через сумму n матриц, в каждой из которых продиференцирована одна строка

\begin{align*}
  \dot{W} =
  \begin{vmatrix}
    \dot{\xi_1} \\
    \xi_2 \\
    \vdots \\
    \xi_{n-1} \\
    \xi_n \\
  \end{vmatrix}
  + &\dots +
  \begin{vmatrix}
    \xi_1 \\
    \xi_2 \\
    \vdots \\
    \xi_{n-1} \\
    \dot{\xi_n} \\
  \end{vmatrix}
  = \\ =
  \begin{vmatrix}
    a_{11}\xi_1 + \dots + a_{1n} \xi_n \\
    \xi_2 \\
    \vdots \\
    \xi_{n-1} \\
    \xi_n \\
  \end{vmatrix}
  + &\dots +
  \begin{vmatrix}
    \xi_1 \\
    \xi_2 \\
    \vdots \\
    \xi_{n-1} \\
    a_{n1}\xi_1 + \dots + a_{nn} \xi_n \\
  \end{vmatrix}
  = \\ = \text{\{Каждый определитель разбивается на n, в }&\text{каждом из которых ненулевой только один\}} = \\ =
  \begin{vmatrix}
    a_{11}\xi_1 \\
    \xi_2 \\
    \vdots \\
    \xi_{n-1} \\
    \xi_n \\
  \end{vmatrix}
  + &\dots +
  \begin{vmatrix}
    \xi_1 \\
    \xi_2 \\
    \vdots \\
    \xi_{n-1} \\
    a_{nn} \xi_n \\
  \end{vmatrix}
  = \\ \\ = (a_{11} + \dots + a_{nn}) &W(t) = \tr A(t) W(t).
\end{align*}

\Endproof

\Theor{Формула Лиувилля-Остроградского}
Пусть $x_1, \dots, x_n$ - решения уравнения.

Тогда $\forall t \in I \hookrightarrow W(t) = W(t_0) e^{\int_{t_0}^t \tr A(t) dt}$

\Proof

Из предыдущей леммы:

\begin{align*}
  \dot{W}(t) &= \tr A(t) \cdot W(t) \\
  \frac{dW}{W} &= \tr A(t) dt \\
  \int_{t_0}^t \frac{dW}{W} &= \int_{t_0}^t  \tr A(t) dt \\
  \ln |W(t)| - \ln |W(t_0)| &= \int_{t_0}^t  \tr A(t) dt \\
  \ln \left|\frac{W(t)}{W(t_0)}\right| &= \int_{t_0}^t  \tr A(t) dt \\
  \frac{W(t)}{W(t_0)} &= e^{\int_{t_0}^t  \tr A(t) dt}
\end{align*}

Модуль можно опустить, так как числитель и знаменатель будут одного знака

\Endproof
