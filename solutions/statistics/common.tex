\subsection{Вспомогательные утверждения}

\textbf{Виды сходимости случайных величин}

\Def \hypertarget{probability_convergence}{Последовательность $\xi_n$ сходится к $\xi$ \textit{по вероятности} ($\xi_n \xrightarrow{P} \xi$)}, если
\[
\forall \varepsilon > 0: \quad \lim_{n \to \infty} P(|\xi_n - \xi| > \varepsilon) = 0.
\]

\Def Последовательность $\xi_n$ сходится к $\xi$ \textit{почти наверное} ($\xi_n \xrightarrow{\text{п.н.}} \xi$), если
\[
P(\omega: \xi_n(\omega) \rightarrow \xi(\omega)) = 1.
\]

\Def Последовательность $\xi_n$ сходится к $\xi$ \textit{по распределению} (слабо, $\xi_n \xrightarrow{d} \xi$), если
\[
\lim_{n \to \infty} F_{\xi_n}(x) = F_\xi(x) \quad \text{во всех точках непрерывности } F_\xi.
\]

\textbf{Замечание.} Иерархия: п.н. $\Rightarrow$ по вероятности $\Rightarrow$ по распределению. Обратные импликации в общем случае неверны.

\textbf{Усиленный закон больших чисел}

\Theor{Усиленный ЗБЧ (Колмогорова)}

Пусть $X_1, X_2, \ldots$ — независимые одинаково распределённые случайные величины с $\mathbb{E}|X_1| < \infty$. Тогда
\[
\overline{X}_n = \frac{X_1 + \ldots + X_n}{n} \xrightarrow{\text{п.н.}} \mathbb{E}X_1.
\]

\textbf{Замечание.} В отличие от ЗБЧ Чебышева, здесь сходимость почти наверное (более сильная), и достаточно конечности первого момента (не второго).

\textbf{Теорема о наследовании сходимости}

\Theor{О непрерывном отображении}

Пусть $\xi_n \xrightarrow{P} \xi$ (или $\xi_n \xrightarrow{\text{п.н.}} \xi$, или $\xi_n \xrightarrow{d} \xi$), и $g$ — непрерывная функция. Тогда
\[
g(\xi_n) \xrightarrow{P} g(\xi) \quad \text{(соотв. п.н., соотв. } d\text{)}.
\]

\textbf{Замечание.} Непрерывное отображение сохраняет тип сходимости.

\textbf{Лемма Слуцкого}

\Lemma{Слуцкого}

Пусть $\xi_n \xrightarrow{d} \xi$ и $\eta_n \xrightarrow{P} c$ (константа). Тогда:
\begin{enumerate}
    \item $\xi_n + \eta_n \xrightarrow{d} \xi + c$;
    \item $\xi_n \cdot \eta_n \xrightarrow{d} c \cdot \xi$;
    \item $\frac{\xi_n}{\eta_n} \xrightarrow{d} \frac{\xi}{c}$ при $c \neq 0$.
\end{enumerate}

\textbf{Замечание.} Лемма Слуцкого позволяет заменять случайные величины, сходящиеся к константе, на саму константу при вычислении предельного распределения.

\textbf{Наследование асимптотической нормальности}

\Theor{Наследование асимптотической нормальности}

Пусть $\sqrt{n}(\hat{\theta}_n - \theta) \xrightarrow{d} \mathcal{N}(0, \sigma^2)$, и $g$ — дифференцируемая функция с $g'(\theta) \neq 0$. Тогда
\[
\sqrt{n}(g(\hat{\theta}_n) - g(\theta)) \xrightarrow{d} \mathcal{N}(0, \sigma^2 (g'(\theta))^2).
\]

\Proof (идея)

По формуле Тейлора: $g(\hat{\theta}_n) - g(\theta) \approx g'(\theta)(\hat{\theta}_n - \theta)$.

Тогда $\sqrt{n}(g(\hat{\theta}_n) - g(\theta)) \approx g'(\theta) \cdot \sqrt{n}(\hat{\theta}_n - \theta) \xrightarrow{d} g'(\theta) \cdot \mathcal{N}(0, \sigma^2) = \mathcal{N}(0, \sigma^2(g'(\theta))^2)$.

\Endproof

\textbf{Многомерный случай.} Если $\sqrt{n}(\hat{\theta}_n - \theta) \xrightarrow{d} \mathcal{N}(0, \Sigma)$ в $\mathbb{R}^k$, и $g: \mathbb{R}^k \to \mathbb{R}^m$ дифференцируема, то
\[
\sqrt{n}(g(\hat{\theta}_n) - g(\theta)) \xrightarrow{d} \mathcal{N}(0, J_g(\theta) \cdot \Sigma \cdot J_g(\theta)^T),
\]
где $J_g(\theta)$ — матрица Якоби функции $g$ в точке $\theta$.

\textbf{Многомерная ЦПТ}

\Theor{Многомерная центральная предельная теорема}

Пусть $X_1, X_2, \ldots$ — независимые одинаково распределённые случайные векторы в $\mathbb{R}^k$ с $\mathbb{E}X_1 = \mu$ и конечной ковариационной матрицей $\Sigma = \text{cov}(X_1)$. Тогда
\[
\sqrt{n}(\overline{X}_n - \mu) \xrightarrow{d} \mathcal{N}(0, \Sigma).
\]

\textbf{Состоятельность и асимптотическая нормальность}

\Def Оценка $\hat{\theta}_n$ называется \textit{состоятельной}, если $\hat{\theta}_n \xrightarrow{P_\theta} \theta$ для всех $\theta \in \Theta$.

\Def Оценка $\hat{\theta}_n$ называется \textit{сильно состоятельной}, если $\hat{\theta}_n \xrightarrow{P_\theta\text{-п.н.}} \theta$ для всех $\theta \in \Theta$.

\Def Оценка $\hat{\theta}_n$ называется \textit{асимптотически нормальной} с асимптотической дисперсией $\sigma^2(\theta)$, если
\[
\sqrt{n}(\hat{\theta}_n - \theta) \xrightarrow{d_\theta} \mathcal{N}(0, \sigma^2(\theta)).
\]

\textbf{Замечание.} Асимптотическая нормальность влечёт состоятельность (но не наоборот).

