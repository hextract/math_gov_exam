\subsection{Достаточные условия равномерной сходимости тригонометрического ряда Фурье.}

\subsubsection{Представление частичной суммы ряда Фурье с ядром Фейера}

\Def Будем говорить, что $x_n$ сходится к $a$ по Чизаро, если $\frac{x_1 + \dots + x_n}{n} \to a$, при $n \to \infty$.

Обозначим $\sigma_n(x) = \frac{\sum_{i=0}^n S_i(x)}{n+1}$ -- среднее арифметическое $n$ частичных сумм ряда Фурье.

Распишем:

\begin{gather*}
    \sigma_n(x)  = \frac{1}{n+1} \sum_{i=0}^n \frac{1}{2\pi} \int_{-\pi}^\pi f(x-t) D_n(t) \,dt = \\
    = \frac{1}{2\pi} \int_{-\pi}^\pi f(x-t) \frac{\sum_{i=0}^n D_n(t)}{n+1} \,dt = \frac{1}{2\pi} \int_{-\pi}^\pi f(x-t) F_n(t) \,dt,
\end{gather*}

где $F_n(t)$ -- ядро Фейера. Распишем его:

\begin{gather*}
    F_n(t) \coloneqq \frac{\sum_{i=0}^n D_n(t)}{n+1} = \frac{1}{(n+1)\sin \frac{t}{2}} \sum_{k=0}^n \sin \left(\left(k + \frac{1}{2}\right)t\right) = \\
    = \frac{1}{(n+1)\sin^2 \frac{t}{2}} \sum_{k=0}^n \sin\left(\frac{t}{2}\right) \sin \left(\left(k + \frac{1}{2}\right)t\right) =
    \frac{1}{2 (n+1)\sin^2 \frac{t}{2}} \sum_{k=0}^n \left(\cos kt - \cos ((k+1) t)\right) \\
    = \frac{1}{2 (n+1)\sin^2 \frac{t}{2}} \left(1 - \cos ((n+1)t)\right) = \frac{1}{n+1} \frac{\sin^2 \left(\frac{n+1}{2} t\right)}{\sin^2 \frac{t}{2}}
\end{gather*}

\textit{(Если $sin \frac{t}{2} = 0$, то $F_n(t) = n+1$)}

Свойства:

\begin{enumerate}
    \item $F_n(t) \geq 0 \; \forall t \in \R$,
    \item $F_n(t)$ -- $2\pi$-периодическая,
    \item $\int_{-\pi}^\pi F_n(t) \,dt = 2\pi$.
\end{enumerate}


\subsubsection{Теорема Фейера}

Пусть $f: \R \to \Cmp$ -- $2\pi$-периодическая; $f \in |I|([-\pi, \pi])$.

Тогда 
\begin{enumerate}
    \item Если $f$ равномерно непрерывна на $E \subset \R$, то $\sigma_n \overset{E}{\rightrightarrows} f$;
    \item Если $f$ непрерывна на $\R$, то $\sigma_n \overset{\R}{\rightrightarrows} f$;
    \item \item Если $f$ непрерывна в точке $x_0$, то $\sigma_n(x_0) \rightarrow f(x_0)$;
\end{enumerate}

\Proof

1.

Заметим свойство, что $\forall \delta > 0 \int_\delta^{+\infty} \Delta_n(x) \,dx \to 0$ при $n \to \infty$.  

Действительно, 
\begin{equation*}    
    \frac{1}{2\pi} \int_\delta^{+\infty} F_n(x) \leq \frac{1}{2\pi (n+1)} \underbrace{\int_\delta^{+\infty} \frac{dt}{\sin^2 \frac{t}{2}}}_{\text{сходится}} \to 0.
\end{equation*}

Обозначим

\begin{equation*}
    \Delta_n(x) = 
    \begin{cases}
        \frac{1}{2\pi} F_n(x), \; |x| \leq \pi, \\
        0, \; |x| > \pi.
    \end{cases}
\end{equation*}

Тогда

\begin{enumerate}
    \item 
    \begin{gather*}
        |\sigma_n(x) - f(x)| = |\int_{-\pi}^\pi f(x-t) \Delta_n(t) \,dt - f(x) \underbrace{\int_{-\pi}^\pi \Delta_n(t)}_{=1} \,dt| = \\
        = |\int_{-\pi}^\pi \left(f(x-t)-f(x)\right) \Delta_n(t) \,dt| \leq |\int_{[-\delta, \delta]} \dots \,dt| + |\int_{[-\pi, \pi] \setminus [-\delta, \delta]} \dots \,dt|
    \end{gather*}
\end{enumerate}

В силу равномерной непрерывности $f$, для любого $\eps$ мы можем выбрать такое $\delta$, что первый интеграл оценивается $\int_{[-\delta, \delta]} \left(f(x-t)-f(x)\right) \Delta_n(t) \,dt \leq \eps \int_{[-\delta, \delta]} \Delta_n(t) \,dt \leq \eps \cdot 1 $.

Во втором интеграле воспользуемся ограниченностью $f$ в силу абсолютной интегрируемости: $\exists M>0 : |f(x)| \leq M$:

\begin{equation*}
    \int_{[-\pi, \pi] \setminus [-\delta, \delta]} \left(f(x-t)-f(x)\right) \Delta_n(t) \,dt \leq 2 \cdot 2M \cdot \int_{[\delta, \pi]} \Delta_n(t)
\end{equation*}

Этот интеграл стремится к 0 по свойству $\Delta_n(x)$ выше.

Итого, $|\sigma_n(x) - f(x)|$ стремится к 0 при $n \to \infty$ вне зависимости от $x$.

Если $E \neq [-\pi, \pi]$, то $E = \cup_{k \in \Z} \left([\pi k - \pi, \pi k + \pi] \cap E\right)$ (пользуемся $2\pi$-периодичностью).

2.

$f \in C(\R) \Rightarrow f \in C([-\pi, \pi]) \Rightarrow f \text{ равн. непр. на } [-\pi, \pi] \Rightarrow \sigma_n \overset{[-\pi, \pi]}{\rightrightarrows} f \Rightarrow \sigma_n \overset{\R}{\rightrightarrows} f$.
 
3. $E = {x_0}$.

\Endproof