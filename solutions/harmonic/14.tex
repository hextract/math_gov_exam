\subsection{Достаточные условия равномерной сходимости тригонометрического ряда Фурье.}

\Note Равенство Парсеваля

\[
\int_a^b |f(x)|^2\,dx = \sum_{k=1}^{\infty} |c_k|^2,\quad c_k = \int_a^b f(x)\overline{\varphi_k(x)}\,dx.
\]

Это верно для ортонормированной системы функций $\varphi_k(x)$ в $L^2([a, b])$ (у нас именно такая).

\Theor{О скорости убывания коэффициентов Фурье}

Пусть $f \in C^{(m-1)} ([-\pi, \pi])$; $f^{(j)} (-\pi) = f^{(j)} (\pi) \;\forall j \in \{0, \dots, m-1\}$; $\exists f^{(m)}$ -- кусочно непрерывная на $[-\pi, \pi]$.

Тогда 

\begin{enumerate}
    \item $c_k (f^{(m)}) = (ik)^m c_k(f), \; k \in \Z$.
    \item $|c_k (f)| = \overline{o}\left(\frac{1}{|k|^m}\right), \; k \to +\infty$, причем $|c_k(f)| = \frac{\gamma_k}{|k|^m}$, где $sum_{k=1}^{+\infty} \gamma_k^2 < +\infty$.
\end{enumerate}

\Proof

Докажем сначала пункт 1 индукцией по $m$.

\emph{База:} Для $m=1$. По определению,
\[
c_k(f') = \frac{1}{2\pi} \int_{-\pi}^{\pi} f'(x) e^{-ikx} \, dx.
\]
Интегрируя по частям, положим $u = e^{-ikx}$, $dv = f'(x) \, dx$, тогда $du = -ik e^{-ikx} \, dx$, $v = f(x)$. Получаем
\[
c_k(f') = \frac{1}{2\pi} \left[ f(x) e^{-ikx} \Big|_{-\pi}^{\pi} + ik \int_{-\pi}^{\pi} f(x) e^{-ikx} \, dx \right].
\]
Поскольку $f(-\pi) = f(\pi)$, граничный член равен нулю, и $c_k(f') = ik c_k(f)$.

\emph{Переход:} Предположим, что для $m-1$ верно. Тогда $c_k(f^{(m)}) = ik c_k(f^{(m-1)})$, и по предположению индукции $c_k(f^{(m-1)}) = (ik)^{m-1} c_k(f)$, следовательно, $c_k(f^{(m)}) = (ik)^m c_k(f)$.

Теперь пункт 2. Из пункта 1 следует, что
\[
c_k(f) = \frac{c_k(f^{(m)})}{(ik)^m}, \quad |c_k(f)| = \frac{|c_k(f^{(m)})|}{|k|^m}.
\]
Поскольку $f^{(m)}$ --- кусочно непрерывная на компактном отрезке, она ограничена и, следовательно, интегрируема. По лемме Римана--Лебега $c_k(f^{(m)}) \to 0$ при $k \to \pm \infty$, поэтому $|c_k(f)| = o\left(\frac{1}{|k|^m}\right)$.

Далее, поскольку $f^{(m)}$ кусочно непрерывна, она принадлежит $L^2([-\pi, \pi])$ (измерима и квадрат интегрируем). Можем записать равенство Парсеваля
\[
\sum_{k=-\infty}^{+\infty} |c_k(f^{(m)})|^2 = \frac{1}{2\pi} \int_{-\pi}^{\pi} |f^{(m)}(x)|^2 \, dx < +\infty,
\]
поскольку интеграл конечен. Положим $\gamma_k = |c_k(f^{(m)})|$ для $k \geq 1$ (и симметрично для отрицательных $k$, но сумма по положительным $k$ также сходится, так как ряд симметричен). Тогда $|c_k(f)| = \frac{\gamma_k}{|k|^m}$ и $\sum_{k=1}^{+\infty} \gamma_k^2 < +\infty$.

\Endproof

\Theor{О равномерной сходимости ряда Фурье}

Пусть 

\begin{enumerate}
    \item $f \in C^{(m-1)} [-\pi, \pi]$ ($m \geq 1$);
    \item $\exists f^{(m)}$ -- кусочно непрерывная на $[-\pi, \pi]$;
    \item $f^{(j)}(-\pi) = f^{(j)}(\pi) \;\forall j \in \{0, \dots, m-1\}$.
\end{enumerate}

Тогда ряд Фурье функции $f$ сходится к $f$ равномерно на $[-\pi, \pi]$, причем 

\begin{equation*}
    |f(x) - S_n(x)| \leq \frac{\delta_n}{n^{m-\frac{1}{2}}},
\end{equation*}

где $\delta_n \to 0$.

\Proof

По теореме о скорости сходимости коэффициентов, $|c_k(f)| = \frac{\gamma_k}{|k|^m}$, где $\sum \gamma_k^2 < \infty$ $\;\forall k \in \Z \setminus \{0\}$.

Применим неравенство о среднем:

\begin{equation*}
    \frac{\gamma_k}{|k|^m} \leq \frac{1}{2} \left(\gamma_k^2 + \frac{1}{|k|^{2m}}\right)
\end{equation*}

Ряд $\sum \frac{1}{|k|^{2m}}$ сходится, так как $2m > 1$. 

Значит по признаку Вейерштрасса ряд $\sum_{k=1}^{\infty} \frac{\gamma_k}{|k|^m}$ сходится.

Далее, 

\begin{equation*}
    |S_n(x)| \leq \sum_{k=-n}^n |c_k| = \sum_{k=-n}^n \frac{\gamma_k}{|k|^m},
\end{equation*}

откуда $\{S_n(x)\}$ сходится равномерно на $[-\pi, \pi]$ в силу признака Вейерштрасса для функциональных последовательностей.

Так как $f$ удовлетворяет условию Дини на $[-\pi, \pi]$ (в силу непрерывности), а также $f(-\pi)=f(\pi)$, значит ее можно продолжить на $\R$, так чтобы условие Дини было выполнено во всех точках.

По теореме о поточечной сходимости, если в точке выполнено условие Дини, то ряд сходится к полусумме односторонних пределов. Но $f$ непрерывна, а значит $S_n$ сходится равномерно к $f$.

Оценим теперь $|f(x) - S_n(x)|$:

\begin{gather*}
    |f(x) - S_n(x)| = |S(x) - S_n(x)| = \left|\sum_{|k| \geq n+1} c_k e^{ikx}\right| \leq \\
    \leq \sum_{|k| \geq n+1} |c_k| = \sum_{|k| \geq n+1} \frac{\gamma_k}{|k|^m} \leq \left(\sum_{|k| \geq n+1} \gamma_k^2\right)^{\frac{1}{2}} \left(\sum_{|k| \geq n+1} \frac{1}{|k|^{2m}}\right)^{\frac{1}{2}}
\end{gather*}

Последняя оценка по неравенству КБШ в $L_2$. 

Первая скобка стремится к 0, так как ряд $\sum_{k=1}^{+\infty} \gamma_k^2$ сходится.

Вторая скобка оценить через интегральный признак:

\begin{gather*}
    (\sum_{|k| \geq n+1}^{+\infty} \frac{1}{k^{2m}})^{\frac{1}{2}} \leq (\int_n^{+\infty} \frac{1}{x^{2m}} dx)^{\frac{1}{2}} = (\frac{1}{2m-1} \frac{1}{n^{2m-1}})^{\frac{1}{2}} = \sqrt{\frac{1}{2m-1}} \frac{1}{n^{m-\frac{1}{2}}}.
\end{gather*}

В качестве $\delta_n$ берем первую скобку умноженную на $\sqrt{\frac{1}{2m-1}}$ и получаем искомое.

\Endproof