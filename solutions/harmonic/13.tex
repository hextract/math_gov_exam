\subsection{Достаточные условия сходимости тригонометрического ряда Фурье в точке.}

\subsubsection{Тригонометрический ряд Фурье}

\Lemma Пусть $f(x) = \frac{a_0}{2} + \sum_{k=1} \left(a_k \cos kx + b_k \sin kx\right)$ и этот ряд сходится равномерно на $\R$.

Тогда
\begin{align*}
    a_0 &= \frac{1}{\pi} \int_{-\pi}^{\pi} f(x) \;dx, \\
    a_k &= \frac{1}{\pi} \int_{-\pi}^{\pi} f(x) \cos kx \;dx, \\
    b_k &= \frac{1}{\pi} \int_{-\pi}^{\pi} f(x) \sin kx \;dx
\end{align*}

\Proof

$f$, как предел равномерно сходящегося ряда из непрерывных функций, является непрерывной на $[-\pi, \pi]$.

Домножим определение $f$ на $\cos nx$ и проинтегрируем:

\begin{equation*}
    \int_{-\pi}^{\pi} f(x) \cos nx dx = \int_{-\pi}^{\pi} \left(\frac{a_0}{2} + \sum_{k=1} \left(a_k \cos kx + b_k \sin kx\right)\right) \cos nx \; dx
\end{equation*}

Справа имеем интеграл суммы равномерно сходящегося ряда, а значит можно переписать как сумму интегралов.

Также вспомним, что $\int_{-\pi}^{\pi} \sin nx \cos mx \;dx = 0$ и $\int_{-\pi}^{\pi} \cos nx \cos mx \;dx = 0$ для $\forall n, m \in \N: n \neq m$.
Это значит, что в сумме справа обнулятся все члены, кроме как для $k=n$.

Тогда получим:

\begin{equation*}
    \int_{-\pi}^{\pi} f(x) \cos nx \;dx = \int_{-\pi}^{\pi} a_n \cos^2 nx \; dx =  a_n \int_{-\pi}^{\pi} \cos^2 nx \; dx = \pi a_n
\end{equation*}

То есть действительно $a_k = \frac{1}{\pi} \int_{-\pi}^{\pi} f(x) \cos kx dx$.

Аналогично расписывается для $a_0$ и $b_k$.

\Endproof

\Def $f(x)$ называется абсолютно интегрируемой (будем обозначать $f \in |I|$), если TODO.

\Def Тригонометрическим рядом Фурье функции $f \in |I|$ будем называть ряд $S(x) = \frac{a_0}{2} + \sum_{k=1} \left(a_k \cos kx + b_k \sin kx\right)$ с коэффициентами, определенными выше.

Частичную сумму ряда будем обозначать $S_n(x) = \frac{a_0}{2} + \sum_{k=1}^n \left(a_k \cos kx + b_k \sin kx\right)$.

\Note Тригонометрический ряд также может быть записан в комплексном виде:

\begin{equation*}
    f \sim \sum_{k \in \Z} c_k e^{ikx}
\end{equation*}, 

где коэффициенты $c_k = \frac{1}{2\pi} \int_{-\pi}^{\pi} f(x) e^{-ikx} \; dx$.

Аналогично предыдущей теореме можно показать, что этот ряд сходится равномерно к функции $f$ только при таких коэффициентах $c_k$.

\subsubsection{Ядро Дирихле. Интегральное представление частичной суммы ряда Фурье с ядром Дирихле}

Введем ядро Дирихле:

\begin{align*}
    D_n(\xi) \coloneqq \sum_{k=-n}^n e^{ik\xi} = [\text{пусть } e^{i\xi} \neq 1] = \frac{e^{i(n+1)\xi} - e^{-in\xi}}{e^{i\xi} - 1} = \frac{e^{i(n+\frac{1}{2})\xi} - e^{-i(n+\frac{1}{2})\xi}}{e^{\frac{i\xi}{2}} - e^{-\frac{i\xi}{2}}} = \frac{\sin((n+1)\xi)}{\sin\left(\frac{\xi}{2}\right)}
\end{align*}

Если же $e^{i\xi} = 1$, то тривиально $D_n(\xi) = 2n+1$.

Свойства ядра Дирихле:
\begin{enumerate}
    \item $D_n(-\xi) = D_n(\xi)$
    \item $\frac{1}{2\pi} \int_{-\pi}^{\pi} D_n(\xi) d\xi = 1$
\end{enumerate}

Теперь распишем частичную сумму ряда Фурье $2\pi$-периодичной функции $f$ через ядро Дирихле:

\begin{gather*}
    S_n(x) = \sum_{k=-n}^{n} c_k e^{ikx} =\sum_{k=-n}^{n} \left(\left(\frac{1}{2\pi} \int_{-\pi}^\pi f(t) e^{-ikt}\;dt\right) e^{ikx}\right) = \\
    \frac{1}{2\pi} \int_{-\pi}^\pi f(t) \sum_{k=-n}^n e^{ik(x-t)}\;dt = \frac{1}{2\pi} \int_{-\pi}^\pi f(t) D_n(x-t) dt = \\
    \text{Замена: }\begin{cases}\tau = x-t, \\ t = x-\tau \end{cases} \\
    = - \frac{1}{2\pi} \int_{x+\pi}^{x-\pi} f(x-\tau) D_n(\tau) d\tau = \frac{1}{2\pi} \int_{-\pi}^\pi f(x-\tau) D_n(\tau) d\tau = \\
    = \frac{1}{2\pi} \int_{-\pi}^0 f(x-\tau) D_n(\tau) \;d\tau + \frac{1}{2\pi} \int_0^{\pi} f(x-\tau) D_n(\tau) \;d\tau = \\
    = \frac{1}{2\pi} \int_0^\pi \left(f(x-\tau) + f(x+\tau)\right) D_n(\tau) \;d\tau
\end{gather*}

\subsubsection{Лемма Римана об осциляции}

Пусть $f \in |I|((a, b))$.

Тогда $\lim\limits_{k \to \infty} \int_a^b f(x) \sin kx \; dx = \lim\limits_{k \to \infty} \int_a^b f(x) \cos kx \;dx = 0$.

\Proof

Рассмотрим $(a, b)$ -- конечный интервал; $\{x_i\}$ -- разбиение Дарбу $(a, b)$, т. ч. $\sum (M_i - m_i) \delta x_i \leq \frac{\eps}{2}$.

\begin{gather*}
    \left|\int_a^b f(x) \cos kx \;dx\right| = \left|\sum_{i=1}^N \int_{x_{i-1}}^{x_i} (f(x) - m_i) \cos kx \;dx + \sum_{i=1}^N m_i \int_{x_{i-1}}^{x_i} \cos kx \;dx\right| \leq \\
    \leq \sum_{i=1}^N \int_{x_{i-1}}^{x_i} |f(x) - m_i| |\cos kx| \;dx + \sum_{i=1}^N |m_i| |\sin kx_i - \sin kx_{i-1}| \leq \\
    \leq \sum_{i=1}^N (M_i - m_i) \delta x_i + \frac{2NM}{k},
\end{gather*}

где $M = \sup\limits_{[a, b]} |f(x)|$.

Итого $\forall \eps > 0 \; \exists N, \, \{x_i\}_{i=1}^N, \, k_0 \,:\, \forall k > k_0 \; \left|\int_a^b f(x) \cos kx \;dx\right| \leq \eps$, то есть $\int_a^b \dots dx \to 0$ при $k \to \infty$.

В общем случае, где $f$ имеет конечное количество особенностей, разбиваем интеграл, приближаясь к ним:

\begin{equation*}
    \left| \int_a^{\text{\textcircled{b}}} f(x) \cos kx \;dx \right| = \left|\int_a^{b'} f(x) \cos kx \;dx\right| + \left|\int_{b'}^{\text{\textcircled{b}}} f(x) \cos kx \;dx\right| < \frac{\eps}{2} + \frac{\eps}{2} = \eps
\end{equation*}

($\forall \eps > 0 \; \exists b' \in (a, b): \int_{b'}^b f(x) \dots dx < \eps$).

\Endproof

\Consequence Коэффициенты Фурье абсолютно интегрируемой функции $f$ стремятся к 0 при $n \to \infty$.

\subsubsection{Принцип локализации}

TODO, возможно не нужен

\subsubsection{Теорема Дини о достаточном условии поточечной сходимости ряда Фурье в точке}

Обозначения:

\begin{align*}
    f(x_0 - 0) &= \lim_{h \to +0} f(x_0 - h) \\
    f(x_0 - 0) &= \lim_{h \to +0} f(x_0 + h) \\
    f'_{-}(x_0) &= \lim_{h \to +0} \frac{f(x_0 - h) - f(x_0)}{h} \\
    f'_{+}(x_0) &= \lim_{h \to +0} \frac{f(x_0 + h) - f(x_0)}{h}
\end{align*}

\Def $f$ удовлетворяет условию Дини в точке $x_0 \in \R$, если

\begin{enumerate}
    \item $\exists f(x_0-0), f(x_0+0)$,
    \item $\exists \delta > 0 : \int_0^\delta \frac{(f(x_0-t)-f(x_0-0)) + (f(x_0+t)-f(x_0+0))}{t} \; dt$ сходится абсолютно.
\end{enumerate}

\Def $f$ кусочно непрерывна на $[a, b]$, если существует разбиение ${x_i}_{i=1}^n$:
\begin{enumerate}
    \item $f \in C((x_{i-1}, x_i)), i = 1, \dots, n$,
    \item $\exists f(x_k-0), f(x_k+0), k = 1, \dots, n-1$.
\end{enumerate}

\Def $f$ кусочно непрерывна на $[a, b]$, если она непрерывна, а ее производная кусочно непрерывна на $[a, b]$.

\Theor{Дини}

Пусть $f \in |I|([-\pi, \pi])$; $2\pi$-периодическая; и удовлетворяет условию Дини в точке $x_0$.

Тогда $S_n(x_0) \to \frac{f(x_0-0) + f(x_0+0)}{2}$ при $n \to \infty$.

\Proof

\begin{gather*}
    \left|S_n(x_0) - \frac{f(x_0-0) + f(x_0+0)}{2}\right| = \\[20pt]
    = \left|\frac{1}{2\pi} \int_0^\pi \frac{f(x_0-t) + f(x_0+t)}{\sin \frac{t}{2}} 
    \sin \left((n+\frac{1}{2})t\right)\,dt 
    - \frac{1}{2\pi} \int_0^\pi 
    \frac{f(x_0-0) + f(x_0+0)}{2} \cdot 
    \frac{1}{2\pi} \int_0^\pi D_n(t)\,dt\right| = \\[20pt]
    = \left|\frac{1}{2\pi} \int_0^\pi 
        \underbrace{
          \underbrace{
            \frac{(f(x_0-t)-f(x_0-0)) + (f(x_0+t)-f(x_0+0))}{t}
          }_{\text{абсолютно сходится, т.\,к. удовлетворяет условию Дини}}
          \cdot
          \underbrace{
            \frac{t}{2\sin\frac{t}{2}}
          }_{\text{непрерывно}}
        }_{\text{абсолютно сходится, как произведение абсолютно сходящегося на непрерывную}}
        \cdot \sin\left((n+\tfrac{1}{2})t\right)\,dt\right|
        \xrightarrow[\;n \to \infty\;]{\text{th. об осциляции}} 0
\end{gather*}

\Endproof