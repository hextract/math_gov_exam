\subsection{Достаточные условия сходимости тригонометрического ряда Фурье в точке.}

\subsubsection{Тригонометрический ряд Фурье}

\Lemma Пусть $f(x) = \frac{a_0}{2} + \sum_{k=1} \left(a_k \cos kx + b_k \sin kx\right)$ и этот ряд сходится равномерно на $\R$.

Тогда
\begin{align*}
    a_0 &= \frac{1}{\pi} \int_{-\pi}^{\pi} f(x) fx, \\
    a_k &= \frac{1}{\pi} \int_{-\pi}^{\pi} f(x) \cos kx \;dx, \\
    b_k &= \frac{1}{\pi} \int_{-\pi}^{\pi} f(x) \sin kx \;dx
\end{align*}

\Proof

$f$, как предел равномерно сходящегося ряда из непрерывных функций, является непрерывной на $[-\pi, \pi]$.

Домножим определение $f$ на $\cos nx$ и проинтегрируем:

\begin{equation*}
    \int_{-\pi}^{\pi} f(x) \cos nx dx = \int_{-\pi}^{\pi} \left(\frac{a_0}{2} + \sum_{k=1} \left(a_k \cos kx + b_k \sin kx\right)\right) \cos nx \; dx
\end{equation*}

Справа имеем интеграл суммы равномерно сходящегося ряда, а значит можно переписать как сумму интегралов.

Также вспомним, что $\int_{-\pi}^{\pi} \sin nx \cos mx \;dx = 0$ и $\int_{-\pi}^{\pi} \cos nx \cos mx \;dx = 0$ для $\forall n, m \in \N: n \neq m$.
Это значит, что в сумме справа обнулятся все члены, кроме как для $k=n$.

Тогда получим:

\begin{equation*}
    \int_{-\pi}^{\pi} f(x) \cos nx \;dx = \int_{-\pi}^{\pi} a_n \cos^2 nx \; dx =  a_n \int_{-\pi}^{\pi} \cos^2 nx \; dx = \pi a_n
\end{equation*}

То есть действительно $a_k = \frac{1}{\pi} \int_{-\pi}^{\pi} f(x) \cos kx dx$.

Аналогично расписывается для $a_0$ и $b_k$.

\Endproof

\Def $f(x)$ называется абсолютно интегрируемой (будем обозначать $f \in |I|$), если TODO.

\Def Тригонометрическим рядом Фурье функции $f \in |I|$ будем называть ряд $S(x) = \frac{a_0}{2} + \sum_{k=1} \left(a_k \cos kx + b_k \sin kx\right)$ с коэффициентами, определенными выше.

Частичную сумму ряда будем обозначать $S_n(x) = \frac{a_0}{2} + \sum_{k=1}^n \left(a_k \cos kx + b_k \sin kx\right)$.

\Note Тригонометрический ряд также может быть записан в комплексном виде:

\begin{equation*}
    f \sim \sum_{k \in \Z} c_k e^{ikx}
\end{equation*}, 

где коэффициенты $c_k = \frac{1}{2\pi} \int_{-\pi}^{\pi} f(x) e^{-ikx} \; dx$.

Аналогично предыдущей теореме можно показать, что этот ряд сходится равномерно к функции $f$ только при таких коэффициентах $c_k$.

\subsubsection{Ядро Дирихле. Интегральное представление частичной суммы ряда Фурье с ядром Дирихле}

\subsubsection{Лемма Римана об осциляции}

\subsubsection{Теорема Дини о достаточном условии поточечной сходимости ряда Фурье в точке}