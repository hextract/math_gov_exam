\subsection{Полная система. Классы Поста. Критерий Поста}

\Def Булевыми функциями $k$ аргументов или множеством $P_2^k$ будем называть функции вида $f: \{0, 1\}^k \to \{0, 1\}$. $P_2 := \bigcup\limits_{k=0}^{+ \infty} P_2^k$.

\Def Литерал --- переменная или ее отрицание.

Булева функция может задаваться:

\begin{enumerate}
  \item Таблицей истинности.
  \item Вектором значений (длины $2^k$) на каждом наборе аргументов.
  \item Формулой.
\end{enumerate}

\Def Формула --- это

\begin{enumerate}
  \item Литерал, константа (0 или 1).
  \item Если $f$ --- формула, то и $\overline{f}$ --- тоже (отрицание).
  \item Если $f, g$ --- формулы, то $f \circ g$ --- тоже, где $\circ$ --- функция двух переменных.
\end{enumerate}

\Def Суперпозицией (композицией) называется сложная функция, полученная из исходных подстановкой одной функции в другую и отождествлением переменных.

\Def Замыканием системы функций называют все такие функции, которые можно получить из нее с помощью композиции.

\Def Систему функций называют замкнутой, если она совпадает со своим замыканием.

\Def Систему функций называют полной, если ее замыкание --- $P_2$.

\textbf{Классы Поста}

\Def Пусть $\alpha, \beta \in \{0, 1\}^k$. Введем отношение нестрогого частичного порядка: $\alpha \preccurlyeq \beta \iff \alpha_i \le \beta_i \quad \forall i$.

\begin{enumerate}
  \item $T_0 := \{ f \mid f(0, 0, \dots 0) = 0 \}$.
  \item $T_1 := \{ f \mid f(1, 1, \dots 1) = 1 \}$.
  \item $L := \{ f \mid f = a_0 + a_1x_1 + \dots a_nx_n \}$. Где ``$+$'' --- \texttt{xor} ($\oplus$). ``$\cdot$'' --- конъюнкция ($\land$).
  \item $S := \{ f \mid \overline{f}(\overline{x_1}, \dots \overline{x_n}) = f(x_1, x_2, \dots x_n) \}$.
  \item $M := \{ f \mid \alpha \preccurlyeq \beta \implies f(\alpha) \le f(\beta) \}$.
\end{enumerate}

\Statement Классы Поста замкнуты.

\Th Критерий Поста.

Система функций полна $\iff$ она не содержится полностью ни в одном классе Поста.

$\square$
\begin{itemize}
  \item Если система функций полностью содержится в одном из классов Поста, то ее замыкание полностью лежит в нем, при этом ни один из классов не совпадает с $P_2$.
  \item Пусть система функций не содержится полностью ни в одном из классов Поста. Докажем, что она полна.

    Рассмотрим функции:

    \begin{enumerate}
      \item $f_0 \notin T_0$
      \item $f_1 \notin T_1$
      \item $f_M \notin M$
      \item $f_L \notin L$
      \item $f_S \notin S$
    \end{enumerate}

    Соберем $\varphi_0(x) := f_0(x, x, \dots x)$. Тогда $\varphi_0(0) = 1$. Значит получили $\neg$ или $1$. Аналогично из $\varphi_1(x) := f_1(x, x, \dots x)$ получаем
    $\neg$ или $0$.

    Рассмотрим 3 случая:

    \begin{enumerate}
      \item Получили $0, 1, \neg$ --- переходим к следующему шагу.
      \item Получили только $0, 1$.

        Возьмем $f_M$ и $\alpha, \beta : \alpha \preccurlyeq \beta, f_M(\alpha) = 1, f_M(\beta) = 0$, которые найдутся по определению.

        Пусть $\gamma(x) : \gamma_i(x) = \alpha_i,$ если $\alpha_i = \beta_i$ и $x$ иначе.

        \begin{figure}[H]
          \centering
          \includegraphics[width=0.25\linewidth]{32_1.png}
          \caption{Пример $\gamma$}
        \end{figure}

        Положим $\varphi_M(x) := f_M(\gamma(x))$. Тогда $\varphi_M(0) = 1, \varphi_M(1) = 0$. Получили и $\neg$.

      \item Получили только $\neg$.

        Возьмем $f_S$ и $\alpha_1, \alpha_2 \dots \alpha_n : f_S(\alpha_1, \alpha_2 \dots \alpha_n) = f_S(\overline{ \alpha_1 }, \overline{\alpha_2} \dots \overline{\alpha_n})$

        Поставим на места 1 --- $x$, а на места 0 --- $\overline{x}$. Соберем $\varphi_S(x)$ вида $f_S(x, \overline{x}, x, \dots)$. Тогда $\varphi_S(x) =
        \varphi_S(\overline{x})$. Значит $\varphi_S \equiv \text{const}$. С помощью отрицания получим и вторую константу.

        Теперь есть $0, 1, \neg$.

        Рассмотрим полином Жегалкина для $f_L$ и наименьшую степень в нем, большую 1 (с ненулевым коэффициентом).

        $f_L(x) = a_0 + \dots a_i x_{i1} x_{i2} \dots x_{ik} \dots$. Отождествим $x_{i1} = y, x_{i2} = z$, $x_{i2..k} = 1$. Остальные $x_{jk} = 0$.

        Тогда получим $\varphi_L(y, z)$, равную полиному от $y, z$.

        То есть вида $1^{\sigma_1} + y^{\sigma_2}z^{\sigma_3} + \alpha_1 y^{\sigma_4} + \alpha_2 z^{\sigma_5}$. Уберем отрицания над переменными и получим $\alpha_0 + yz
        + \alpha_1 y + \alpha_2 z$. Если есть $\alpha_0 = 1$, то возьмем отрицание от всей формулы и избавимся от него.

        Получили формулу вида $yz + \alpha_1 y + \alpha_2 z$.

        Рассмотрим случаи:

        \begin{enumerate}
          \item $\alpha_1 = \alpha_2 = 0$. Тогда получили $y \land z$.
          \item $\alpha_1 = \alpha_2 = 1$. Формула --- $yz + y + z$. Ее отрицание даст $yz + y + z + 1 = (y + 1)(z + 1) = (\neg y) \land (\neg z)$. Уберем отриациня у
            $y, z$ и получим $y \land z$.
          \item $\alpha_1 \ne \alpha_2$. Пусть $\alpha_1 = 1, \alpha_2 = 0$, формула имеет вид $yz + y = (z + 1) y = (\neg z) y$. Уберем отриацине у $z$ и получим $y \land z$.
        \end{enumerate}

        Система $0, 1, \neg, \land$ полна (можно составить КНФ или ДНФ, получив \texttt{or} из законов де Моргана + есть константы).
    \end{enumerate}
\end{itemize}
$\blacksquare$
