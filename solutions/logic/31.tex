\subsection{Сравнения по модулю. Малая теорема Ферма. Теорема Эйлера.}

\Note В теории чисел рассматриваем элементы из $\mathbb{Z}$.

\textbf{Основные определения}

\begin{enumerate}
  \item $a \mathrel{\vdots} b, b \ne 0$, если $\exists c : a = b \cdot c$, т.е $b$ является делителем $a$.
  \item $a \mid d \iff \exists k : d = a \cdot m$, т.е $a$ является делителем $d$.
  \item $a$ делится на $b \ne 0$ с остатком $r$, где $0 \le r < |b|$, если $\exists q: a = bq + r$.
  \item Число $p$ называется \textbf{простым}, если не имеет натуральных делителей кроме $1$ и $p$. В противном случае называется \textbf{составным}. Единицу принято
    считать не простым и не составным.
  \item $(a, b) := d \in \mathbb{N}$, если $a = a'd$, $b = b'd$ и $d$ --- наибольший (является делителем любого другого).
  \item Если $(a, b) = 1$, то такие числа называют взаимнопростыми.
  \item (Лемма Евклида.) Пусть $(a, b) = 1$ и $a \mid (bc)$. Тогда $a \mid c$.
  \item $[a, b]$ --- наименьшее $m \in \mathbb{N}: m \mathrel{\vdots} a, m \mathrel{\vdots} b$.
    $(a, b) \cdot [a, b] = a \cdot b$.
  \item \textbf{Основная теорема арифметики.} Любое $m \in \mathbb{N}$ единственным образом представимо в виде произведения простых делителей

    $m = p_1^{k_1} p_2^{k_2} \dots p_n^{k_n}, k_i \in \mathbb{N}, p_i$ --- простые.
  \item $a \underset{m}{\equiv} b \iff a -b \mathrel{\vdots} m$
\end{enumerate}

\Statement

Пусть $a \underset{p}{\equiv} b, c \underset{p}{\equiv} d$.

Тогда
\begin{enumerate}
  \item $a + c \underset{p}{\equiv} b + d$
  \item $a \cdot c \underset{p}{\equiv} b \cdot d$
\end{enumerate}

\Statement

Пусть $a \cdot c \underset{m}{\equiv} b \cdot c, (c, m) = 1$. Тогда $a \underset{m}{\equiv} b$

$\square$
$ac \underset{m}{\equiv} bc \implies ac - bc \mathrel{\vdots} m \implies (a - b)c \mathrel{\vdots} m \implies a -b \mathrel{\vdots} m \implies a \underset{m}{\equiv} b$
$\blacksquare$

\textbf{Малая теорема Ферма.}

$a^{p-1} \underset{p}{\equiv} 1$, если $p$ --- простое, а $a$ не делится на $p$.

$\square$

Покажем, что $\underset{p}{\equiv}$ действительно является отношением эквивалентности:

\begin{enumerate}
  \item $a \underset{p}{\equiv} a$, т. к. $a - a = 0 \mathrel{\vdots} p$
  \item $a \underset{p}{\equiv} b \implies b \underset{p}{\equiv} a$, т. к. $(a - b) \mathrel{\vdots} p \implies (b -a) = -(a -b) \mathrel{\vdots} p$.
  \item $a \underset{p}{\equiv} b, b \underset{p}{\equiv} c \implies (a -b) \mathrel{\vdots} p, (b - c) \mathrel{\vdots} p \implies (a - b) + (b - c) = (a -c) \mathrel{\vdots} p$
\end{enumerate}

Значит разобьём все неотрицательные числа на классы эквивалентности по этому отношению, обозначая каждый наименьшим его представителем: $[0], [1], \dots [p-1]$.

Рассмотрим набор всех ненулевых остатков $1, 2, 3, \dots p -1$.

Каждое число домножим на $a$: $a \cdot 1, a \cdot 2, \dots a \cdot (p - 1)$

Пусть среди них какие-то 2 дают одинаковый остаток при делении на $p$.

Тогда $ai \underset{p}{\equiv} aj \implies a \cdot (i -j) \underset{p}{\equiv} 0 \implies i \underset{p}{\equiv} j$. Значит все остатки различны и образуют все ненулевые
остатки при делении на $p$.

Следовательно $a^{p - 1} \cdot 1 \cdot 2 \dots (p - 1) \underset{p}{\equiv} 1 \cdot 2 \dots (p - 1)$ (т.к содержат одни и те же остатки, но возможно в разном порядке).
Но $1 \cdot 2 \dots (p - 1)$ взаимнопросто с $p$. Значит $1
\underset{p}{\equiv} a^{p -1}$.

$\blacksquare$

\Def

\textit{Функцией Эйлера} $\varphi(n)$ называется количество натуральных чисел, не превосходящих $n$ и взаимнопростых с ним.

\Def

Функция $f: \mathbb{N} \to \mathbb{N}$ называется \textit{мультипликативной}, если  $\forall a, b: (a, b) = 1 \quad f(a \cdot b) = f(a) \cdot f(b)$.

\Lemma $(x, mn) = 1 \iff (x, m) = 1, (x, n) = 1$

\Statement

Функция Эйлера --- мультипликативная.

$\square$

\textbf{Вариант доказательства 1}

Докажем $(m, n) = 1 \implies \varphi(mn) = \varphi(m) \varphi(n)$

Т. к. $(m, n) = 1$, значит $\exists x, y \in \mathbb{Z}: mx + ny = 1$ (следует из корректности алгоритма Евклида).

Тогда $\forall a \in \mathbb{Z} \quad \exists x_a = xa, y_a = ya \in \mathbb{Z}: m x_a + n y_a = a$.

Рассмотрим выражение $m x + n y$, где $x$ и $y$ пробегают все возможные остатки $0, 1 \dots n-1$ и $0, 1 \dots m-1$ соответственно.

Пусть значения выражений совпали при различных $x$ и $y$. Но тогда $m x_1 + n y_1 \underset{mn}{\equiv} m x_2 + n y_2 \implies m (x_1 - x_2) + n (y_1 - y_2)
\underset{mn}{\equiv} 0 \implies x_1 \underset{n}{\equiv} x_2, y_1 \underset{m}{\equiv} y_2$.

А значит таким образом получаем в этом выражении все возможные остатки при делении на $mn$.

Теперь рассмотрим произвольное $t \in \{1, 2, \dots mn \}$. Тогда можно представить $t \underset{mn}{\equiv} mx + ny$.

Если $(m, t) = 1, (n, t) = 1 \implies (mn, t) = 1$.

Если $(mn,  t) \ne 1$, то $(t, m) \ne 1$ или $(t, n) \ne 1$.

Кроме того, из $t \underset{mn}{\equiv} mx + ny$ следует $t \underset{n}{\equiv} mx$ и $t \underset{m}{\equiv} ny$. Так как $(m,n)=1$, имеем $(t,n)=1 \iff (x,n)=1$ и
$(t,m)=1 \iff (y,m)=1$. Следовательно
\[
  (t,mn)=1 \iff (t,m)=1 \ \text{и}\ (t,n)=1 \iff (x,n)=1 \ \text{и}\ (y,m)=1,
\]
и число таких пар равно $\varphi(n)\varphi(m)$.

\textbf{Вариант доказательства 2}

Рассмотрим отображение: $f : \Z_{mn} \to \Z_m \times \Z_n, \quad f(x) = (x \ mod \ m, x \ mod \ n)$

По Китайской теореме об остатках, $f$ - взаимно однозначное отображение (биекция)

Сузим $f$ на множество $A = \{ x \in \{1, 2, ... mn\} | (x, mn) = 1 \}$

По лемме выше $(a, b) \in f(A) \iff (a, m) = 1, (a, n) = 1$. Количество таких пар - ровно $\varphi(m) \cdot \varphi(n)$.

Но $f$ - биекция. Значит $\varphi(mn) = |A| = |f(A)| = \varphi(m) \cdot \varphi(n)$

$\blacksquare$

\Theor{Эйлера}

Пусть $(a, n) = 1$. Тогда $a^{\varphi(n)} \underset{n}{\equiv} 1$.

$\square$

Рассмотрим все натуральные числа, меньшие $n$ и взаимнопростые с ним:

$r_1, r_2, \dots r_{\varphi(n)}$

Домножим каждое на $a$ и рассмотрим остаток при делении на $n$. Т. к. $(a, n) = 1$, то все числа останутся взаимно простыми с $n$, при этом никакие не совпадают
(показывается аналогично малой т. Ферма). Тогда
$(a r_1) (a r_2) \dots (ar_{\varphi(n)}) \underset{n}{\equiv} r_1 r_2 \dots r_{\varphi(n)} \implies a^{\varphi(n)} \underset{n}{\equiv} 1$.

$\blacksquare$
