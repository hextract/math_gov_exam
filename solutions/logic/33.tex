\subsection{Разрешимые и перечислимые языки. Вычислимые функции. Примеры алгоритмически неразрешимых задач.}

\Def \textit{Машиной Тьюринга} называется кортеж $(\Sigma, Q, F, q_0, \delta, \Lambda)$, где:

\begin{enumerate}
    \item $\Sigma$ --- конечный алфавит.
    \item $Q$ --- конечное множество состояний.
    \item $\Sigma \cap Q = \varnothing$.
    \item $F \subset Q, q_0 \in Q$ --- финальные и начальное состояние.
    \item $\Lambda \in \Sigma$ --- пробельный символ
    \item $\delta : \Sigma \times Q \to \Sigma \times \{-1, 0, 1\} \times Q$ --- правила перехода.
\end{enumerate}

На вход в МТ подается слово, каретка установлена на его начало. Лента в обе стороны бесконечна и заполнена пробельными символами. На каждом шаге (такте) рассматривается текущее состояние и символ на ленте под кареткой. По $\delta$ определяется новая буква на этой позиции, сдвиг каретки и смена состояния. При попадании в одно из финальных (терминальных) состояний МТ останавливается.

\Def МТ $\mathcal{M}$ \textit{разрешает} язык $\mathcal{L}$, если 
\begin{enumerate}
    \item $\forall x \in \mathcal{L}$ $\mathcal{M}$ останавливается в $q_{yes}$
    \item $\forall x \notin \mathcal{L}$ $\mathcal{M}$ останавливается в $q_{no}$
\end{enumerate}

Такой язык называется \textit{разрешимым}.

\Def МТ $\mathcal{M}$ \textit{допускает} язык $\mathcal{L}$, если 
\begin{enumerate}
    \item $\forall x \in \mathcal{L}$ $\mathcal{M}$ останавливается в $q_{yes}$
    \item $\forall x \notin \mathcal{L}$ $\mathcal{M}$ не останавливается в $q_{yes}$ (может остановиться в $q_{no}$ или работать бесконечно)
\end{enumerate}

Такой язык называется \textit{допустимым}.

\Def МТ $\mathcal{M}$ \textit{перечисляет} язык $\mathcal{L}$, если $\mathcal{M}$, работая с пустого входа, печатает на ленте все слова $\mathcal{L}$ (через разделяющий символ) таким образом, что любое слово из $\mathcal{L}$ появится на ленте за \underline{конечное} время. Слово не из $\mathcal{L}$ на ленте не появится никогда.

Такой язык называется \textit{перечислимым}.

\Th

$\mathcal{L}$ перечислим $\iff$ $\mathcal{L}$ допустим

$\square$
\begin{itemize}
    \item $\Rightarrow$
    
    Возьмем $\mathcal{M}$, перечисляющую $\mathcal{L}$. Построим $\mathcal{M}'$, которая будет эмулировать вывод очередного слова у исходной машины и проверять, равно ли оно $x$ (переходя в $q_{yes}$ в случае положительного результата). Если $x \in \mathcal{L}$, то это будет определено за конечное время, в противном случае $\mathcal{M}'$ не остановится, либо перейдет в $q_{no}$ при остановке $\mathcal{M}$.

    \item $\Leftarrow$

    Пусть $\mathcal{M}$ допускает язык $\mathcal{L}$. Соберем бесконечную таблицу: сверху будет возрастающее количество тактов, слева --- все возможные слова по порядку. $\mathcal{M}'$ будет обходить ее змейкой и запускать $\mathcal{M}$ с нужным входным словом на соответствующее число тактов. Если $\mathcal{M}$ остановилась в $q_{yes}$, то выведем текущее слово. В результате получили МТ, которая перечисляет $\mathcal{L}$, т. к. выводятся только слова из $\mathcal{L}$, при этом любое слово будет выведено за конечное время.

    \begin{figure}[H]
            \centering
            \includegraphics[width=0.40\linewidth]{33_1.png}
            \caption{Пример таблицы}
    \end{figure}
\end{itemize}
$\blacksquare$

\Note Не все языки разрешимы.

Количество МТ счетно, языков всего $2^{\Sigma^*}$ --- более чем счетно.

\Def 

$f: \mathbb{N}^k \to \mathbb{N}$ называется \textit{вычислимой}, если существует МТ $\mathcal{M}$ : $\forall (x_1, x_2, \dots x_k)$ из области определения $f$ по входу 

$\underbrace{1 1 \dots 1}_{x_1} \# \underbrace{1 1 \dots 1}_{x_2} \# \dots \# \underbrace{1 1 \dots 1}_{x_k}$ 

останавливается за конечное время с 

$\underbrace{1 1 \dots 1}_{f(x_1, x_2, \dots x_k) + 1}$

на выходной ленте.

\textbf{Проблема останова}

\Def

$\text{HALT} := \{ \langle \mathcal{M}, w \rangle \mid \mathcal{M} \; \text{останавливается на входе} \; w\}$

\Statement

Язык HALT не является разрешимым.

$\square$

Пронумеруем все МТ и закодирум каждое слово алфавита $\Sigma$ натуральным числом.

Пусть существует такая МТ $\mathcal{H}$, что она отрабатывает за конечное время на корректном входе и

$
\mathcal{H}(\mathcal{M}, w) =
\begin{cases}
1, & \mathcal{M} \; \text{останавливается на} \; w\\
0, & \mathcal{M} \; \text{не останавливается на} \; w\\
\end{cases}
$

Тогда вычислима и функция $\mathcal{F}(i) := \mathcal{H}(\mathcal{M}_i, i)$.

Построим МТ $\mathcal{H}'$, которая принимает на вход $i$ и запускает вычисление $\mathcal{F}(i)$.

\begin{enumerate}
    \item Если вычисление $\mathcal{F}(i)$ завершилось с результатом 1, то $H'$ переходит в вечный цикл.
    \item Если вычисление $\mathcal{F}(i)$ завершилось с результатом 0, то $H'$ останавливается.
\end{enumerate}

Но у $\mathcal{H}$ есть свой номер $N$.

Теперь рассмотрим работу $\mathcal{H'}(N)$.

\begin{enumerate}
    \item Если $\mathcal{H'}$ остановилась, то $\mathcal{F}(N) = 0$ по определению $\mathcal{H'}$, но из определения $\mathcal{H}$ следует $\mathcal{F}(N) = \mathcal{H}(\mathcal{M}_N, N) = 1$ Получили противоречие.

    \item Если $\mathcal{H'}$ не остановилась, то $\mathcal{F}(N) = 1$ по определению $\mathcal{H'}$,но из определения $\mathcal{H}$ следует $\mathcal{F}(N) = \mathcal{H}(\mathcal{M}_N, N) = 0$ Получили противоречие.

$\blacksquare$

\textbf{Функция трудолюбия Радо}

\Def 

$R(n) :=$ \\ Количество единиц, которое может вывести на ленту и остановиться МТ с не более чем $n$ состояниями, начиная с пустого входа.

\Statement

$R(n)$ не является вычислимой.

$\square$

Пусть существует $\mathcal{M}_R$ с $C_1$ состояниями, вычисляющая $R(n)$.

Очевидно, что $R(n)$ нестрого возрастающая.

Существует МТ, которая имеет $3n + 1$ состояние, печатает $4n + 1$ единицу с пустого входа и останавливается. \textit{Без доказательства}.

Построим МТ с $3n + 1 + C_1 + C_2$ состояниями, которая будет:

\begin{enumerate}
    \item Печатать на ленту $4n + 1$ единицу.
    \item Возвращаться в начало слова.
    \item Запускать $\mathcal{M}_R$.
\end{enumerate}

Тогда $R(3n + 1 + C_1 + C_2) \ge R(4n)$. Получим противоречие при достаточно больших $n$.

$\blacksquare$

\Def

Пусть $\mathcal{A}, \mathcal{B}$ --- языки.

Говорят, что $\mathcal{A} \le_m \mathcal{B}$, если $\exists f: \Sigma^* \to \Sigma^*$ вычислимая и всюду определенная : $ x \in \mathcal{A} \iff f(x) \in \mathcal{B}$.

\Statement

Пусть $\mathcal{A} \le_m \mathcal{B}$

\begin{itemize}
    \item Если $\mathcal{B}$ разрешимый, то $\mathcal{A}$ тоже.
    \item Если $\mathcal{A}$ не разрешимый, то $\mathcal{B}$ тоже.
\end{itemize}
