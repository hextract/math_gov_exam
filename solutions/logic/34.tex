\subsection{Разделяющие суффиксы, теорема Майхилла-Нероуда.}

\Def \textit{Алфавит} $\Sigma$ --- конечное множество.

\Def \textit{Слово} над алфавитом --- конечная последовательность элементов $\Sigma$.

\Def \textit{Пустое слово} ($\varepsilon$) --- последовательность из 0 элементов.

\Def 

\textit{Конкатенация} $x, y$ ($xy$) --- слова $x$ и $y$, записанные подряд.

$x^n := \underbrace{xx \dots x}_{n}$

\Def $\Sigma^*$ --- множество всех слов над алфавитом $\Sigma$.

\Def $\mathcal{L} \subseteq \Sigma^*$ называется \textit{формальным языком}.

\Def

\begin{enumerate}
    \item \textit{Конкатенация языков} $\mathcal{X} \cdot \mathcal{Y} := \{xy \mid x \in \mathcal{X}, y \in \mathcal{Y}\}$
    
    \item \textit{Степень} $\mathcal{X}^n := \underbrace{\mathcal{X} \cdot \mathcal{X} \dots \mathcal{X}}_n$
    
    $\mathcal{X}^0 := \{ \varepsilon \}$
    \item \textit{Объединение языков} $\mathcal{X} \vert \mathcal{Y} := \mathcal{X} + \mathcal{Y} := \{z \mid z \in \mathcal{X} \cup \mathcal{Y} \}$

    \item \textit{Звезда Клини} $\mathcal{X}^* := \mathcal{X}^0 + \mathcal{X}^1 + \mathcal{X}^2 + \dots$

    \item \textit{Плюс Клини} $\mathcal{X}^+ := \mathcal{X}^1 + \mathcal{X}^2 + \dots$
\end{enumerate}

\Def Пусть задан алфавит $\Sigma$.

REG --- множество \textit{регулярных языков}, при этом:

\begin{enumerate}
    \item $\varnothing \in $ REG.
    \item $\forall a \in \Sigma \quad \{ a \} \in $ REG.
    \item $\forall \mathcal{A}, \mathcal{B} \in $ REG $\quad \mathcal{A} \cdot \mathcal{B} \in $ REG.
    \item $\forall \mathcal{A}, \mathcal{B} \in $ REG $\quad \mathcal{A} + \mathcal{B} \in $ REG.
    \item $\forall \mathcal{A} \in $ REG $\quad \mathcal{A}^* \in $ REG.
\end{enumerate}

\underline{Других нет.}

\textbf{Следствие.}

\begin{itemize}
    \item $\{ \varepsilon \} \in $ REG.
    \item $\mathcal{A}^+ \in $ REG.
\end{itemize}

\Def \textit{Конечный автомат} --- кортеж ($\Sigma, Q, Q_f, q_0, \delta$), где:

\begin{enumerate}
    \item $\Sigma$ --- алфавит.
    \item $Q$ --- множество состояний.
    \item $Q_f \subseteq Q$ --- множество терминальных состояний.
    \item $q_0 \in Q$ --- начальное состояние.
    \item $\delta$ --- правила перехода.
\end{enumerate}

\Def \textit{ДКА (детерменированный конечный автомат)} --- конечный автомат, у которого $\delta : Q \times \Sigma \to Q$.

\Def \textit{НКА (недетерменированный конечный автомат)} --- конечный автомат, у которого $\delta : Q \times \Sigma \to 2^Q$.

\Def \textit{ПДКА (полный детерменированный конечный автомат)} --- ДКА, у которого область определения $\delta$ совпадает с $Q \times \Sigma$.

\Def \textit{Конфигурация ДКА} --- пара $( q, \alpha )$. Значит, что сейчас автомат в состоянии $q$, нужно считать еще слово $\alpha$.

\Def Конфигурация $( q, \beta )$ \textit{выводится} из конфигурации $( p, \alpha )$ за один шаг

$(p, \alpha) \vdash (q, \beta)$

если $\exists c \in \Sigma : \\ \begin{cases}
    \alpha = c \beta \\
    q = \delta(p, c)
\end{cases}$

Выводимость за конечное число шагов будем записывать следующим образом:

$(p, \alpha) \vdash^* (q, \beta)$

\Def ДКА \textit{принимает} слово $w$, если $\exists q_f \in Q_f : (q_0, w) \vdash^* (q_f, \varepsilon)$

\Statement

НКА можно свести к ДКА. Множества языков, задаваемых ими, равны (Теорема Клини).

ДКА можно свести к ПДКА (добавить мусорное состояние и все несуществующие переходы в него).

\Statement

Пусть DA --- языки, задаваемые ДКА.

REG $=$ DA.

\Def

Введем $\equiv_\mathcal{L}$ --- эквивалентность по языку $\mathcal{L} \subseteq \Sigma^*$.

$x \equiv_\mathcal{L} y$, если $\forall z \in \Sigma^* \quad xz \in \mathcal{L} \iff yz \in \mathcal{L}$

\Def 

Суффикс $z$ будем называть \textit{разделяющим} по языку $\mathcal{L}$ для слов $x$ и $y$, если

$
\left[
    \begin{array}{l}
        \begin{cases}
            xz \in \mathcal{L} \\
            yz \notin \mathcal{L}
        \end{cases} \\
        \begin{cases}
            xz \notin \mathcal{L} \\
            yz \in \mathcal{L}
        \end{cases}
    \end{array}
\right.
$

\Statement

$\equiv_\mathcal{L}$ --- действительно отношение эквивалентности.

\begin{enumerate}
    \item Рефлексивность $x \equiv_\mathcal{L} x \quad \forall x \in \Sigma^*$
    \item Симметричность $a \equiv_\mathcal{L} b \implies b \equiv_\mathcal{L} a \quad \forall a, b \in \Sigma^*$
    \item Транзитивность $a \equiv_\mathcal{L} b \equiv_\mathcal{L} c \implies a \equiv_\mathcal{L} c \quad \forall a, b, c \in \Sigma^*$
\end{enumerate}

\Def 

$[w]_\mathcal{L}$ --- класс эквивалентности по языку $\mathcal{L}$, в котором лежит слово $w$.

\Th \textbf{Майхилла--Нероуда.}

Язык регулярный $\iff$ количество классов эквивалентности по этому языку конечно.

$\square$

\begin{itemize}
    \item $\Rightarrow$
    
    Построим ПДКА по регулярному языку. Каждое состояние будет соответствовать своему классу эквивалентности.
    \item $\Leftarrow$

    Построим ПДКА для распознавания языка.

    \begin{enumerate}
        \item Каждому классу эквивалентности сопоставим состояние автомата.
        \item Начальное состояние $q_0 = [\varepsilon]_\mathcal{L}$
        \item $Q_f = \{[w]_\mathcal{L} \mid w \in \mathcal{L}\}$
    \end{enumerate}
\end{itemize}

$\blacksquare$
