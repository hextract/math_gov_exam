\subsection{Размещения, перестановки. Сочетания без повторений и с повторениями. Бином Ньютона.}

Пусть дано множество из $n$ различных элементов.

\Def \textit{Перестановка} --- упорядоченная цепочка из $n$ элементов без повторений.

\Def \textit{Размещение} --- упорядоченная цепочка из $k \le n$ элементов без повторений.

\Statement Существует ровно $n! := \displaystyle\prod_{i=1}^n i$ вариантов перестановок из n элементов.

$\square$

Рассмотрим цепочку из $n$ элементов. На первую позицию можно поставить любой из $n$ элементов. На позицию $i$ можно поставить любой из $n + 1 - i$ оставшихся. Получаем требуемое равенство.

$\blacksquare$


\Statement Существует ровно $A_n^k := \frac{n!}{(n-k)!}$ вариантов размещений $k$ элементов и $n$ -- элементного множества.

$\square$

Рассмотрим цепочку из $k$ элементов. По рассуждениям в пункте выше получим  $\displaystyle\prod_{i=1}^k (n + 1 - i) = \frac{n!}{(n-k)!}$ вариантов размещений.

$\blacksquare$

\Def \textit{Сочетание без повторений} --- $k$ -- элементное подмножество заданного множества. $k \le n$.

\Def \textit{Сочетание c повторениями} --- $k$ -- элементный набор из заданного множества, в котором каждый элемент может встречаться несколько раз.

\Statement Существует ровно $C_n^k := \binom{n}{k} := \frac{n!}{(n-k)! k!}$ сочетаний без повторений из $k$ элементов $n$ -- элементного множества.

$\square$

Каждому сочетанию соответствует ровно $k!$ размещений. Различным сочетаниям не может соответствовать одно и то же размещение.

Значит $C_n^k = \frac{A_n^k}{k!} = \frac{n!}{(n-k)! k!}$

$\blacksquare$

\Statement Существует ровно $C_{n + k -1}^{k}$ сочетаний с повторениями из $k$ элементов $n$ -- элементного множества.

$\square$

Заметим, что каждому такому сочетанию соответствует разбиение $k$ на $n$ неотрицательных целых слагаемых с учетом порядка.

Рассмотрим $k$ белых и $n-1$ чёрных шариков.

Расставим их все в один ряд. Каждой получившейся цепочке однозначно соответствует разложение $k$ на $n$ неотрицательных целых слагаемых.

Для доказательства построим биекцию. Пусть $k = k_1 + k_2 + \dots + k_n$. Ему будет соответствовать следующий набор:

\[
\underbrace{\bigcirc\!\bigcirc\!\cdots\!\bigcirc}_{k_1}\;{\large\bullet}\;
\underbrace{\bigcirc\!\bigcirc\!\cdots\!\bigcirc}_{k_2}\;{\large\bullet}\;
\cdots\;
{\large\bullet}\;
\underbrace{\bigcirc\!\bigcirc\!\cdots\!\bigcirc}_{k_n}
\]

\[
k = k_1 + k_2 + \dots + k_n
\]



Аналогично и в обратную сторону.

Заметим, что таких цепочек $C_{n + k - 1}^k$ --- количество вариантов выбрать позиции для черных шариков.

$\blacksquare$

\textbf{Биномиальный коэффициент}

Рассмотрим выражение $(x+y)^n$. Задача --- определить коэффициент при $x^ky^{n-k}$.

Пронумеруем скобки следующим образом: $(x_1 + y_1)\cdot(x_2 + y_2) \cdot \dots \cdot (x_n + y_n)$. Заметим, что коэффициент будет равен количеству вариантов выбора $k$ cкобок, из которых будет выбран $x$. А значит равен $C_n^k$.

\textbf{Полиномиальный коэффициент}

Рассмотрим выражение $(x_1 + x_2 + \dots + x_m)^n$. Задача --- определить коэффициент при $x^{n_1}x^{n_2}\dots x^{n_m}$, где $\displaystyle\sum_{i=1}^m x^{n_i} = n$.

Занумеруем скобки. Выбрать скобки с $x_1$ будет $C_n^{n_1}$ способов. Из оставшихся с $x_2$ --- $C_{n-n_1}^{n_2}$ способов и т. д. В итоге получим $C_n^{n_1} \cdot C_{n-n_1}^{n_2}\dots C^{n_m}_{n - n_1 - n_2 - \dots - n_m} = \frac{n!}{n_1! \cdot n_2! \dots n_m!}$ способов.
