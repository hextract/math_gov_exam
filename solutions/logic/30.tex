\subsection{Двудольные графы и паросочетания. Теорема Холла.}

\Def Граф $G = \langle L \cup R, E \rangle$ называется \textit{двудольным}, если $\forall (u, v) \in E$ $u$ и $v$ лежат в разных множествах $L, R$.

\Def \textit{Паросочетанием} в графе называется набор попарно несмежных ребер.

\Def Паросочетание называют \textit{совершенным}, если оно покрывает все вершины графа.

\Def Для $X \subset V$ \textit{множеством соседей} назовем $N(X) := \{ y \in V \mid (x, y) \in E, x \in X \}$.

\textbf{Теорема Холла.}

В двудольном графе $G \; \exists$ паросочетание с участием всех вершин $L$.

$\iff$

$\forall A \subset L \quad |N(A)| \ge |A|$.

$\square$
\begin{itemize}
  \item $\Rightarrow$ Если такое паросочетание существует, то количество соседей следует из определения паросочетания.
  \item $\Leftarrow$ Будем доказывать индукцией по размеру $L$.

    \textbf{База.} $|L| = 1$. Тогда у единственной вершины ровно один сосед. Выберем паросочетание с ним.

    \textbf{Переход.} $|L| = k + 1$. Разделим $L$ на $(l, L' := L \setminus \{l\})$. По предположению индукции построим $P'$ --- паросочетание, покрывающее $L'$. Пусть
    $H$ --- множество всех достижимых из $l$ вершин, таких, что из $L \to R$ путь идет по любым ребрам, а из $R \to L$ только по ребрам из $P'$. Тогда в $H_R := H \cap
    R$ найдется хотя бы одна вершина $r : r \notin P'$. Иначе не будет выполнено условие на количество соседей для $H_L$. Тогда существует путь $l \rightsquigarrow r$,
    чередование вдоль которого увеличит размер паросочетания на 1.
\end{itemize}

\begin{figure}[H]
  \centering
  \includegraphics[width=0.25\linewidth]{30_1.png}
  \caption{Пример увеличивающей цепи}
\end{figure}
$\blacksquare$
