\subsection{Математическая логика и теория сложности}

\subsubsection{11. Полнота системы булевых функций}

\textbf{Условие.} Определите, полна ли система функций $F = \{(01011010), (10111101), (00011101)\}$?

\Solution

\textbf{Критерий полноты (теорема Поста).}

Система булевых функций полна тогда и только тогда, когда она не содержится ни в одном из пяти замкнутых классов:
\begin{itemize}
  \item $T_0$ — функции, сохраняющие 0: $f(0, 0, \ldots, 0) = 0$;
  \item $T_1$ — функции, сохраняющие 1: $f(1, 1, \ldots, 1) = 1$;
  \item $S$ — самодвойственные функции: $f(\overline{x_1}, \ldots, \overline{x_n}) = \overline{f(x_1, \ldots, x_n)}$;
  \item $M$ — монотонные функции;
  \item $L$ — линейные функции (представимые как $a_0 \oplus a_1 x_1 \oplus \ldots \oplus a_n x_n$).
\end{itemize}

\textbf{Шаг 1. Интерпретируем функции.}

Каждая функция задана вектором значений из 8 бит, то есть это функции от 3 переменных.

Обозначим наборы аргументов $(x_1, x_2, x_3)$ в лексикографическом порядке:
\begin{center}
  \begin{tabular}{c|c|c|c|c|c|c|c|c}
    $(x_1, x_2, x_3)$ & 000 & 001 & 010 & 011 & 100 & 101 & 110 & 111 \\
    \hline
    $f_1$ & 0 & 1 & 0 & 1 & 1 & 0 & 1 & 0 \\
    $f_2$ & 1 & 0 & 1 & 1 & 1 & 1 & 0 & 1 \\
    $f_3$ & 0 & 0 & 0 & 1 & 1 & 1 & 0 & 1
  \end{tabular}
\end{center}

\textbf{Шаг 2. Проверяем класс $T_0$.}

$f \in T_0 \Leftrightarrow f(0,0,0) = 0$.
\begin{itemize}
  \item $f_1(0,0,0) = 0$ \checkmark
  \item $f_2(0,0,0) = 1$ — не принадлежит $T_0$
  \item $f_3(0,0,0) = 0$ \checkmark
\end{itemize}

Система содержит $f_2 \notin T_0$, значит система не целиком в $T_0$.

\textbf{Шаг 3. Проверяем класс $T_1$.}

$f \in T_1 \Leftrightarrow f(1,1,1) = 1$.
\begin{itemize}
  \item $f_1(1,1,1) = 0$ — не принадлежит $T_1$
  \item $f_2(1,1,1) = 1$ \checkmark
  \item $f_3(1,1,1) = 1$ \checkmark
\end{itemize}

Система содержит $f_1 \notin T_1$, значит система не целиком в $T_1$.

\textbf{Шаг 4. Проверяем класс $S$ (самодвойственность).}

$f$ самодвойственна $\Leftrightarrow f(\bar{x}) = \overline{f(x)}$ для всех $x$.

Это означает, что $f(x) + f(\bar{x}) = 1$ для всех $x$.

Для функции от 3 переменных достаточно проверить пары: $(000, 111)$, $(001, 110)$, $(010, 101)$, $(011, 100)$.

\textit{Проверяем $f_1$:}
\begin{itemize}
  \item $f_1(000) = 0$, $f_1(111) = 0$. Должно быть $0 + 0 = 1$? Нет.
\end{itemize}

$f_1 \notin S$.

Система не целиком в $S$.

\textbf{Шаг 5. Проверяем класс $M$ (монотонность).}

$f$ монотонна $\Leftrightarrow$ из $x \leq y$ (покоординатно) следует $f(x) \leq f(y)$.

\textit{Проверяем $f_1$:}

$f_1(001) = 1$, $f_1(101) = 0$. Но $(0,0,1) \leq (1,0,1)$, значит должно быть $f_1(001) \leq f_1(101)$, т.е. $1 \leq 0$ — противоречие.

$f_1 \notin M$.

Система не целиком в $M$.

\textbf{Шаг 6. Проверяем класс $L$ (линейность).}

$f$ линейна $\Leftrightarrow f(x_1, x_2, x_3) = a_0 \oplus a_1 x_1 \oplus a_2 x_2 \oplus a_3 x_3$.

Линейная функция от 3 переменных имеет вид $a_0 \oplus a_1 x_1 \oplus a_2 x_2 \oplus a_3 x_3$, где $a_i \in \{0, 1\}$.

Для линейной функции: полином Жегалкина не содержит членов степени $\geq 2$.

\textit{Строим полином Жегалкина для $f_1$:}

Используем метод неопределённых коэффициентов. Пусть:
\[
  f_1 = a_0 \oplus a_1 x_1 \oplus a_2 x_2 \oplus a_3 x_3 \oplus a_{12} x_1 x_2 \oplus a_{13} x_1 x_3 \oplus a_{23} x_2 x_3 \oplus a_{123} x_1 x_2 x_3.
\]

Из $f_1(000) = 0$: $a_0 = 0$.

Из $f_1(001) = 1$: $a_0 \oplus a_3 = 1 \Rightarrow a_3 = 1$.

Из $f_1(010) = 0$: $a_0 \oplus a_2 = 0 \Rightarrow a_2 = 0$.

Из $f_1(100) = 1$: $a_0 \oplus a_1 = 1 \Rightarrow a_1 = 1$.

Из $f_1(011) = 1$: $a_0 \oplus a_2 \oplus a_3 \oplus a_{23} = 0 \oplus 0 \oplus 1 \oplus a_{23} = 1 \Rightarrow a_{23} = 0$.

Из $f_1(101) = 0$: $a_0 \oplus a_1 \oplus a_3 \oplus a_{13} = 0 \oplus 1 \oplus 1 \oplus a_{13} = 0 \Rightarrow a_{13} = 0$.

Из $f_1(110) = 1$: $a_0 \oplus a_1 \oplus a_2 \oplus a_{12} = 0 \oplus 1 \oplus 0 \oplus a_{12} = 1 \Rightarrow a_{12} = 0$.

Из $f_1(111) = 0$: $a_0 \oplus a_1 \oplus a_2 \oplus a_3 \oplus a_{12} \oplus a_{13} \oplus a_{23} \oplus a_{123} = 0 \oplus 1 \oplus 0 \oplus 1 \oplus 0 \oplus 0 \oplus
0 \oplus a_{123} = 0 \Rightarrow a_{123} = 0$.

Итак, $f_1 = x_1 \oplus x_3$ — линейная функция!

\textit{Проверяем $f_2$:}

Из $f_2(000) = 1$: $a_0 = 1$.

Из $f_2(001) = 0$: $a_0 \oplus a_3 = 0 \Rightarrow a_3 = 1$.

Из $f_2(010) = 1$: $a_0 \oplus a_2 = 1 \Rightarrow a_2 = 0$.

Из $f_2(100) = 1$: $a_0 \oplus a_1 = 1 \Rightarrow a_1 = 0$.

Из $f_2(011) = 1$: $1 \oplus 0 \oplus 1 \oplus a_{23} = 1 \Rightarrow a_{23} = 1 \neq 0$.

Функция $f_2$ содержит нелинейный член $x_2 x_3$, значит $f_2 \notin L$.

Система не целиком в $L$.

\textbf{Вывод.}

Система $F = \{f_1, f_2, f_3\}$ не содержится целиком ни в одном из пяти замкнутых классов Поста:
\begin{itemize}
  \item $f_2 \notin T_0$;
  \item $f_1 \notin T_1$;
  \item $f_1 \notin S$;
  \item $f_1 \notin M$;
  \item $f_2 \notin L$.
\end{itemize}

По теореме Поста система \textbf{полна}.

\textbf{Ответ:} Да, система $F$ полна.

\subsubsection{12. Разрешимость языка с лексикографическим перечислением}

\textbf{Условие.} Докажите, что если слова языка $L$ можно перечислить в лексикографическом порядке, то язык $L$ разрешим.

\Solution

\textbf{Определения.}

\begin{itemize}
  \item Язык $L \subseteq \Sigma^*$ называется \textit{перечислимым}, если существует алгоритм (машина Тьюринга), который перечисляет все слова языка $L$ (возможно, в
    произвольном порядке).
  \item Язык $L$ называется \textit{разрешимым}, если существует алгоритм, который для любого слова $w \in \Sigma^*$ за конечное время отвечает, принадлежит ли $w$ языку $L$.
  \item \textit{Лексикографический порядок} на $\Sigma^*$: сначала сравниваем по длине, затем при равной длине — посимвольно слева направо.
\end{itemize}

\textbf{Доказательство.}

Пусть существует алгоритм $E$, перечисляющий слова языка $L$ в лексикографическом порядке: $w_1, w_2, w_3, \ldots$

Покажем, как построить алгоритм $A$, разрешающий $L$.

\textbf{Алгоритм $A$ для проверки $w \in L$:}

\begin{enumerate}
  \item Запустить перечислитель $E$.
  \item На каждом шаге получаем очередное слово $w_i$ из $L$.
  \item Сравниваем $w_i$ с $w$:
    \begin{itemize}
      \item Если $w_i = w$, то $w \in L$. \textbf{Ответ: ДА.}
      \item Если $w_i > w$ в лексикографическом порядке, то $w \notin L$. \textbf{Ответ: НЕТ.}
      \item Если $w_i < w$, продолжаем перечисление.
    \end{itemize}
  \item Если перечисление $E$ завершилось (язык $L$ конечен) и слово $w$ не встретилось, то $w \notin L$. \textbf{Ответ: НЕТ.}
\end{enumerate}

\textbf{Корректность.}

Ключевое свойство лексикографического порядка: если слова перечисляются в порядке $w_1 < w_2 < w_3 < \ldots$, то:
\begin{itemize}
  \item Если $w \in L$, то $w = w_k$ для некоторого $k$, и мы найдём его за конечное число шагов.
  \item Если $w \notin L$, то либо перечисление закончится раньше, чем мы дойдём до $w$, либо мы встретим слово $w_i > w$, и тогда все последующие слова также будут больше $w$.
\end{itemize}

В обоих случаях алгоритм $A$ завершается за конечное время.

\textbf{Замечание.} Без условия лексикографической упорядоченности это неверно: существуют перечислимые, но неразрешимые языки (например, язык кодов останавливающихся
машин Тьюринга).

\subsubsection{13. Совпадение классов}

\textbf{Условие.} Докажите, что $\text{co}(\text{NPC}) = (\text{coNP})C$.

\textbf{Расшифровка обозначений:}
\begin{itemize}
  \item $\text{NPC}$ — класс NP-полных задач;
  \item $\text{co}(\text{NPC})$ — дополнения задач из NPC (если $L \in \text{NPC}$, то $\overline{L} \in \text{co}(\text{NPC})$);
  \item $\text{coNP}$ — класс языков, дополнения которых лежат в NP;
  \item $(\text{coNP})C$ — coNP-полные задачи.
\end{itemize}

\Solution

\textbf{Утверждение:} $L$ является NP-полной $\Leftrightarrow$ $\overline{L}$ является coNP-полной.

\textbf{Доказательство.}

\textbf{Часть 1.} Покажем, что $\overline{L} \in \text{coNP}$.

Если $L \in \text{NP}$, то по определению $\overline{L} \in \text{coNP}$.

\textbf{Часть 2.} Покажем, что $\overline{L}$ является coNP-трудной.

Пусть $L$ — NP-полная. Нужно показать, что любой язык $M \in \text{coNP}$ полиномиально сводится к $\overline{L}$.

Возьмём произвольный $M \in \text{coNP}$. Тогда $\overline{M} \in \text{NP}$.

Поскольку $L$ — NP-полная, существует полиномиальная сводимость $\overline{M} \leq_p L$, т.е. существует полиномиально вычислимая функция $f$ такая, что:
\[
  x \in \overline{M} \Leftrightarrow f(x) \in L.
\]

Тогда:
\[
  x \in M \Leftrightarrow x \notin \overline{M} \Leftrightarrow f(x) \notin L \Leftrightarrow f(x) \in \overline{L}.
\]

Таким образом, $M \leq_p \overline{L}$ с той же функцией $f$.

Поскольку $M$ — произвольный язык из coNP, получаем, что $\overline{L}$ является coNP-трудной.

\textbf{Часть 3.} Вместе с частью 1 это означает, что $\overline{L}$ — coNP-полная.

\textbf{Обратное направление.}

Если $\overline{L}$ — coNP-полная, то $\overline{\overline{L}} = L$ является NP-полной (по симметрии рассуждения).

\textbf{Вывод.}

Отображение $L \mapsto \overline{L}$ устанавливает биекцию между NPC и (coNP)C:
\[
  L \in \text{NPC} \Leftrightarrow \overline{L} \in (\text{coNP})C.
\]

Это в точности означает, что $\text{co}(\text{NPC}) = (\text{coNP})C$.

\textbf{Ответ:} Доказано. Взятие дополнения переводит NP-полные задачи в coNP-полные и обратно.
