\subsection{Гармонический анализ}

\subsubsection{14. Ряд Фурье по системе косинусов}

\textbf{Условие.} Построить график суммы ряда Фурье функции $f(x) = \frac{5}{2} - x$, $x \in \left[0; \frac{5}{2}\right]$ по системе функций $\left\{ \cos
\frac{(2k+1)\pi x}{5} \right\}_{k=1}^{+\infty}$ на одном периоде. Коэффициенты Фурье не вычислять! Исследовать ряд Фурье на равномерную сходимость.

\Solution

\textbf{Шаг 1. Анализ системы функций.}

Система $\left\{ \cos \frac{(2k+1)\pi x}{5} \right\}$ — это система косинусов с нечётными номерами гармоник на отрезке $\left[0;\frac{5}{2}\right]$ (обычно удобно
считать $k=0,1,2,\ldots$; сдвиг индекса ничего не меняет по смыслу).

Эта система соответствует разложению функции, которая:
\begin{itemize}
  \item Чётно продолжена относительно $x = 0$;
  \item Нечётно продолжена относительно $x = 5/2$.
\end{itemize}

Такое продолжение даёт функцию с периодом $T = 10$.

\textbf{Шаг 2. Построение продолжения.}

Исходная функция: $f(x) = \frac{5}{2} - x$ на $\left[0, \frac{5}{2}\right]$.

\textit{Продолжение на $\left[-\frac{5}{2}, 0\right]$ (чётное относительно 0):}
\[
  f(x) = f(-x) = \frac{5}{2} - (-x) = \frac{5}{2} + x, \quad x \in \left[-\frac{5}{2}, 0\right].
\]

\textit{Продолжение на $\left[\frac{5}{2}, 5\right]$ (нечётное относительно $\frac{5}{2}$):}

При нечётном продолжении относительно точки $a = \frac{5}{2}$:
\[
  f(a + t) = -f(a - t) + 2f(a) = -f(a - t) + 2 \cdot 0 = -f(a - t).
\]

Здесь $f(a) = f\left(\frac{5}{2}\right) = \frac{5}{2} - \frac{5}{2} = 0$.

Для $x = a + t \in \left[\frac{5}{2}, 5\right]$, т.е. $t \in \left[0, \frac{5}{2}\right]$:
\[
  f\left(\frac{5}{2} + t\right) = -f\left(\frac{5}{2} - t\right) = -\left(\frac{5}{2} - \left(\frac{5}{2} - t\right)\right) = -t.
\]

Подставляя $t = x - \frac{5}{2}$:
\[
  f(x) = -\left(x - \frac{5}{2}\right) = \frac{5}{2} - x, \quad x \in \left[\frac{5}{2}, 5\right].
\]

Итого на $[0, 5]$ формула единая:
\[
  f(x) = \frac{5}{2} - x,
\]
и она принимает как положительные значения (на $[0,\frac{5}{2})$), так и отрицательные (на $(\frac{5}{2},5]$).

На $[-\frac{5}{2}, 0]$:
\[
  f(x) = \frac{5}{2} + x.
\]

\textit{Продолжение на $\left[-5, -\frac{5}{2}\right]$ (нечётное относительно $-\frac{5}{2}$):}

Аналогично получаем $f(x) = -\frac{5}{2} - x$.

Далее функция продолжается периодически с периодом $T = 10$.

\textbf{Шаг 3. Описание продолженной функции на периоде $[-5, 5]$.}

\[
  \tilde{f}(x) =
  \begin{cases}
    \frac{5}{2} + x, & x \in \left[-5, 0\right], \\
    \frac{5}{2} - x, & x \in \left[0, 5\right].
  \end{cases}
\]

Это непрерывная кусочно-линейная периодическая функция с периодом $10$.

\textit{Проверка значений:}
\begin{itemize}
  \item $\tilde{f}(-5) = \frac{5}{2} + (-5) = -\frac{5}{2}$
  \item $\tilde{f}\left(-\frac{5}{2}\right) = \frac{5}{2} + \left(-\frac{5}{2}\right) = 0$
  \item $\tilde{f}(0) = \frac{5}{2}$
  \item $\tilde{f}\left(\frac{5}{2}\right) = 0$
  \item $\tilde{f}(5) = \frac{5}{2} - 5 = -\frac{5}{2}$
\end{itemize}

\textbf{Шаг 4. Сходимость ряда Фурье.}

По теореме Дирихле ряд Фурье сходится:
\begin{itemize}
  \item В точках непрерывности — к значению функции.
  \item В точках разрыва — к полусумме односторонних пределов.
\end{itemize}

Продолженная функция $\tilde{f}(x)$ непрерывна на $\R$ (проверьте совпадение значений на границах). Поэтому ряд Фурье сходится к $\tilde{f}(x)$ во всех точках.

\textbf{Шаг 5. График суммы ряда Фурье.}

\drawsomesmall{problems_14.png}

На интервале $\left[0, \frac{5}{2}\right]$ это исходная функция $f(x) = \frac{5}{2} - x$, убывающая от $\frac{5}{2}$ до $0$.

\textbf{Шаг 6. Исследование на равномерную сходимость.}

\textbf{Достаточное условие:} если $2\pi$-периодическая кусочно-гладкая функция (т.е непрерывна, а производная кусочно непрерывна), то её ряд Фурье сходится к ней равномерно.

Проверим условия для $\tilde{f}(x)$:

\begin{enumerate}
  \item \textbf{Непрерывность:} $\tilde{f}(x)$ непрерывна на $\R$ (проверено выше).

  \item \textbf{Кусочная гладкость:} На каждом интервале $\left(-5, 0\right)$, $\left(0, 5\right)$ функция линейна, следовательно, бесконечно дифференцируема. В точке
    излома $x = 0. x = \pm 5$ существуют односторонние производные.
\end{enumerate}

\textbf{Вывод:} Ряд Фурье функции $\tilde{f}(x)$ сходится равномерно на $[-5, 5]$.
