\subsection{Теория вероятностей}

\subsubsection{5. Формула Байеса}

\textbf{Условие.} На заводе есть три станка А, В и С, выпускающие деталь одного типа. На станке А делается 50\% всех деталей, на станке В — 30\%, на станке С — 20\%. Доля бракованных деталей на станке А составляет 3\%, на станке В — 2\%, на станке С — 1\%. Найти вероятность того, что выпущенная на заводе бракованная деталь сделана на станке А.

\Solution

Введём обозначения:
\begin{itemize}
    \item $A$, $B$, $C$ — события «деталь сделана на станке А, В, С» соответственно;
    \item $D$ — событие «деталь бракованная».
\end{itemize}

По условию:
\[
\mathbb{P}(A) = 0{,}5, \quad \mathbb{P}(B) = 0{,}3, \quad \mathbb{P}(C) = 0{,}2.
\]
\[
\mathbb{P}(D \mid A) = 0{,}03, \quad \mathbb{P}(D \mid B) = 0{,}02, \quad \mathbb{P}(D \mid C) = 0{,}01.
\]

Требуется найти $\mathbb{P}(A \mid D)$ — вероятность того, что бракованная деталь сделана на станке А.

По \textbf{формуле Байеса}:
\[
\mathbb{P}(A \mid D) = \frac{\mathbb{P}(A) \cdot \mathbb{P}(D \mid A)}{\mathbb{P}(D)}.
\]

Сначала найдём $\mathbb{P}(D)$ по \textbf{формуле полной вероятности}:
\[
\mathbb{P}(D) = \mathbb{P}(A)\mathbb{P}(D \mid A) + \mathbb{P}(B)\mathbb{P}(D \mid B) + \mathbb{P}(C)\mathbb{P}(D \mid C).
\]

Подставляем числа:
\[
\mathbb{P}(D) = 0{,}5 \cdot 0{,}03 + 0{,}3 \cdot 0{,}02 + 0{,}2 \cdot 0{,}01 = 0{,}015 + 0{,}006 + 0{,}002 = 0{,}023.
\]

Теперь применяем формулу Байеса:
\[
\mathbb{P}(A \mid D) = \frac{0{,}5 \cdot 0{,}03}{0{,}023} = \frac{0{,}015}{0{,}023} = \frac{15}{23} \approx 0{,}652.
\]

\textbf{Ответ:} $\dfrac{15}{23} \approx 0{,}652$.

\subsubsection{6. Формула полной вероятности}

\textbf{Условие.} В корзине 5 теннисных мячей: 3 новых и 2 старых (тех, которыми уже играли). Первый теннисист взял 2 мяча, поиграл ими и положил обратно в корзину. Потом пришёл второй теннисист и взял один мяч. Найти вероятность того, что этот мяч новый.

\Solution

Введём обозначения:
\begin{itemize}
    \item $H_0$ — первый теннисист взял 0 новых мячей (т.е. 2 старых);
    \item $H_1$ — первый теннисист взял 1 новый и 1 старый мяч;
    \item $H_2$ — первый теннисист взял 2 новых мяча;
    \item $N$ — событие «второй теннисист взял новый мяч».
\end{itemize}

\textbf{Шаг 1.} Найдём вероятности гипотез $H_0$, $H_1$, $H_2$.

Всего способов выбрать 2 мяча из 5: $C_5^2 = 10$.

\begin{itemize}
    \item $H_0$: выбрать 2 старых из 2 — $C_2^2 = 1$ способ. $\mathbb{P}(H_0) = \dfrac{1}{10}$.
    \item $H_1$: выбрать 1 новый из 3 и 1 старый из 2 — $C_3^1 \cdot C_2^1 = 6$ способов. $\mathbb{P}(H_1) = \dfrac{6}{10} = \dfrac{3}{5}$.
    \item $H_2$: выбрать 2 новых из 3 — $C_3^2 = 3$ способа. $\mathbb{P}(H_2) = \dfrac{3}{10}$.
\end{itemize}

\textbf{Шаг 2.} Найдём условные вероятности $\mathbb{P}(N \mid H_i)$.

После игры оба мяча становятся старыми. Поэтому после возврата мячей в корзину:

\begin{itemize}
    \item При $H_0$: было 3 новых, 2 старых $\to$ осталось 3 новых, 2 старых. $\mathbb{P}(N \mid H_0) = \dfrac{3}{5}$.
    \item При $H_1$: был 1 новый среди взятых $\to$ стал старым. Осталось 2 новых, 3 старых. $\mathbb{P}(N \mid H_1) = \dfrac{2}{5}$.
    \item При $H_2$: оба взятых были новыми $\to$ оба стали старыми. Осталось 1 новый, 4 старых. $\mathbb{P}(N \mid H_2) = \dfrac{1}{5}$.
\end{itemize}

\textbf{Шаг 3.} Применяем \textbf{формулу полной вероятности}:
\[
\mathbb{P}(N) = \mathbb{P}(H_0)\mathbb{P}(N \mid H_0) + \mathbb{P}(H_1)\mathbb{P}(N \mid H_1) + \mathbb{P}(H_2)\mathbb{P}(N \mid H_2).
\]

Подставляем:
\[
\mathbb{P}(N) = \frac{1}{10} \cdot \frac{3}{5} + \frac{6}{10} \cdot \frac{2}{5} + \frac{3}{10} \cdot \frac{1}{5}
= \frac{3}{50} + \frac{12}{50} + \frac{3}{50} = \frac{18}{50} = \frac{9}{25}.
\]

\textbf{Ответ:} $\dfrac{9}{25} = 0{,}36$.

