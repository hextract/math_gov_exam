\subsection{Линейная алгебра}

\subsubsection{8. Прямая как пересечение плоскостей}

\textbf{Условие.} В общей декартовой системе координат прямая задана пересечением плоскостей:
\[
  \begin{cases}
    2x + 3y - z - 2 = 0 \\
    4x + y - z + 2 = 0
  \end{cases}
\]
Найти каноническое уравнение прямой.

\Solution

\textbf{Шаг 1. Находим направляющий вектор прямой.}

Направляющий вектор прямой — это ненулевой вектор из ядра матрицы $A$, т.е. решение однородной системы $Ad = 0$
\[
  \begin{cases}
    2a + 3b - c = 0 \\
    4a + b - c = 0
  \end{cases} \text{, где} \ d = (a, b, c)
\]
\[
  (1) - (2) \Rightarrow -2a + 2b = 0 \Leftrightarrow a = b
\]
Подставляя, получим
\[
  5a - c = 0 \Leftrightarrow c = 5a
\]
Таким образом имеем вектор $d = a (1, 1, 5), a \in \R$. Можно взять $a = 1 \Rightarrow d = (1, 1, 5)$

\textbf{Шаг 2. Находим точку на прямой.}

Положим $z = 0$ и решим систему:
\[
  \begin{cases}
    2x + 3y = 2 \\
    4x + y = -2
  \end{cases}
\]

Из второго уравнения: $y = -2 - 4x$.

Подставляем в первое:
\[
  2x + 3(-2 - 4x) = 2 \quad \Rightarrow \quad 2x - 6 - 12x = 2 \quad \Rightarrow \quad -10x = 8 \quad \Rightarrow \quad x = -\frac{4}{5}.
\]

Тогда:
\[
  y = -2 - 4 \cdot \left(-\frac{4}{5}\right) = -2 + \frac{16}{5} = \frac{-10 + 16}{5} = \frac{6}{5}.
\]

Точка на прямой: $M_0 = \left(-\frac{4}{5}, \frac{6}{5}, 0\right)$.

\textbf{Шаг 3. Записываем каноническое уравнение.}

Каноническое уравнение прямой:
\[
  \frac{x - x_0}{s_x} = \frac{y - y_0}{s_y} = \frac{z - z_0}{s_z}.
\]

Подставляем:
\[
  \boxed{\frac{x + \frac{4}{5}}{1} = \frac{y - \frac{6}{5}}{1} = \frac{z - 0}{5}}
\]

\subsubsection{9. Система линейных уравнений}

\textbf{Условие.} Найти общее решение системы линейных уравнений с 5 неизвестными, указать частное решение и ФСР однородной системы:
\[
  \begin{cases}
    3x_1 + 2x_2 + x_3 - x_4 = 7 \\
    -2x_1 + 3x_2 + 12x_3 + 2x_5 = 18
  \end{cases}
\]

\Solution

\textbf{Шаг 1. Записываем расширенную матрицу.}

\[
  \left(
    \begin{array}{ccccc|c}
      3 & 2 & 1 & -1 & 0 & 7 \\
      -2 & 3 & 12 & 0 & 2 & 18
    \end{array}
  \right)
\]

Заметим, что в матрице достаточно удачно уже выражены $x_4, x_5$, таким образом что они не зависят друг от друга. Можем положить их базисными переменными, а свободными
оставить остальные (т.к ранг матрицы - 2, необходимо ровно 2 базисных переменных).

\textbf{Шаг 2. Выражаем базисные переменные.}

\[
  x_4 = 3x_1 + 2x_2 + x_3 - 7
\]
\[
  x_5 = x_1 - 1.5x_2 - 6x_3 + 9
\]

\textbf{Шаг 4. Общее решение.}

Обозначим $x_1 = t$, $x_2 = s$, $x_3 = u$ — свободные параметры.

\[
  x_4 = 3t + 2s + u - 7, \quad x_5 = t - 1.5s - 6u + 9.
\]

\textbf{Шаг 5. Частное решение.}

При $t = s = u = 0$:
\[
  x_1 = 0, \quad x_2 = 0, \quad x_3 = 0, \quad x_4 = -7, \quad x_5 = 9.
\]

Частное решение: $\vec{x}_{\text{ч}} = \left( 0, 0, 0, -7, 9 \right)$.

\textbf{Шаг 6. ФСР однородной системы.}

Для однородной системы полагаем правую часть равной нулю.

Общее решение однородной системы:
\[
  x_1 = t, \quad x_2 = s, \quad x_3 = u, \quad x_4 = 3t + 2s + u, \quad x_5 = t - 1.5s - 6u.
\]

ФСР состоит из трёх векторов (по числу свободных переменных, или же $k = n - r = 5 - 2 = 3$):

При $t = 1, s = 0, u = 0$:
\[
  \vec{e}_1 = \left( 1, 0, 0, 3, 1 \right).
\]

При $t = 0, s = 1, u = 0$:
\[
  \vec{e}_2 = \left( 0, 1, 0, 2, -3/2 \right).
\]

При $t = 0, s = 0, u = 1$:
\[
  \vec{e}_3 = \left( 0, 0, 1, 1, -6 \right).
\]

\subsubsection{10. Ортогональное дополнение подпространства}

\textbf{Условие.} В евклидовом пространстве подпространство $L$ задано системой линейных уравнений с матрицей
\[
  \begin{pmatrix}
    1 & 1 & 1 & -17 \\
    2 & -13 & 0 & 1
  \end{pmatrix}
\]
в ОНБ. Найти базис в ортогональном дополнении $L^\perp$ подпространства $L$.

\Solution

\textbf{Теоретическая основа.}

Ортогональное дополнение есть множество векторов, таких, что каждый вектор из множества ортогонален всем векторам из $L$. Т.е можно записать это так:
\[
  \begin{cases}
    (a_1, x) = 0 \\
    (a_2, x) = 0
  \end{cases} \Leftrightarrow Ax = 0
\]

Это пространство можно задать $n-k = 4 - 2 = 2$ векторами $b_i$, такими, что для каждого из них верно $(b_i, a_j) = 0 \ \forall i, j$, а также эти векторы должны быть
ЛНЗ. Но мы же буквально написали определение ФСР. То есть задача свелась к нахождению ФСР системы $A$. Сделаем это

\[
  \begin{pmatrix}
    1 & 1 & 1 & -17 \\
    2 & -13 & 0 & 1
  \end{pmatrix} \sim
  \begin{pmatrix}
    0 & 15 & 2 & -35 \\
    2 & -13 & 0 & 1
  \end{pmatrix}
\]
Возьмем $x_1, x_3$ как базисные, $x_2, x_4$ как свободные:
\[
  x_1 = 6.5 s - 0.5 t \quad x_3 = -7.5 s + 17.5 t \quad x_2 = s \quad x_4 = t
\]
Осталось просто написать ФСР:
\begin{enumerate}
  \item $s = 1, t = 0$
    \[
      (6.5, 1, -7.5, 0) \sim (13, 2, -15, 0)
    \]
  \item $s = 0, t = 1$
    \[
      (-0.5, 0, 17.5, 1) \sim (-1, 0, 35, 2)
    \]
\end{enumerate}

Это и есть базис в $L^T$ по утверждениям в начале.
\[
  \begin{pmatrix}
    13 & 2 & -15 & 0 \\
    -1 & 0 & 35 & 2
  \end{pmatrix}
\]
Несложно в этом убедиться, посчитав в уме скалярные произведения.
