\subsection{Линейная алгебра}

\subsubsection{8. Прямая как пересечение плоскостей}

\textbf{Условие.} В общей декартовой системе координат прямая задана пересечением плоскостей:
\[
\begin{cases}
2x + 3y - z - 2 = 0 \\
4x + y - z + 2 = 0
\end{cases}
\]
Найти каноническое уравнение прямой.

\Solution

\textbf{Шаг 1. Находим направляющий вектор прямой.}

Направляющий вектор прямой — это ненулевой вектор из ядра матрицы $A$, т.е. решение однородной системы $Ad = 0$
\[
\begin{cases}
2a + 3b - c = 0 \\
4a + b - c = 0
\end{cases} \text{, где} \ d = (a, b, c)
\]
\[
(1) - (2) \Rightarrow -2a + 2b = 0 \Leftrightarrow a = b
\]
Подставляя, получим
\[
5a - c = 0 \Leftrightarrow c = 5a
\]
Таким образом имеем вектор $d = a (1, 1, 5), a \in \R$. Можно взять $a = 1 \Rightarrow d = (1, 1, 5)$ 

\textbf{Шаг 2. Находим точку на прямой.}

Положим $z = 0$ и решим систему:
\[
\begin{cases}
2x + 3y = 2 \\
4x + y = -2
\end{cases}
\]

Из второго уравнения: $y = -2 - 4x$.

Подставляем в первое:
\[
2x + 3(-2 - 4x) = 2 \quad \Rightarrow \quad 2x - 6 - 12x = 2 \quad \Rightarrow \quad -10x = 8 \quad \Rightarrow \quad x = -\frac{4}{5}.
\]

Тогда:
\[
y = -2 - 4 \cdot \left(-\frac{4}{5}\right) = -2 + \frac{16}{5} = \frac{-10 + 16}{5} = \frac{6}{5}.
\]

Точка на прямой: $M_0 = \left(-\frac{4}{5}, \frac{6}{5}, 0\right)$.

\textbf{Шаг 3. Записываем каноническое уравнение.}

Каноническое уравнение прямой:
\[
\frac{x - x_0}{s_x} = \frac{y - y_0}{s_y} = \frac{z - z_0}{s_z}.
\]

Подставляем:
\[
\boxed{\frac{x + \frac{4}{5}}{1} = \frac{y - \frac{6}{5}}{1} = \frac{z - 0}{5}}
\]

\subsubsection{9. Система линейных уравнений}

\textbf{Условие.} Найти общее решение системы линейных уравнений с 5 неизвестными, указать частное решение и ФСР однородной системы:
\[
\begin{cases}
3x_1 + 2x_2 + x_3 - x_4 = 7 \\
-2x_1 + 3x_2 + 12x_3 + 2x_5 = 18
\end{cases}
\]

\Solution

\textbf{Шаг 1. Записываем расширенную матрицу.}

\[
\left(
\begin{array}{ccccc|c}
3 & 2 & 1 & -1 & 0 & 7 \\
-2 & 3 & 12 & 0 & 2 & 18
\end{array}
\right)
\]

\textbf{Шаг 2. Приводим к ступенчатому виду методом Гаусса.}

Умножим первую строку на 2, вторую на 3, затем сложим:

$(I) \cdot 2$: $(6, 4, 2, -2, 0 \mid 14)$

$(II) \cdot 3$: $(-6, 9, 36, 0, 6 \mid 54)$

$(I) \cdot 2 + (II) \cdot 3$: $(0, 13, 38, -2, 6 \mid 68)$

Новая матрица:
\[
\left(
\begin{array}{ccccc|c}
3 & 2 & 1 & -1 & 0 & 7 \\
0 & 13 & 38 & -2 & 6 & 68
\end{array}
\right)
\]

\textbf{Шаг 3. Выражаем базисные переменные.}

Ранг матрицы = 2. Базисные переменные: $x_1$, $x_2$. Свободные переменные: $x_3$, $x_4$, $x_5$.

Из второй строки:
\[
13x_2 + 38x_3 - 2x_4 + 6x_5 = 68.
\]
\[
x_2 = \frac{68 - 38x_3 + 2x_4 - 6x_5}{13}.
\]

Из первой строки:
\[
3x_1 + 2x_2 + x_3 - x_4 = 7.
\]
\[
x_1 = \frac{7 - 2x_2 - x_3 + x_4}{3}.
\]

Подставим $x_2$:
\[
x_1 = \frac{7 - \frac{2(68 - 38x_3 + 2x_4 - 6x_5)}{13} - x_3 + x_4}{3}.
\]

Упрощаем:
\[
x_1 = \frac{1}{3} \left( 7 - x_3 + x_4 - \frac{136 - 76x_3 + 4x_4 - 12x_5}{13} \right).
\]

\[
x_1 = \frac{1}{3} \cdot \frac{91 - 13x_3 + 13x_4 - 136 + 76x_3 - 4x_4 + 12x_5}{13}.
\]

\[
x_1 = \frac{-45 + 63x_3 + 9x_4 + 12x_5}{39} = \frac{-15 + 21x_3 + 3x_4 + 4x_5}{13}.
\]

\textbf{Шаг 4. Общее решение.}

Обозначим $x_3 = t$, $x_4 = s$, $x_5 = u$ — свободные параметры.

\[
x_1 = \frac{-15 + 21t + 3s + 4u}{13}, \quad x_2 = \frac{68 - 38t + 2s - 6u}{13}.
\]

\textbf{Шаг 5. Частное решение.}

При $t = s = u = 0$:
\[
x_1 = \frac{-15}{13}, \quad x_2 = \frac{68}{13}, \quad x_3 = 0, \quad x_4 = 0, \quad x_5 = 0.
\]

Частное решение: $\vec{x}_{\text{ч}} = \left( -\frac{15}{13}, \frac{68}{13}, 0, 0, 0 \right)$.

\textbf{Шаг 6. ФСР однородной системы.}

Для однородной системы полагаем правую часть равной нулю.

Общее решение однородной системы:
\[
x_1 = \frac{21t + 3s + 4u}{13}, \quad x_2 = \frac{-38t + 2s - 6u}{13}, \quad x_3 = t, \quad x_4 = s, \quad x_5 = u.
\]

ФСР состоит из трёх векторов (по числу свободных переменных):

При $t = 1, s = 0, u = 0$:
\[
\vec{e}_1 = \left( \frac{21}{13}, -\frac{38}{13}, 1, 0, 0 \right).
\]

При $t = 0, s = 1, u = 0$:
\[
\vec{e}_2 = \left( \frac{3}{13}, \frac{2}{13}, 0, 1, 0 \right).
\]

При $t = 0, s = 0, u = 1$:
\[
\vec{e}_3 = \left( \frac{4}{13}, -\frac{6}{13}, 0, 0, 1 \right).
\]

\textbf{Ответ.}

\textit{Частное решение:}
\[
\vec{x}_{\text{ч}} = \left( -\frac{15}{13}, \frac{68}{13}, 0, 0, 0 \right).
\]

\textit{ФСР однородной системы:}
\[
\vec{e}_1 = \left( \frac{21}{13}, -\frac{38}{13}, 1, 0, 0 \right), \quad
\vec{e}_2 = \left( \frac{3}{13}, \frac{2}{13}, 0, 1, 0 \right), \quad
\vec{e}_3 = \left( \frac{4}{13}, -\frac{6}{13}, 0, 0, 1 \right).
\]

\textit{Общее решение:}
\[
\vec{x} = \vec{x}_{\text{ч}} + c_1 \vec{e}_1 + c_2 \vec{e}_2 + c_3 \vec{e}_3, \quad c_1, c_2, c_3 \in \mathbb{R}.
\]

\subsubsection{10. Ортогональное дополнение подпространства}

\textbf{Условие.} В евклидовом пространстве подпространство $L$ задано системой линейных уравнений с матрицей 
\[
\begin{pmatrix}
1 & 1 & 1 & -17 \\
2 & -13 & 0 & 1
\end{pmatrix}
\]
в ОНБ. Найти базис в ортогональном дополнении $L^\perp$ подпространства $L$.

\Solution

\textbf{Теоретическая основа.}

Пусть $L = \{x \in \R^n : Ax = 0\}$ — ядро линейного отображения, заданного матрицей $A$. Тогда:
\[
L^\perp = \text{Im}(A^T) = \text{Lin}(\text{строки } A).
\]

Это следует из того, что $x \in L$ означает $\langle a_i, x \rangle = 0$ для всех строк $a_i$ матрицы $A$, т.е. $x$ ортогонален всем строкам $A$.

\textbf{Шаг 1. Записываем строки матрицы.}

Матрица $A$ имеет строки:
\[
a_1 = (1, 1, 1, -17), \quad a_2 = (2, -13, 0, 1).
\]

\textbf{Шаг 2. Проверяем линейную независимость.}

Строки $a_1$ и $a_2$ линейно независимы (ни одна не пропорциональна другой), поэтому $\dim L^\perp = 2$.

\textbf{Шаг 3. Формируем базис.}

Базис $L^\perp$ состоит из строк матрицы $A$:
\[
e_1 = (1, 1, 1, -17), \quad e_2 = (2, -13, 0, 1).
\]

\textbf{Шаг 4 (необязательный). Ортогонализация.}

Если требуется ортогональный (или ортонормированный) базис, применяем процесс Грама–Шмидта.

Оставляем $f_1 = e_1 = (1, 1, 1, -17)$.

Вычисляем $f_2 = e_2 - \text{proj}_{f_1} e_2$:
\[
\text{proj}_{f_1} e_2 = \frac{\langle e_2, f_1 \rangle}{\langle f_1, f_1 \rangle} f_1.
\]

Вычислим скалярные произведения:
\[
\langle e_2, f_1 \rangle = 2 \cdot 1 + (-13) \cdot 1 + 0 \cdot 1 + 1 \cdot (-17) = 2 - 13 + 0 - 17 = -28.
\]
\[
\langle f_1, f_1 \rangle = 1^2 + 1^2 + 1^2 + (-17)^2 = 1 + 1 + 1 + 289 = 292.
\]

\[
\text{proj}_{f_1} e_2 = \frac{-28}{292} f_1 = -\frac{7}{73} (1, 1, 1, -17) = \left( -\frac{7}{73}, -\frac{7}{73}, -\frac{7}{73}, \frac{119}{73} \right).
\]

\[
f_2 = e_2 - \text{proj}_{f_1} e_2 = \left( 2 + \frac{7}{73}, -13 + \frac{7}{73}, 0 + \frac{7}{73}, 1 - \frac{119}{73} \right).
\]
\[
f_2 = \left( \frac{153}{73}, -\frac{942}{73}, \frac{7}{73}, -\frac{46}{73} \right) = \frac{1}{73}(153, -942, 7, -46).
\]

Можно упростить, взяв $f_2 = (153, -942, 7, -46)$.

\textbf{Ответ.}

\textit{Базис $L^\perp$:}
\[
e_1 = (1, 1, 1, -17), \quad e_2 = (2, -13, 0, 1).
\]

\textit{Ортогональный базис $L^\perp$ (если требуется):}
\[
f_1 = (1, 1, 1, -17), \quad f_2 = (153, -942, 7, -46).
\]

\textbf{Проверка.} $\dim L = 4 - 2 = 2$ (т.к. ранг матрицы $A$ равен 2). Тогда $\dim L^\perp = 4 - 2 = 2$.
