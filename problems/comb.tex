\subsection{Комбинаторика}

\subsubsection{7. Числа с нестрого убывающими цифрами}

\textbf{Условие.} Сколько существует 8-значных чисел, цифры которых расположены в порядке нестрогого убывания (то есть каждая следующая меньше либо равна предыдущей)?

\Solution

\textbf{Переформулировка задачи.}

8-значное число с нестрого убывающими цифрами — это последовательность $(d_1, d_2, \ldots, d_8)$, где:
\begin{itemize}
  \item $d_1 \in \{1, 2, \ldots, 9\}$ (первая цифра не может быть 0);
  \item $d_i \in \{0, 1, \ldots, 9\}$ для $i \geq 2$;
  \item $d_1 \geq d_2 \geq \ldots \geq d_8$.
\end{itemize}

Заметим, что каждое такое число однозначно определяется набором цифр (без учёта порядка), поскольку при нестрогом убывании порядок цифр восстанавливается однозначно.

Таким образом, нужно подсчитать количество мультимножеств из 8 цифр $\{0, 1, \ldots, 9\}$, в которых хотя бы одна цифра ненулевая (чтобы старшая цифра была $\geq 1$).

Общее число способов выбрать 8 цифр с повторениями из множества $\{0, 1, \ldots, 9\}$ (10 элементов):
\[
  C_{10 + 8 - 1}^{8} = C_{17}^{8}.
\]

Это формула для сочетаний с повторениями: выбираем 8 элементов из 10 с возможностью повторений.

Заметим, что из-за условия $d_1 \ge d_2 \ge \dots \ge d_8$ первая цифра $d_1$ равна максимуму среди выбранных цифр. Поэтому требование «первая цифра ненулевая»
эквивалентно требованию «не все 8 цифр равны нулю».

Остаётся исключить единственный набор из восьми нулей (число $00000000$ не является 8-значным). Таких случаев ровно 1, поэтому ответ равен $C_{17}^{8} - 1$.

\textbf{Ответ:} $C_{17}^{8} - 1 = 24310 - 1 = \boxed{24309}$.
