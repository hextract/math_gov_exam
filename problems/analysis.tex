\subsection{Математический анализ}

\subsubsection{1. Вычисление пределов}

\textbf{Условие.} Найти предел:
\[
\lim_{x \to 0} \frac{\sin(\tg^2 x) - \ln(1 + x^2 + x^3)}{\dfrac{\arctg(\ch x - 1)}{x} - \dfrac{x}{2}}.
\]

\Solution

Будем использовать разложения в ряд Тейлора при $x \to 0$.

\textbf{Шаг 1. Разложение числителя.}

Сначала разложим $\tg x$:
\[
\tg x = x + \frac{x^3}{3} + \frac{2x^5}{15} + o(x^5).
\]

Тогда:
\[
\tg^2 x = \left(x + \frac{x^3}{3} + o(x^4)\right)^2 = x^2 + \frac{2x^4}{3} + o(x^4).
\]

Используем разложение $\sin u = u - \frac{u^3}{6} + o(u^3)$ при $u = \tg^2 x$:
\[
\sin(\tg^2 x) = \tg^2 x - \frac{(\tg^2 x)^3}{6} + o(x^{10}) = x^2 + \frac{2x^4}{3} + o(x^5).
\]

Теперь разложим $\ln(1 + x^2 + x^3)$. Используем $\ln(1+u) = u - \frac{u^2}{2} + \frac{u^3}{3} + o(u^3)$ при $u = x^2 + x^3$:
\[
\ln(1 + x^2 + x^3) = (x^2 + x^3) - \frac{(x^2 + x^3)^2}{2} + o(x^5).
\]

Вычислим $(x^2 + x^3)^2 = x^4 + 2x^5 + x^6$:
\[
\ln(1 + x^2 + x^3) = x^2 + x^3 - \frac{x^4}{2} - x^5 + o(x^5).
\]

Числитель:
\[
\sin(\tg^2 x) - \ln(1 + x^2 + x^3) = \left(x^2 + \frac{2x^4}{3} + o(x^5)\right) - \left(x^2 + x^3 - \frac{x^4}{2} + o(x^5)\right).
\]
\[
= -x^3 + \frac{2x^4}{3} + \frac{x^4}{2} + o(x^4) = -x^3 + \frac{7x^4}{6} + o(x^4).
\]

\textbf{Шаг 2. Разложение знаменателя.}

Разложим $\ch x$:
\[
\ch x = 1 + \frac{x^2}{2} + \frac{x^4}{24} + o(x^5).
\]

Тогда $\ch x - 1 = \frac{x^2}{2} + \frac{x^4}{24} + o(x^5)$.

Используем $\arctg u = u - \frac{u^3}{3} + o(u^4)$ при $u = \ch x - 1$:
\[
\arctg(\ch x - 1) = \left(\frac{x^2}{2} + \frac{x^4}{24}\right) - \frac{1}{3}\left(\frac{x^2}{2}\right)^3 + o(x^4) = \frac{x^2}{2} + \frac{x^4}{24} + o(x^4).
\]

Делим на $x$:
\[
\frac{\arctg(\ch x - 1)}{x} = \frac{x}{2} + \frac{x^3}{24} + o(x^4).
\]

Знаменатель:
\[
\frac{\arctg(\ch x - 1)}{x} - \frac{x}{2} = \frac{x}{2} + \frac{x^3}{24} + o(x^4) - \frac{x}{2} = \frac{x^3}{24} + o(x^4).
\]

\textbf{Шаг 3. Вычисление предела.}

\[
\lim_{x \to 0} \frac{-x^3 + \frac{7x^4}{6} + o(x^4)}{\frac{x^3}{24} + o(x^4)} = \lim_{x \to 0} \frac{x^3\left(-1 + \frac{7x}{6} + o(x)\right)}{x^3\left(\frac{1}{24} + o(x)\right)} = \frac{-1}{\frac{1}{24}} = -24.
\]

\textbf{Ответ:} $-24$.

\subsubsection{2. Исследование функции}

\textbf{Условие.} Найти асимптоты, точки локального экстремума и перегиба и построить график функции:
\[
y = -3x + 1 - \frac{3}{x - 4}.
\]

\Solution

\textbf{1. Область определения.}

$D(y) = \mathbb{R} \setminus \{4\}$ (функция не определена при $x = 4$).

\textbf{2. Асимптоты.}

\textit{Вертикальная асимптота:}
\[
\lim_{x \to 4^-} y = \lim_{x \to 4^-} \left(-3x + 1 - \frac{3}{x-4}\right) = -11 - \frac{3}{0^-} = +\infty.
\]
\[
\lim_{x \to 4^+} y = -11 - \frac{3}{0^+} = -\infty.
\]

Вертикальная асимптота: $x = 4$.

\textit{Наклонная асимптота} $y = kx + b$:
\[
k = \lim_{x \to \pm\infty} \frac{y}{x} = \lim_{x \to \pm\infty} \frac{-3x + 1 - \frac{3}{x-4}}{x} = -3.
\]
\[
b = \lim_{x \to \pm\infty} (y - kx) = \lim_{x \to \pm\infty} \left(-3x + 1 - \frac{3}{x-4} + 3x\right) = \lim_{x \to \pm\infty} \left(1 - \frac{3}{x-4}\right) = 1.
\]

Наклонная асимптота: $y = -3x + 1$.

\textbf{3. Первая производная и экстремумы.}

\[
y' = -3 + \frac{3}{(x-4)^2}.
\]

Найдём критические точки ($y' = 0$):
\[
-3 + \frac{3}{(x-4)^2} = 0 \quad \Rightarrow \quad (x-4)^2 = 1 \quad \Rightarrow \quad x - 4 = \pm 1.
\]

Критические точки: $x_1 = 3$, $x_2 = 5$.

Исследуем знак $y'$:
\begin{itemize}
    \item При $x < 3$: $(x-4)^2 > 1$, поэтому $\frac{3}{(x-4)^2} < 3$, значит $y' < 0$ (убывание).
    \item При $3 < x < 4$: $(x-4)^2 < 1$, поэтому $\frac{3}{(x-4)^2} > 3$, значит $y' > 0$ (возрастание).
    \item При $4 < x < 5$: $(x-4)^2 < 1$, поэтому $y' > 0$ (возрастание).
    \item При $x > 5$: $(x-4)^2 > 1$, поэтому $y' < 0$ (убывание).
\end{itemize}

В точке $x = 3$: смена знака с $-$ на $+$ $\Rightarrow$ \textbf{локальный минимум}.
\[
y(3) = -9 + 1 - \frac{3}{-1} = -9 + 1 + 3 = -5.
\]

В точке $x = 5$: смена знака с $+$ на $-$ $\Rightarrow$ \textbf{локальный максимум}.
\[
y(5) = -15 + 1 - \frac{3}{1} = -15 + 1 - 3 = -17.
\]

\textbf{4. Вторая производная и точки перегиба.}

\[
y'' = \frac{d}{dx}\left(\frac{3}{(x-4)^2}\right) = 3 \cdot (-2)(x-4)^{-3} = -\frac{6}{(x-4)^3}.
\]

$y'' = 0$ не имеет решений, но $y''$ меняет знак при переходе через $x = 4$:
\begin{itemize}
    \item При $x < 4$: $(x-4)^3 < 0$, значит $y'' > 0$ (выпукла вниз | вогнута).
    \item При $x > 4$: $(x-4)^3 > 0$, значит $y'' < 0$ (выпукла вверх | выпукла).
\end{itemize}

Точка $x = 4$ не входит в область определения, поэтому \textbf{точек перегиба нет}.

\textbf{5. Дополнительные точки.}

Точка пересечения с осью $Oy$: $y(0) = 0 + 1 - \frac{3}{-4} = 1 + \frac{3}{4} = \frac{7}{4}$.

Точки пересечения с осью $Ox$: решаем $-3x + 1 - \frac{3}{x-4} = 0$.

Умножим на $(x-4)$: $(-3x + 1)(x - 4) - 3 = 0$.
\[
-3x^2 + 12x + x - 4 - 3 = 0 \quad \Rightarrow \quad -3x^2 + 13x - 7 = 0 \quad \Rightarrow \quad 3x^2 - 13x + 7 = 0.
\]
\[
x = \frac{13 \pm \sqrt{169 - 84}}{6} = \frac{13 \pm \sqrt{85}}{6}.
\]

$x_1 \approx 0{,}63$, $x_2 \approx 3{,}7$.

\textbf{Ответ:}
\begin{itemize}
    \item Вертикальная асимптота: $x = 4$
    \item Наклонная асимптота: $y = -3x + 1$
    \item Локальный минимум: $(3, -5)$
    \item Локальный максимум: $(5, -17)$
    \item Точек перегиба нет
\end{itemize}

\drawsomesmall{problems_2.png}

\subsubsection{3. Исследование функционального ряда}

\textbf{Условие.} Исследовать функциональный ряд $\displaystyle\sum_{n=1}^{+\infty} x e^n \arctg \frac{x}{4^n}$ на сходимость и равномерную сходимость на множествах $E_1 = (0; 1)$ и $E_2 = (1; +\infty)$.

\Solution

Обозначим $f_n(x) = x e^n \arctg \frac{x}{4^n}$.

\textbf{Асимптотика члена ряда.}

При малых значениях аргумента $\arctg u \approx u$, поэтому:
\[
\arctg \frac{x}{4^n} \approx \frac{x}{4^n} \quad \text{при } \frac{x}{4^n} \to 0.
\]

Тогда:
\[
f_n(x) \approx x e^n \cdot \frac{x}{4^n} = \frac{x^2 e^n}{4^n} = x^2 \left(\frac{e}{4}\right)^n.
\]

Поскольку $\frac{e}{4} \approx 0{,}68 < 1$, ряд $\sum \left(\frac{e}{4}\right)^n$ сходится (геометрическая прогрессия).

\textbf{1. Сходимость на $E_1 = (0; 1)$.}

Для $x \in (0, 1)$ имеем $\frac{x}{4^n} < \frac{1}{4^n}$, и:
\[
|f_n(x)| = x e^n \arctg \frac{x}{4^n} \leq x e^n \cdot \frac{x}{4^n} = \frac{x^2 e^n}{4^n} < \frac{e^n}{4^n} = \left(\frac{e}{4}\right)^n.
\]

Ряд $\sum \left(\frac{e}{4}\right)^n$ сходится, поэтому по признаку Вейерштрасса исходный ряд \textbf{сходится равномерно} на $E_1$.

\textbf{2. Сходимость на $E_2 = (1; +\infty)$.}

Для больших $x$ оценка $\arctg \frac{x}{4^n} \approx \frac{x}{4^n}$ работает только при $n$ достаточно больших.

При фиксированном $x > 1$:
\[
f_n(x) \sim \frac{x^2 e^n}{4^n} = x^2 \left(\frac{e}{4}\right)^n.
\]

Ряд $\sum x^2 \left(\frac{e}{4}\right)^n$ сходится для любого фиксированного $x$ (сумма геометрической прогрессии).

Таким образом, ряд \textbf{сходится поточечно} на $E_2$.

\textbf{Проверка равномерной сходимости на $E_2$.}

Достаточно проверить необходимое условие равномерной сходимости: если ряд $\sum f_n(x)$ сходится равномерно на $E_2$, то обязательно $f_n \to 0$ равномерно на $E_2$.

Но при любом фиксированном $n$ имеем при $x \to +\infty$:
\[
\arctg\!\left(\frac{x}{4^n}\right) \to \frac{\pi}{2}
\quad \Rightarrow \quad
f_n(x) = x e^n \arctg\!\left(\frac{x}{4^n}\right) \sim \frac{\pi}{2}\, x e^n \to +\infty.
\]
Следовательно, $\sup_{x \in (1,+\infty)} |f_n(x)| = +\infty$ для каждого $n$, и тем более $f_n$ не стремится к нулю равномерно. Значит, равномерной сходимости на $E_2$ нет.

Например, при $x_n = 4^n$:
\[
f_n(x_n) = 4^n e^n \arctg(1) = \frac{\pi}{4}(4e)^n \to +\infty.
\]

\textbf{Ответ:}
\begin{itemize}
    \item На $E_1 = (0; 1)$: ряд сходится равномерно.
    \item На $E_2 = (1; +\infty)$: ряд сходится поточечно, но не равномерно.
\end{itemize}

\subsubsection{4. Исследование несобственного интеграла}

\textbf{Условие.} Исследовать на сходимость интеграл:
\[
\int_0^{+\infty} \frac{\arctg \frac{3x}{(2+x)^4}}{\ln^3(3+x) \cdot x^{2\alpha}} \, dx.
\]

\Solution

Проведём разбиение интеграла на сумму двух интегралов по промежуткам $[0, 1]$ и $[1, +\infty)$:
\[
\int_0^{+\infty} \frac{\arctg \frac{3x}{(2+x)^4}}{\ln^3(3+x) \cdot x^{2\alpha}} \, dx = \int_0^1 \frac{\arctg \frac{3x}{(2+x)^4}}{\ln^3(3+x) \cdot x^{2\alpha}} \, dx + \int_1^{+\infty} \frac{\arctg \frac{3x}{(2+x)^4}}{\ln^3(3+x) \cdot x^{2\alpha}} \, dx.
\]
Далее будем исследовать сходимость каждого из этих интегралов отдельно.


\textbf{1. Исследование при $x \to 0^+$.}

При $x \to 0$:
\[
\arctg \frac{3x}{(2+x)^4} \sim \frac{3x}{16} \quad (\text{т.к. } (2+x)^4 \to 16).
\]
\[
\ln^3(3+x) \to \ln^3 3 \neq 0.
\]

Подынтегральное выражение ведёт себя как:
\[
\frac{\frac{3x}{16}}{\ln^3 3 \cdot x^{2\alpha}} = \frac{3}{16 \ln^3 3} \cdot x^{1 - 2\alpha}.
\]

Интеграл $\int_0^1 x^{1-2\alpha} dx$ сходится при $1 - 2\alpha > -1$, т.е. при $\alpha < 1$.

\textbf{2. Исследование при $x \to +\infty$.}

При $x \to +\infty$:
\[
\frac{3x}{(2+x)^4} \sim \frac{3x}{x^4} = \frac{3}{x^3} \to 0.
\]
\[
\arctg \frac{3x}{(2+x)^4} \sim \frac{3x}{(2+x)^4} \sim \frac{3}{x^3}.
\]
\[
\ln^3(3+x) \sim \ln^3 x.
\]

Подынтегральное выражение ведёт себя как:
\[
\frac{\frac{3}{x^3}}{\ln^3 x \cdot x^{2\alpha}} = \frac{3}{x^{3 + 2\alpha} \ln^3 x}.
\]

Интеграл $\int_1^{+\infty} \frac{dx}{x^{3+2\alpha} \ln^3 x}$ сходится при $3 + 2\alpha > 1$, т.е. при $\alpha > -1$.

Кроме того, при $\alpha = -1$ имеем $\int_1^{+\infty} \frac{dx}{x \ln^3 x}$, и этот интеграл тоже сходится (степень логарифма $>1$).

\textbf{3. Общий критерий сходимости.}

Объединяя условия:
\[
\alpha < 1 \quad \text{и} \quad \alpha \ge -1.
\]

\textbf{Ответ:} Интеграл сходится при $\boxed{-1 \le \alpha < 1}$.


