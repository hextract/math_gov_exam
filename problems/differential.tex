\subsection{Дифференциальные уравнения}


\subsubsection{Уравнения с разделяющимися переменными}

\textbf{Общий вид}

\begin{equation*}
    \frac{dy}{dx} = f(x)g(y)
\end{equation*}

\textbf{Примеры}

Решить уравнение:

\begin{equation*}
    \frac{dy}{dx} = y \sin x
\end{equation*}

Решение:

\begin{align*}
    \frac{dy}{y} &= \sin x \, dx \\
    \int \frac{1}{y} \, dy &= \int \sin x \, dx \\
    \ln |y| &= -\cos x + C \\
    y &= (\pm e^{C}) e^{-\cos x} = C_1 e^{-\cos x}
\end{align*}

\subsubsection{Однородные уравнения 1-го порядка}

\textbf{Общий вид}

\begin{equation*}
    \frac{dy}{dx} = F\left(\frac{y}{x}\right)
\end{equation*}

\textbf{Примеры}

Решить уравнение:

\begin{equation*}
    \frac{dy}{dx} = \frac{x+y}{x}
\end{equation*}

Решение:

\begin{align*}
    \frac{dy}{dx} &= \frac{x + y}{x} \\
    \text{Замена: } y = vx \quad &\Rightarrow \quad \frac{dy}{dx} = v + x\frac{dv}{dx} \\
    v + x \frac{dv}{dx} &= \frac{x + vx}{x} = 1 + v \\
    x \frac{dv}{dx} &= 1 + v - v = 1 \\
    \frac{dv}{dx} &= \frac{1}{x} \\
\end{align*}

Свели к уравнению с разделяющимися:

\begin{align*}
    \int dv &= \int \frac{1}{x} dx \\
    v &= \ln|x| + C \\
    y &= vx = x(\ln|x| + C)
\end{align*}

\subsubsection{Уравнения в полных дифференциалах}

\textbf{Общий вид}

\begin{equation*}
    M(x,y)dx + N(x,y)dy = 0
\end{equation*}

где \(\frac{\partial M}{\partial y} = \frac{\partial N}{\partial x}\) — условие полной дифференциальности.

\textbf{Примеры}

Решить уравнение:

\begin{equation*}
(2xy + 1)dx + (x^2 + 3y^2)dy = 0
\end{equation*}

Решение:

Проверим условие полной дифференциальности:

\begin{align*}
    \frac{\partial M}{\partial y} &= \frac{\partial}{\partial y}(2xy + 1) = 2x \\
    \frac{\partial N}{\partial x} &= \frac{\partial}{\partial x}(x^2 + 3y^2) = 2x
\end{align*}

Условие выполнено, уравнение полное.

Ищем функцию $F(x,y)$, такую что $\frac{\partial F}{\partial x} = M; \frac{\partial F}{\partial y} = N$.

\begin{align*}
    \begin{cases}
        \frac{\partial F}{\partial x} &= 2xy + 1 \quad \Rightarrow \quad F(x,y) = x^2 y + x + \phi(y) \\
        \\
        \frac{\partial F}{\partial y} &= x^2 + 3y^2 \quad \Rightarrow \quad \frac{\partial}{\partial y}(x^2 y + x + \phi(y)) = x^2 + \phi'(y) = x^2 + 3y^2 \\
    \end{cases}
\end{align*}

\begin{align*}
    \phi'(y) &= 3y^2 \quad \Rightarrow \quad \phi(y) = y^3\\
\end{align*}

Следовательно, $F(x,y) = x^2 y + x + y^3$.

Общее решение: $F(x,y) = C \quad \Rightarrow \quad x^2 y + x + y^3 = C$.

\subsubsection{Линейные уравнения 1-го порядка}

\textbf{Общий вид}

\begin{equation*}
    \frac{dy}{dx} + P(x)y = Q(x)
\end{equation*}

\textbf{Примеры}

Решить уравнение:

\begin{equation*}
    y' + y = e^x
\end{equation*}

Найти интегрирующий множитель -- такую функцию, при умножении уравнения на которую левая часть станет полной производной:

\begin{equation*}
    \frac{d}{dx} [\mu(x)y] = \mu(x) Q(x)
\end{equation*}

Для данного уравнения интегрирующий множитель $e^x$:

\begin{align*}
    e^x \frac{dy}{dx} + e^x y &= e^{2x} \\
    \frac{d}{dx}(y e^x) &= e^{2x}
\end{align*}

Интегрируем обе части:

\begin{align*}
    \int \frac{d}{dx}(y e^x) dx &= \int e^{2x} dx \\
    y e^x &= \frac{1}{2} e^{2x} + C
\end{align*}

Выражаем $y$:

\begin{align*}
    y &= \frac{1}{2} e^x + C e^{-x}
\end{align*}

\subsubsection{Линейные уравнения второго порядка с постоянными коэффициентами}

\textbf{Общий вид}

\begin{equation*}
    y'' + a y' + b y = f(x)
\end{equation*}

\textbf{Пример}

Решить уравнение:

\begin{equation*}
    y'' - 3y' + 2y = e^{2x}
\end{equation*}

Решаем однородное уравнение:

\begin{align*}
    y_h'' - 3y_h' + 2y_h &= 0 \\ \\
    r^2 - 3r + 2 &= 0 \\
    r_1 = 1, &\quad r_2 = 2 \\
    y_h = C_1 e^x &+ C_2 e^{2x}
\end{align*}

Ищем частное решение в виде:

\begin{equation*}
    y_p = u_1(x) e^x + u_2(x) e^{2x}
\end{equation*}

где \(u_1(x)\) и \(u_2(x)\) — функции, которые нужно определить.

Составляем систему для \(u_1'\) и \(u_2'\):

\begin{align*}
    u_1' e^x + u_2' e^{2x} &= 0 \\
    u_1' e^x + 2 u_2' e^{2x} &= e^{2x}
\end{align*}

Решение системы:

\begin{align*}
    u_2' &= 1 \quad \Rightarrow \quad u_2 = \int 1 dx = x \\
    u_1' &= - u_2' e^x = - e^x \quad \Rightarrow \quad u_1 = \int - e^x dx = - e^x
\end{align*}

Итак: частное решение

\begin{align*}
    y_p &= u_1 e^x + u_2 e^{2x} \\
    y_p &= (- e^x) e^x + x e^{2x} \\
    y_p &= - e^{2x} + x e^{2x} = (x-1) e^{2x}
\end{align*}

Общее решение:

\begin{align*}
    y &= y_h + y_p \\
    y &= C_1 e^x + C_2 e^{2x} + (x-1) e^{2x}
\end{align*}

(Можно переписать как $y = C_1 e^x + C_3 e^{2x} + xe^{2x}$)

\subsubsection{Однородное линейное уравнение с корнем кратности 3}

Решить уравнение:

\begin{equation*}
    y''' - 6y'' + 12y' - 8y = 0
\end{equation*}

Характеристическое уравнение:

\begin{align*}
    r^3 - 6r^2 + 12r - 8 &= 0 \\
    (r - 2)^3 &= 0 \\
    r_1 = r_2 = r_3 &= 2
\end{align*}

Общее решение:

\begin{align*}
    y_h &= (C_1 + C_2 x + C_3 x^2) e^{r x} \\
    y_h &= (C_1 + C_2 x + C_3 x^2) e^{2x}
\end{align*}

\subsubsection{Система линейных дифференциальных уравнений}

Решить систему:

\begin{equation*}
    \begin{cases}
        x'(t) = 2x + y \\
        y'(t) = x + 2y + z \\
        z'(t) = y + 2z
    \end{cases}
\end{equation*}

\textbf{Решение}

Сначала записываем систему в матричной форме:

\begin{equation*}
    \begin{pmatrix} x' \\ y' \\ z' \end{pmatrix} =
    \begin{pmatrix}
        2 & 1 & 0 \\
        1 & 2 & 1 \\
        0 & 1 & 2
    \end{pmatrix}
    \begin{pmatrix} x \\ y \\ z \end{pmatrix}
\end{equation*}

Находим собственные значения матрицы \(A\) из характеристического уравнения:

\begin{align*}
    \det(A - \lambda I) &=
    \begin{vmatrix}
        2-\lambda & 1 & 0 \\
        1 & 2-\lambda & 1 \\
        0 & 1 & 2-\lambda
    \end{vmatrix} = 0
\end{align*}

Вычисляя определитель, получаем:

\begin{equation*}
(2-\lambda)\big((2-\lambda)^2 - 1\big) - 1\big(1\cdot(2-\lambda) - 1\cdot 0\big) + 0 = 0
\end{equation*}

\begin{equation*}
(2-\lambda)\big((2-\lambda)^2 - 1\big) - (2-\lambda) = 0
\end{equation*}

\begin{equation*}
(2-\lambda)\big((2-\lambda)^2 - 2\big) = 0
\end{equation*}

Отсюда собственные значения:

\begin{equation*}
    \lambda_1 = 2-\sqrt{2}, \quad \lambda_2 = 2, \quad \lambda_3 = 2+\sqrt{2}
\end{equation*}

Для каждого собственного значения находим собственный вектор \(v_i\):

\begin{align*}
    \lambda_1 = 2-\sqrt{2}: & \quad v_1 = \begin{pmatrix} 1 \\ -\sqrt{2} \\ 1 \end{pmatrix} \\
    \lambda_2 = 2: & \quad v_2 = \begin{pmatrix} 1 \\ 0 \\ -1 \end{pmatrix} \\
    \lambda_3 = 2+\sqrt{2}: & \quad v_3 = \begin{pmatrix} 1 \\ \sqrt{2} \\ 1 \end{pmatrix}
\end{align*}

Общее решение системы записывается как линейная комбинация собственных решений:

\begin{equation*}
    \begin{pmatrix} x(t) \\ y(t) \\ z(t) \end{pmatrix} =
    C_1
    \begin{pmatrix} 1 \\ -\sqrt{2} \\ 1 \end{pmatrix} e^{(2-\sqrt{2})t} +
    C_2
    \begin{pmatrix} 1 \\ 0 \\ -1 \end{pmatrix} e^{2t} +
    C_3
    \begin{pmatrix} 1 \\ \sqrt{2} \\ 1 \end{pmatrix} e^{(2+\sqrt{2})t}
\end{equation*}

где \(C_1, C_2, C_3\) — произвольные константы.

\subsubsection{Система линейных дифференциальных уравнений с комплексными собственными значениями}

Решить систему:

\begin{equation*}
    \begin{cases}
        x'(t) = -y \\
        y'(t) = x
    \end{cases}
\end{equation*}

Решение:

Сначала записываем систему в матричной форме:

\begin{equation*}
    \begin{pmatrix} x' \\ y' \end{pmatrix} =
    \begin{pmatrix}
        0 & -1 \\
        1 & 0
    \end{pmatrix}
    \begin{pmatrix} x \\ y \end{pmatrix}
\end{equation*}

Находим собственные значения матрицы \(A\):

\begin{align*}
    \det(A - \lambda I) &=
    \begin{vmatrix}
        -\lambda & -1 \\
        1 & -\lambda
    \end{vmatrix} = \lambda^2 + 1 = 0
\end{align*}

Отсюда:

\begin{equation*}
    \lambda_{1,2} = \pm i
\end{equation*}

Для \(\lambda = i\) находим собственный вектор \(v_1 = \begin{pmatrix} 1 \\ -i \end{pmatrix}\), для \(\lambda = -i\) — \(v_2 = \begin{pmatrix} 1 \\ i \end{pmatrix}\).

Общее комплексное решение:

\begin{equation*}
    \begin{pmatrix} x(t) \\ y(t) \end{pmatrix} = C_1
    \begin{pmatrix} 1 \\ -i \end{pmatrix} e^{i t} +
    C_2
    \begin{pmatrix} 1 \\ i \end{pmatrix} e^{-i t}
\end{equation*}

Овеществление решения:

Используем формулу Эйлера \(e^{i t} = \cos t + i \sin t\) и разложим комплексные экспоненты:

\begin{align*}
    C_1
    \begin{pmatrix} 1 \\ -i \end{pmatrix} e^{i t} +
    C_2
    \begin{pmatrix} 1 \\ i \end{pmatrix} e^{-i t}
    &= C_1
    \begin{pmatrix} 1 \\ -i \end{pmatrix} (\cos t + i \sin t) +
    C_2
    \begin{pmatrix} 1 \\ i \end{pmatrix} (\cos t - i \sin t) \\
    &= (C_1 + C_2)
    \begin{pmatrix} \cos t \\ \sin t \end{pmatrix} +
    i(C_1 - C_2)
    \begin{pmatrix} \sin t \\ -\cos t \end{pmatrix}
\end{align*}

Вводим новые вещественные константы \(A = C_1 + C_2\), \(B = i(C_1 - C_2)\), получаем вещественное решение:

\begin{align*}
    x(t) &= A \cos t + B \sin t \\
    y(t) &= A \sin t - B \cos t
\end{align*}

\subsubsection{Система второго порядка с кратным вещественным корнем}

Рассмотрим систему:

\begin{equation*}
    \begin{cases}
        x'(t) = 2x + y \\
        y'(t) = -4x - 2y
    \end{cases}
\end{equation*}

Сначала записываем систему в матричной форме:

\begin{equation*}
    \begin{pmatrix} x' \\ y' \end{pmatrix} =
    \begin{pmatrix}
        2 & 1 \\
        -4 & -2
    \end{pmatrix}
    \begin{pmatrix} x \\ y \end{pmatrix}
\end{equation*}

Находим собственные значения матрицы \(A\):

\begin{align*}
    \det(A - \lambda I) &=
    \begin{vmatrix}
        2-\lambda & 1 \\
        -4 & -2-\lambda
    \end{vmatrix}
    = (2-\lambda)(-2-\lambda) - (-4) = \lambda^2 = 0
\end{align*}

Отсюда:

\begin{equation*}
    \lambda = 0 \quad \text{(кратность 2)}
\end{equation*}

Находим обычный собственный вектор \(v_1\):

\begin{equation*}
(A - 0 I)v_1 = 0 \quad \Rightarrow \quad v_1 = \begin{pmatrix} 1 \\ -2 \end{pmatrix}
\end{equation*}

Так как кратность корня = 2, а собственный вектор один, находим обобщённый собственный вектор \(v_2\):

\begin{equation*}
(A - 0 I)v_2 = v_1 \quad \Rightarrow \quad v_2 = \begin{pmatrix} 0 \\ 1 \end{pmatrix}
\end{equation*}

Общее решение системы с учётом кратного корня и цепочки обобщённых векторов:

\begin{equation*}
    \begin{pmatrix} x(t) \\ y(t) \end{pmatrix} =
    C_1 v_1 + C_2 (v_1 t + v_2)
    = C_1 \begin{pmatrix} 1 \\ -2 \end{pmatrix} +
    C_2 \left( t \begin{pmatrix} 1 \\ -2 \end{pmatrix} + \begin{pmatrix} 0 \\ 1 \end{pmatrix} \right)
\end{equation*}

В раскрытом виде по компонентам:

\begin{align*}
    x(t) &= C_1 + C_2 t \\
    y(t) &= -2 C_1 - 2 C_2 t + C_2 = -2 C_1 + C_2 (1 - 2 t)
\end{align*}

где \(C_1, C_2\) — произвольные константы.

\Note
Если система $X' = A X$ имеет вещественный собственный корень \(\lambda\) кратности \(k\), а собственный вектор всего 1, то необходимо строить цепочку обобщённых собственных векторов:

\begin{align*}
(A - \lambda I)v_1 &= 0, \\
(A - \lambda I)v_2 &= v_1, \\
(A - \lambda I)v_3 &= v_2, \\
&\dots \\
(A - \lambda I)v_k &= v_{k-1}.
\end{align*}

Общее решение, соответствующее кратному корню \(\lambda\), имеет вид:
\begin{equation*}
    X(t) = \Big(C_1 v_1 + C_2 (v_1 t + v_2) + C_3 \big(v_1 \frac{t^2}{2} + v_2 t + v_3\big) + \dots + C_k \big(v_1 \frac{t^{k-1}}{(k-1)!} + v_2 \frac{t^{k-2}}{(k-2)!} + \dots + v_k\big)\Big) e^{\lambda t},
\end{equation*}
где \(C_1, C_2, \dots, C_k\) — произвольные константы.

\subsubsection{15. Автономная система: положение равновесия и фазовый портрет}

\textbf{Условие.} Найти все действительные решения системы дифференциальных уравнений
\[
\begin{cases}
\dot{x} = 4x + 3y \\
\dot{y} = 3x + 4y
\end{cases}
\]
Найти положение равновесия автономной системы, определить его характер и нарисовать фазовые траектории системы в окрестности положения равновесия.

\Solution

\textbf{Часть 1. Нахождение общего решения.}

Записываем систему в матричной форме:
\[
\begin{pmatrix} \dot{x} \\ \dot{y} \end{pmatrix} = 
\begin{pmatrix} 4 & 3 \\ 3 & 4 \end{pmatrix}
\begin{pmatrix} x \\ y \end{pmatrix}
\]

\textit{Шаг 1. Находим собственные значения.}

Характеристическое уравнение:
\[
\det(A - \lambda I) = 
\begin{vmatrix} 4 - \lambda & 3 \\ 3 & 4 - \lambda \end{vmatrix} 
= (4 - \lambda)^2 - 9 = \lambda^2 - 8\lambda + 16 - 9 = \lambda^2 - 8\lambda + 7 = 0.
\]

Корни: $\lambda_{1,2} = \frac{8 \pm \sqrt{64 - 28}}{2} = \frac{8 \pm 6}{2}$.

\[
\lambda_1 = 7, \quad \lambda_2 = 1.
\]

\textit{Шаг 2. Находим собственные векторы.}

Для $\lambda_1 = 7$:
\[
(A - 7I)v = 0 \Rightarrow 
\begin{pmatrix} -3 & 3 \\ 3 & -3 \end{pmatrix} v = 0.
\]

Уравнение: $-3v_1 + 3v_2 = 0 \Rightarrow v_1 = v_2$.

Собственный вектор: $v_1 = \begin{pmatrix} 1 \\ 1 \end{pmatrix}$.

Для $\lambda_2 = 1$:
\[
(A - I)v = 0 \Rightarrow 
\begin{pmatrix} 3 & 3 \\ 3 & 3 \end{pmatrix} v = 0.
\]

Уравнение: $3v_1 + 3v_2 = 0 \Rightarrow v_1 = -v_2$.

Собственный вектор: $v_2 = \begin{pmatrix} 1 \\ -1 \end{pmatrix}$.

\textit{Шаг 3. Записываем общее решение.}

\[
\begin{pmatrix} x(t) \\ y(t) \end{pmatrix} = 
C_1 \begin{pmatrix} 1 \\ 1 \end{pmatrix} e^{7t} + 
C_2 \begin{pmatrix} 1 \\ -1 \end{pmatrix} e^{t}
\]

Или в координатной форме:
\[
x(t) = C_1 e^{7t} + C_2 e^{t}, \quad y(t) = C_1 e^{7t} - C_2 e^{t}.
\]

\textbf{Часть 2. Положение равновесия.}

Положение равновесия — это точка, в которой $\dot{x} = \dot{y} = 0$:
\[
\begin{cases}
4x + 3y = 0 \\
3x + 4y = 0
\end{cases}
\]

Определитель системы: $\det A = 16 - 9 = 7 \neq 0$, значит единственное решение — $(x, y) = (0, 0)$.

\textbf{Положение равновесия: $(0, 0)$.}

\textbf{Часть 3. Характер положения равновесия.}

Характер определяется собственными значениями матрицы $A$:
\begin{itemize}
    \item $\lambda_1 = 7 > 0$
    \item $\lambda_2 = 1 > 0$
\end{itemize}

Оба собственных значения положительные и различные, поэтому положение равновесия — \textbf{неустойчивый узел}.

\textbf{Часть 4. Фазовый портрет.}

Введём новые координаты $(\xi, \eta)$ вдоль собственных векторов:
\[
\xi = x + y, \quad \eta = x - y.
\]

Тогда:
\[
\dot{\xi} = \dot{x} + \dot{y} = (4x + 3y) + (3x + 4y) = 7(x + y) = 7\xi,
\]
\[
\dot{\eta} = \dot{x} - \dot{y} = (4x + 3y) - (3x + 4y) = x - y = \eta.
\]

В новых координатах система расцеплена:
\[
\dot{\xi} = 7\xi, \quad \dot{\eta} = \eta.
\]

Решения: $\xi = \xi_0 e^{7t}$, $\eta = \eta_0 e^{t}$.

Исключаем параметр $t$:
\[
t = \ln\left(\frac{\eta}{\eta_0}\right), \quad \xi = \xi_0 e^{7\ln(\eta/\eta_0)} = \xi_0 \left(\frac{\eta}{\eta_0}\right)^7.
\]

\[
\xi = C \cdot \eta^7, \quad \text{где } C = \frac{\xi_0}{\eta_0^7}.
\]

\drawsomemedium{pix/problems_15.png}

\textit{Общее решение:}
\[
x(t) = C_1 e^{7t} + C_2 e^{t}, \quad y(t) = C_1 e^{7t} - C_2 e^{t}.
\]

\textit{Положение равновесия:} $(0, 0)$.

\textit{Характер:} Неустойчивый узел ($\lambda_1 = 7 > \lambda_2 = 1 > 0$).

\textit{Фазовый портрет:} Траектории выходят из начала координат, касаясь «медленного» направления $y = -x$, и уходят на бесконечность вдоль «быстрого» направления $y = x$.

\subsubsection{16. Составление ОДУ по частным решениям}

\textbf{Условие.} Составить однородное дифференциальное уравнение с постоянными коэффициентами наименьшего порядка, имеющее частные решения: $y_1 = e^x$, $y_2 = x \sin 2x$.

\Solution

\textbf{Теоретическая основа.}

Для линейного однородного ОДУ с постоянными коэффициентами:
\[
y^{(n)} + a_{n-1} y^{(n-1)} + \ldots + a_1 y' + a_0 y = 0
\]

решения определяются корнями характеристического уравнения:
\[
\lambda^n + a_{n-1} \lambda^{n-1} + \ldots + a_1 \lambda + a_0 = 0.
\]

\begin{itemize}
    \item Простому вещественному корню $\lambda$ соответствует решение $e^{\lambda x}$.
    \item Кратному вещественному корню $\lambda$ кратности $k$ соответствуют решения $e^{\lambda x}, x e^{\lambda x}, \ldots, x^{k-1} e^{\lambda x}$.
    \item Паре комплексно сопряжённых корней $\alpha \pm i\beta$ соответствуют решения $e^{\alpha x} \cos \beta x$, $e^{\alpha x} \sin \beta x$.
    \item Кратной паре $\alpha \pm i\beta$ кратности $k$ соответствуют решения $x^j e^{\alpha x} \cos \beta x$, $x^j e^{\alpha x} \sin \beta x$ для $j = 0, 1, \ldots, k-1$.
\end{itemize}

\textbf{Шаг 1. Анализ решения $y_1 = e^x$.}

Решение $e^x$ соответствует простому корню $\lambda_1 = 1$.

\textbf{Шаг 2. Анализ решения $y_2 = x \sin 2x$.}

Решение $x \sin 2x$ имеет вид $x \cdot e^{0 \cdot x} \sin 2x$, т.е. $\alpha = 0$, $\beta = 2$, и множитель $x$ означает кратность не менее 2.

Следовательно, $\lambda_{2,3} = \pm 2i$ — комплексно сопряжённая пара кратности 2.

Этому соответствуют 4 линейно независимых решения:
\[
\cos 2x, \quad \sin 2x, \quad x \cos 2x, \quad x \sin 2x.
\]

\textbf{Шаг 3. Минимальное характеристическое уравнение.}

Характеристическое уравнение должно иметь корни:
\begin{itemize}
    \item $\lambda = 1$ (простой);
    \item $\lambda = 2i$ (кратность 2);
    \item $\lambda = -2i$ (кратность 2).
\end{itemize}

Характеристический многочлен:
\[
p(\lambda) = (\lambda - 1) \cdot (\lambda - 2i)^2 \cdot (\lambda + 2i)^2 = (\lambda - 1) \cdot \left[(\lambda - 2i)(\lambda + 2i)\right]^2.
\]

Вычисляем:
\[
(\lambda - 2i)(\lambda + 2i) = \lambda^2 + 4.
\]

\[
p(\lambda) = (\lambda - 1)(\lambda^2 + 4)^2.
\]

Раскроем $(\lambda^2 + 4)^2$:
\[
(\lambda^2 + 4)^2 = \lambda^4 + 8\lambda^2 + 16.
\]

Теперь умножаем на $(\lambda - 1)$:
\[
p(\lambda) = (\lambda - 1)(\lambda^4 + 8\lambda^2 + 16).
\]

\[
p(\lambda) = \lambda^5 + 8\lambda^3 + 16\lambda - \lambda^4 - 8\lambda^2 - 16.
\]

\[
p(\lambda) = \lambda^5 - \lambda^4 + 8\lambda^3 - 8\lambda^2 + 16\lambda - 16.
\]

\textbf{Шаг 4. Записываем дифференциальное уравнение.}

Дифференциальное уравнение получается заменой $\lambda^k \to y^{(k)}$:

\[
y^{(5)} - y^{(4)} + 8y''' - 8y'' + 16y' - 16y = 0.
\]

\textbf{Проверка.}

\textit{Проверим $y_1 = e^x$:}
\[
y^{(k)} = e^x \text{ для всех } k.
\]
\[
e^x - e^x + 8e^x - 8e^x + 16e^x - 16e^x = 0. \quad \checkmark
\]

\textit{Проверим $y_2 = x \sin 2x$:}

Вычислим производные:
\begin{align*}
y &= x \sin 2x \\
y' &= \sin 2x + 2x \cos 2x \\
y'' &= 2\cos 2x + 2\cos 2x - 4x\sin 2x = 4\cos 2x - 4x\sin 2x \\
y''' &= -8\sin 2x - 4\sin 2x - 8x\cos 2x = -12\sin 2x - 8x\cos 2x \\
y^{(4)} &= -24\cos 2x - 8\cos 2x + 16x\sin 2x = -32\cos 2x + 16x\sin 2x \\
y^{(5)} &= 64\sin 2x + 16\sin 2x + 32x\cos 2x = 80\sin 2x + 32x\cos 2x
\end{align*}

Подставляем:
\begin{align*}
&y^{(5)} - y^{(4)} + 8y''' - 8y'' + 16y' - 16y = \\
&= (80\sin 2x + 32x\cos 2x) - (-32\cos 2x + 16x\sin 2x) \\
&\quad + 8(-12\sin 2x - 8x\cos 2x) - 8(4\cos 2x - 4x\sin 2x) \\
&\quad + 16(\sin 2x + 2x\cos 2x) - 16(x\sin 2x) \\
&= 80\sin 2x + 32x\cos 2x + 32\cos 2x - 16x\sin 2x \\
&\quad - 96\sin 2x - 64x\cos 2x - 32\cos 2x + 32x\sin 2x \\
&\quad + 16\sin 2x + 32x\cos 2x - 16x\sin 2x \\
&= (80 - 96 + 16)\sin 2x + (32 - 64 + 32)x\cos 2x \\
&\quad + (32 - 32)\cos 2x + (-16 + 32 - 16)x\sin 2x \\
&= 0 \cdot \sin 2x + 0 \cdot x\cos 2x + 0 \cdot \cos 2x + 0 \cdot x\sin 2x = 0. \quad \checkmark
\end{align*}

\textbf{Ответ.}

Минимальный порядок уравнения: $n = 5$.

Уравнение:
\[
\boxed{y^{(5)} - y^{(4)} + 8y''' - 8y'' + 16y' - 16y = 0.}
\]
